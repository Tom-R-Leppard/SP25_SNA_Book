% Options for packages loaded elsewhere
\PassOptionsToPackage{unicode}{hyperref}
\PassOptionsToPackage{hyphens}{url}
\PassOptionsToPackage{dvipsnames,svgnames,x11names}{xcolor}
%
\documentclass[
  letterpaper,
  DIV=11,
  numbers=noendperiod]{scrreprt}

\usepackage{amsmath,amssymb}
\usepackage{iftex}
\ifPDFTeX
  \usepackage[T1]{fontenc}
  \usepackage[utf8]{inputenc}
  \usepackage{textcomp} % provide euro and other symbols
\else % if luatex or xetex
  \usepackage{unicode-math}
  \defaultfontfeatures{Scale=MatchLowercase}
  \defaultfontfeatures[\rmfamily]{Ligatures=TeX,Scale=1}
\fi
\usepackage{lmodern}
\ifPDFTeX\else  
    % xetex/luatex font selection
\fi
% Use upquote if available, for straight quotes in verbatim environments
\IfFileExists{upquote.sty}{\usepackage{upquote}}{}
\IfFileExists{microtype.sty}{% use microtype if available
  \usepackage[]{microtype}
  \UseMicrotypeSet[protrusion]{basicmath} % disable protrusion for tt fonts
}{}
\makeatletter
\@ifundefined{KOMAClassName}{% if non-KOMA class
  \IfFileExists{parskip.sty}{%
    \usepackage{parskip}
  }{% else
    \setlength{\parindent}{0pt}
    \setlength{\parskip}{6pt plus 2pt minus 1pt}}
}{% if KOMA class
  \KOMAoptions{parskip=half}}
\makeatother
\usepackage{xcolor}
\setlength{\emergencystretch}{3em} % prevent overfull lines
\setcounter{secnumdepth}{5}
% Make \paragraph and \subparagraph free-standing
\makeatletter
\ifx\paragraph\undefined\else
  \let\oldparagraph\paragraph
  \renewcommand{\paragraph}{
    \@ifstar
      \xxxParagraphStar
      \xxxParagraphNoStar
  }
  \newcommand{\xxxParagraphStar}[1]{\oldparagraph*{#1}\mbox{}}
  \newcommand{\xxxParagraphNoStar}[1]{\oldparagraph{#1}\mbox{}}
\fi
\ifx\subparagraph\undefined\else
  \let\oldsubparagraph\subparagraph
  \renewcommand{\subparagraph}{
    \@ifstar
      \xxxSubParagraphStar
      \xxxSubParagraphNoStar
  }
  \newcommand{\xxxSubParagraphStar}[1]{\oldsubparagraph*{#1}\mbox{}}
  \newcommand{\xxxSubParagraphNoStar}[1]{\oldsubparagraph{#1}\mbox{}}
\fi
\makeatother

\usepackage{color}
\usepackage{fancyvrb}
\newcommand{\VerbBar}{|}
\newcommand{\VERB}{\Verb[commandchars=\\\{\}]}
\DefineVerbatimEnvironment{Highlighting}{Verbatim}{commandchars=\\\{\}}
% Add ',fontsize=\small' for more characters per line
\usepackage{framed}
\definecolor{shadecolor}{RGB}{241,243,245}
\newenvironment{Shaded}{\begin{snugshade}}{\end{snugshade}}
\newcommand{\AlertTok}[1]{\textcolor[rgb]{0.68,0.00,0.00}{#1}}
\newcommand{\AnnotationTok}[1]{\textcolor[rgb]{0.37,0.37,0.37}{#1}}
\newcommand{\AttributeTok}[1]{\textcolor[rgb]{0.40,0.45,0.13}{#1}}
\newcommand{\BaseNTok}[1]{\textcolor[rgb]{0.68,0.00,0.00}{#1}}
\newcommand{\BuiltInTok}[1]{\textcolor[rgb]{0.00,0.23,0.31}{#1}}
\newcommand{\CharTok}[1]{\textcolor[rgb]{0.13,0.47,0.30}{#1}}
\newcommand{\CommentTok}[1]{\textcolor[rgb]{0.37,0.37,0.37}{#1}}
\newcommand{\CommentVarTok}[1]{\textcolor[rgb]{0.37,0.37,0.37}{\textit{#1}}}
\newcommand{\ConstantTok}[1]{\textcolor[rgb]{0.56,0.35,0.01}{#1}}
\newcommand{\ControlFlowTok}[1]{\textcolor[rgb]{0.00,0.23,0.31}{\textbf{#1}}}
\newcommand{\DataTypeTok}[1]{\textcolor[rgb]{0.68,0.00,0.00}{#1}}
\newcommand{\DecValTok}[1]{\textcolor[rgb]{0.68,0.00,0.00}{#1}}
\newcommand{\DocumentationTok}[1]{\textcolor[rgb]{0.37,0.37,0.37}{\textit{#1}}}
\newcommand{\ErrorTok}[1]{\textcolor[rgb]{0.68,0.00,0.00}{#1}}
\newcommand{\ExtensionTok}[1]{\textcolor[rgb]{0.00,0.23,0.31}{#1}}
\newcommand{\FloatTok}[1]{\textcolor[rgb]{0.68,0.00,0.00}{#1}}
\newcommand{\FunctionTok}[1]{\textcolor[rgb]{0.28,0.35,0.67}{#1}}
\newcommand{\ImportTok}[1]{\textcolor[rgb]{0.00,0.46,0.62}{#1}}
\newcommand{\InformationTok}[1]{\textcolor[rgb]{0.37,0.37,0.37}{#1}}
\newcommand{\KeywordTok}[1]{\textcolor[rgb]{0.00,0.23,0.31}{\textbf{#1}}}
\newcommand{\NormalTok}[1]{\textcolor[rgb]{0.00,0.23,0.31}{#1}}
\newcommand{\OperatorTok}[1]{\textcolor[rgb]{0.37,0.37,0.37}{#1}}
\newcommand{\OtherTok}[1]{\textcolor[rgb]{0.00,0.23,0.31}{#1}}
\newcommand{\PreprocessorTok}[1]{\textcolor[rgb]{0.68,0.00,0.00}{#1}}
\newcommand{\RegionMarkerTok}[1]{\textcolor[rgb]{0.00,0.23,0.31}{#1}}
\newcommand{\SpecialCharTok}[1]{\textcolor[rgb]{0.37,0.37,0.37}{#1}}
\newcommand{\SpecialStringTok}[1]{\textcolor[rgb]{0.13,0.47,0.30}{#1}}
\newcommand{\StringTok}[1]{\textcolor[rgb]{0.13,0.47,0.30}{#1}}
\newcommand{\VariableTok}[1]{\textcolor[rgb]{0.07,0.07,0.07}{#1}}
\newcommand{\VerbatimStringTok}[1]{\textcolor[rgb]{0.13,0.47,0.30}{#1}}
\newcommand{\WarningTok}[1]{\textcolor[rgb]{0.37,0.37,0.37}{\textit{#1}}}

\providecommand{\tightlist}{%
  \setlength{\itemsep}{0pt}\setlength{\parskip}{0pt}}\usepackage{longtable,booktabs,array}
\usepackage{calc} % for calculating minipage widths
% Correct order of tables after \paragraph or \subparagraph
\usepackage{etoolbox}
\makeatletter
\patchcmd\longtable{\par}{\if@noskipsec\mbox{}\fi\par}{}{}
\makeatother
% Allow footnotes in longtable head/foot
\IfFileExists{footnotehyper.sty}{\usepackage{footnotehyper}}{\usepackage{footnote}}
\makesavenoteenv{longtable}
\usepackage{graphicx}
\makeatletter
\newsavebox\pandoc@box
\newcommand*\pandocbounded[1]{% scales image to fit in text height/width
  \sbox\pandoc@box{#1}%
  \Gscale@div\@tempa{\textheight}{\dimexpr\ht\pandoc@box+\dp\pandoc@box\relax}%
  \Gscale@div\@tempb{\linewidth}{\wd\pandoc@box}%
  \ifdim\@tempb\p@<\@tempa\p@\let\@tempa\@tempb\fi% select the smaller of both
  \ifdim\@tempa\p@<\p@\scalebox{\@tempa}{\usebox\pandoc@box}%
  \else\usebox{\pandoc@box}%
  \fi%
}
% Set default figure placement to htbp
\def\fps@figure{htbp}
\makeatother
% definitions for citeproc citations
\NewDocumentCommand\citeproctext{}{}
\NewDocumentCommand\citeproc{mm}{%
  \begingroup\def\citeproctext{#2}\cite{#1}\endgroup}
\makeatletter
 % allow citations to break across lines
 \let\@cite@ofmt\@firstofone
 % avoid brackets around text for \cite:
 \def\@biblabel#1{}
 \def\@cite#1#2{{#1\if@tempswa , #2\fi}}
\makeatother
\newlength{\cslhangindent}
\setlength{\cslhangindent}{1.5em}
\newlength{\csllabelwidth}
\setlength{\csllabelwidth}{3em}
\newenvironment{CSLReferences}[2] % #1 hanging-indent, #2 entry-spacing
 {\begin{list}{}{%
  \setlength{\itemindent}{0pt}
  \setlength{\leftmargin}{0pt}
  \setlength{\parsep}{0pt}
  % turn on hanging indent if param 1 is 1
  \ifodd #1
   \setlength{\leftmargin}{\cslhangindent}
   \setlength{\itemindent}{-1\cslhangindent}
  \fi
  % set entry spacing
  \setlength{\itemsep}{#2\baselineskip}}}
 {\end{list}}
\usepackage{calc}
\newcommand{\CSLBlock}[1]{\hfill\break\parbox[t]{\linewidth}{\strut\ignorespaces#1\strut}}
\newcommand{\CSLLeftMargin}[1]{\parbox[t]{\csllabelwidth}{\strut#1\strut}}
\newcommand{\CSLRightInline}[1]{\parbox[t]{\linewidth - \csllabelwidth}{\strut#1\strut}}
\newcommand{\CSLIndent}[1]{\hspace{\cslhangindent}#1}

\KOMAoption{captions}{tableheading}
\makeatletter
\@ifpackageloaded{bookmark}{}{\usepackage{bookmark}}
\makeatother
\makeatletter
\@ifpackageloaded{caption}{}{\usepackage{caption}}
\AtBeginDocument{%
\ifdefined\contentsname
  \renewcommand*\contentsname{Table of contents}
\else
  \newcommand\contentsname{Table of contents}
\fi
\ifdefined\listfigurename
  \renewcommand*\listfigurename{List of Figures}
\else
  \newcommand\listfigurename{List of Figures}
\fi
\ifdefined\listtablename
  \renewcommand*\listtablename{List of Tables}
\else
  \newcommand\listtablename{List of Tables}
\fi
\ifdefined\figurename
  \renewcommand*\figurename{Figure}
\else
  \newcommand\figurename{Figure}
\fi
\ifdefined\tablename
  \renewcommand*\tablename{Table}
\else
  \newcommand\tablename{Table}
\fi
}
\@ifpackageloaded{float}{}{\usepackage{float}}
\floatstyle{ruled}
\@ifundefined{c@chapter}{\newfloat{codelisting}{h}{lop}}{\newfloat{codelisting}{h}{lop}[chapter]}
\floatname{codelisting}{Listing}
\newcommand*\listoflistings{\listof{codelisting}{List of Listings}}
\makeatother
\makeatletter
\makeatother
\makeatletter
\@ifpackageloaded{caption}{}{\usepackage{caption}}
\@ifpackageloaded{subcaption}{}{\usepackage{subcaption}}
\makeatother

\usepackage{bookmark}

\IfFileExists{xurl.sty}{\usepackage{xurl}}{} % add URL line breaks if available
\urlstyle{same} % disable monospaced font for URLs
\hypersetup{
  pdftitle={Introduction to Network Analysis: An ADAPT Guide},
  pdfauthor={Tom R. Leppard},
  colorlinks=true,
  linkcolor={blue},
  filecolor={Maroon},
  citecolor={Blue},
  urlcolor={Blue},
  pdfcreator={LaTeX via pandoc}}


\title{Introduction to Network Analysis: An ADAPT Guide}
\author{Tom R. Leppard}
\date{2025-05-16}

\begin{document}
\maketitle

\renewcommand*\contentsname{Table of contents}
{
\hypersetup{linkcolor=}
\setcounter{tocdepth}{2}
\tableofcontents
}

\bookmarksetup{startatroot}

\chapter*{Preface}\label{preface}
\addcontentsline{toc}{chapter}{Preface}

\markboth{Preface}{Preface}

\bookmarksetup{startatroot}

\chapter{Introduction}\label{introduction}

This book is designed for students taking 'Introduction to Network
Analysis, a data science course on social network analysis taught by Tom
R. Leppard. Each chapter of the book with take you through the source
session-by-session.

The book is divided into three sections that mirror the units of the
course. First, it starts building your skills cleaning and transforming
network data. In this unit you will be learning about network data
structured, bringing network data into R and best practices for cleaning
network data. Second, it transitions into a unit on network
visualisation. This unit builds your skills in basic, intermediate, and
advanced network visualisation. The aim of this unit is to help you
create clean network visualisations that tell a clear story, and engage
your viewers. Finally, the book finishes with some modules on analysing
network data. Specifically, a discussion on the units of network
analysis, individuals (nodes in the networks), communities (clusters of
nodes in the networks), and the network itself. There is much to learn
beyond this book, however, by the end of this you will have learned the
fundamental principles and will build competence in network analysis.

\section{On ADAPT}\label{on-adapt}

This book is patterned after the ADAPT model championed by North
Carolina State University's Data Science and AI Academy. This
\href{https://datascienceacademy.ncsu.edu/courses/course-model/}{model}
stands for All Campus Data science and AI through Project based Teaching
and learning.

At its core, the model is designed to make computational skills
accessible to everyone, regardless of their discipline. In this spirit
of this model, you'll see that I use some fun data (e.g., Harry Potter,
UK Grime music, etc.) to teach network principles and R skills. I find
these appeal to a very broad student body. I also teach based on
concepts more than explaining code. Understanding the what and the why
behind social network analysis more than explaining every syntactical
function, I find, facilitates learning for many.

\subsection{ADAPT For Students}\label{adapt-for-students}

Students, all you need to know is that you will be working on a project
for the duration of the course. For more details see
\href{Final\%20Project\%20Instructions.qmd}{the Final Project
Instructions}.

Each unit and assignment is a milestone in your project development. For
this project, you may select any social network data that appeals to
you. I have collated quite a few from various sources available on a
\href{https://drive.google.com/drive/folders/18BRSRSLjBQcbYOiu59cWZ4fzmmK3ZoWJ}{Google
Drive} (you will need to sign in with your NCSU email to access it).
Alternatively, you may wish to explore other network data (from places
like the \href{https://networkrepository.com/index.php}{network
repository}, \href{https://www.kaggle.com/}{Kaggle} or others).

\subsection{ADAPT For Instructors}\label{adapt-for-instructors}

The teaching model is a living document that the Data Science and AI
team, headed by Rachel Levy, are constantly working on to mold. As
previously mentioned, the students are at the core of this model as is
the belief that data science acumen can be taught to people from every
disciplinary background. In fact, it behooves all to engage with data
and become data literate (understanding how to interpret data) in a
data-driven world!

I do not go into detail here but suffice it to say that the ADAPT model
comprises three parts:

\begin{enumerate}
\def\labelenumi{\arabic{enumi}.}
\item
  Project-based Learning
\item
  10 Common Learning Elements
\item
  Workforce Preparedness
\end{enumerate}

Throughout this book each of the units and modules are designed with
students in mind. The overarching theme of this book is to enact these
principles and enable other instructors wishing to utilise this content
to do the same.

\part{Unit 1: Cleaning and Transforming Network Data}

\chapter{Unit 1}\label{unit-1}

In this unit, we will cover how to transform and wrangle network data.
By the end of this unit you will:

\begin{enumerate}
\def\labelenumi{\arabic{enumi}.}
\item
  Understand how network data is structured
\item
  Know how to bring network data into RStudio
\item
  Know how to clean network data
\item
  Learn the difference between one and two mode networks
\end{enumerate}

\section{Project Milestones}\label{project-milestones}

\begin{longtable}[]{@{}
  >{\raggedright\arraybackslash}p{(\linewidth - 2\tabcolsep) * \real{0.5278}}
  >{\raggedright\arraybackslash}p{(\linewidth - 2\tabcolsep) * \real{0.4722}}@{}}
\toprule\noalign{}
\begin{minipage}[b]{\linewidth}\raggedright
Milestone (assignments linked)
\end{minipage} & \begin{minipage}[b]{\linewidth}\raggedright
Explanation
\end{minipage} \\
\midrule\noalign{}
\endhead
\bottomrule\noalign{}
\endlastfoot
\href{A1_My\%20Ego\%20Network.qmd}{My Ego Network} & Students will draw
their own ego network following an assignment in class. \\
\href{A2_Project\%20Prospectus.qmd}{Project Prospectus} & Students are
expected to discuss potential project ideas, possible sources of data,
and research questions they could explore. \\
\end{longtable}

Enjoy!

\chapter{Introducing Network Data
Structures}\label{introducing-network-data-structures}

One of the first steps to wrangling network data is to understand how it
is strcutured. This section will walk you through the basic ways that
network data is structured and demonstrate how to bring them into the R
envrionment. During class time we will work on converting these data
objects into network objects in R using the igraph package.

Keep in mind, you be using network data that is already stored into
these formats or may be formatting original data. We will first discuss
edgelists and then adjacency matrices.

Regardless of how your data is structred, the easiest way to store any
network data is in a .csv excel file. We can use the read.csv() function
in base R to read it into our environment.

\section{Edgelists}\label{edgelists}

Your data may be stored as an edgelist. An edgelist is what it says, a
list of edges or relationships that exist between the nodes in your
network. Since these are edges between nodes, the data are stored in a
dyadic format (pairs).

Split across two columns you have the names of everyone in the network
that share a connection. The basic format for any edgelist is to have a
`from' and a `to' column. The titles of the columns are arbitrary, but
are helpful for you as the researcher, especially if the connection is
directed. You may wish to call the columns `sender' and `receiver.'

This code chunk shows how to read in a .csv that is formatted as an
edgelist. Note, the header = TRUE option tells R that the first row are
headers (column names). Using the head() command, we see the first lines
of these network data.

This is a network of romantic affiliations based on students from the
Harry Potter saga. Note the column names reflect this.

PAUSE AND THINK: Is this a directed or an undirected network? What can
you see to indicate whether it is or is not?

\begin{Shaded}
\begin{Highlighting}[]
\NormalTok{my\_edge }\OtherTok{\textless{}{-}} \FunctionTok{read.csv}\NormalTok{(}\FunctionTok{file.choose}\NormalTok{(), }\AttributeTok{header =} \ConstantTok{TRUE}\NormalTok{)}

\FunctionTok{head}\NormalTok{(my\_edge)}
\end{Highlighting}
\end{Shaded}

\begin{verbatim}
           Crusher            Crush
1     Harry Potter    Ginny Weasley
2     Harry Potter        Cho Chang
3      Ron Weasley Hermione Granger
4 Hermione Granger      Ron Weasley
5      Ron Weasley   Lavender Brown
6    Ginny Weasley     Harry Potter
\end{verbatim}

\section{Adjacecny Matrices}\label{adjacecny-matrices}

Your data may be stored as an adjacency matrix. An adjacency matrix is a
datasheet that uses a numerical system (usually a binary system 0 and 1
for unweighted networks) to denote the ties that exist between cells in
the spreadsheet. 0 indicates no tie and 1 indicates a tie. In a weigted
network, the number may be higher than 1 (i.e.~to indicate the number of
interactions, the distance, or other weight).

The most important element of an adjacency matrix is that the first row
and the first column have the list of nodes. Each cell is an individual
node and this node is mirrored on the other side of the matrix. For
example, cell A2 is the same as B1. These two lines (the first row and
column) must have the same names in them in order for R to recognise it
as a network. In other words, an adjacecny matrix has all the possible
dyads (pairs) in the network with 1s and 0s to indicate whether they
share a tie. Note that A1 should always be left empty.

One final characteristic of an adjacency matrix is the line where the
same cell overlaps. This is called the diagonal. Cell A2 and B1 are the
same name, the coordinates whether those cells meet (B2) can indicate
whether that node is connected to itself. The same is true all the way
down the diagonal of the matrix. The researcher (YOU) must decide
whether self loops/ties make sense given the characteristics/parameters
of the network when you collect network data. For example, in a network
of sending text messages, it may not make sense.

This code chunk shows you how to bring in a .csv with network data
stored as an adjacency matrix. These data are affiliation data between
Harry Potter Characters based on their houses in Hogwarts. A 1
represents that they share a house-affiliation tie with the other
character (i.e.~they are in the same school house). Note, the row.names
= 1 option is used here to ensure R recognises row 1 as none names not
connections.

\begin{Shaded}
\begin{Highlighting}[]
\NormalTok{my\_adj  }\OtherTok{\textless{}{-}} \FunctionTok{read.csv}\NormalTok{(}\FunctionTok{file.choose}\NormalTok{(), }\AttributeTok{row.names=}\DecValTok{1}\NormalTok{) }

\FunctionTok{head}\NormalTok{(my\_adj)}
\end{Highlighting}
\end{Shaded}

\begin{verbatim}
                 Harry.Potter Draco.Malfoy Hermione.Granger Ron.Weasley
Harry.Potter                0            0                1           1
Draco.Malfoy                0            0                0           0
Hermione.Granger            1            0                0           1
Ron.Weasley                 1            0                1           0
Ginny.Weasley               1            0                1           1
Lily.Potter                 1            0                1           1
                 Ginny.Weasley Lily.Potter James.Potter Severus.Snape
Harry.Potter                 1           1            1             0
Draco.Malfoy                 0           0            0             1
Hermione.Granger             1           1            1             0
Ron.Weasley                  1           1            1             0
Ginny.Weasley                0           1            1             0
Lily.Potter                  1           0            1             0
                 Sirius.Black Lavender.Brown Nymphadora.Tonks Remus.Lupin
Harry.Potter                1              1                0           1
Draco.Malfoy                0              0                0           0
Hermione.Granger            1              1                0           1
Ron.Weasley                 1              1                0           1
Ginny.Weasley               1              1                0           1
Lily.Potter                 1              1                0           1
                 Cho.Chang Cedric.Diggory Teddy.Lupin James.Sirius.Potter
Harry.Potter             0              0           0                   1
Draco.Malfoy             0              0           0                   0
Hermione.Granger         0              0           0                   1
Ron.Weasley              0              0           0                   1
Ginny.Weasley            0              0           0                   1
Lily.Potter              0              0           0                   1
\end{verbatim}

\chapter{Cleaning Network Data - Trimming and
Adding}\label{cleaning-network-data---trimming-and-adding}

\begin{Shaded}
\begin{Highlighting}[]
\FunctionTok{library}\NormalTok{(igraph)}
\end{Highlighting}
\end{Shaded}

This script is intended to help you to clean up network data that you
have collected or got access to. One very common issue with cleaning
network data is knowing what to do with isolates. Isolates are those who
are a part of your network, but who have no connections to others in the
group. Isolates are stored in network data differently depending on how
your data are stored.

If your data are stored in an adjacency matrix, then isolates are those
with no 1s in the matrix. Ensuring that R recognises them as isolated is
very simple. Bring in the data, and then convert it into a matrix. Any
that are isolated will show as isolates.

\begin{Shaded}
\begin{Highlighting}[]
\NormalTok{hog\_crush\_matrix }\OtherTok{\textless{}{-}} \FunctionTok{read.csv}\NormalTok{(}\FunctionTok{file.choose}\NormalTok{(), }\AttributeTok{row.names =} \DecValTok{1}\NormalTok{, }\AttributeTok{header =} \ConstantTok{TRUE}\NormalTok{)}
\NormalTok{crush\_matrix }\OtherTok{\textless{}{-}} \FunctionTok{as.matrix}\NormalTok{(hog\_crush\_matrix)}
\NormalTok{hog\_crush\_net\_mat }\OtherTok{\textless{}{-}} \FunctionTok{graph\_from\_adjacency\_matrix}\NormalTok{(crush\_matrix, }\AttributeTok{mode =} \StringTok{"directed"}\NormalTok{, }\AttributeTok{diag =} \ConstantTok{FALSE}\NormalTok{)}
\FunctionTok{plot}\NormalTok{(hog\_crush\_net\_mat)}
\end{Highlighting}
\end{Shaded}

\pandocbounded{\includegraphics[keepaspectratio]{Cleaning-Network-Data---Trimming-and-Adding_files/figure-pdf/unnamed-chunk-2-1.pdf}}

However, things are not as straightforward when you are working with
edgelists.With this structure, you have only two columns, one for
senders and the other for receivers. If there is an individual in the
group who neither sends nor receives, what do you do with them? One way
of recording such isolates in an edgelist is list them as connected to
themselves (known as a self loop). Take a look at this edgelist and you
will see that these individuals are connected to themselves

\begin{Shaded}
\begin{Highlighting}[]
\NormalTok{hog\_crush\_correct }\OtherTok{\textless{}{-}} \FunctionTok{read.csv}\NormalTok{(}\FunctionTok{file.choose}\NormalTok{(), }\AttributeTok{header=}\ConstantTok{TRUE}\NormalTok{) }\CommentTok{\# select Hogwarts Crushes Edgelist\_CORRECT.csv}

\CommentTok{\#Take a look at the data}
\NormalTok{hog\_crush\_correct}
\end{Highlighting}
\end{Shaded}

\begin{verbatim}
            Crusher            Crush
1      Harry Potter    Ginny Weasley
2      Harry Potter        Cho Chang
3       Ron Weasley Hermione Granger
4  Hermione Granger      Ron Weasley
5       Ron Weasley   Lavender Brown
6     Ginny Weasley     Harry Potter
7       Lily Potter     James Potter
8      James Potter      Lily Potter
9     Severus Snape      Lily Potter
10 Nymphadora Tonks      Remus Lupin
11      Remus Lupin Nymphadora Tonks
12   Lavender Brown      Ron Weasley
13        Cho Chang   Cedric Diggory
14        Cho Chang     Harry Potter
15   Cedric Diggory        Cho Chang
16        McGonagal        McGonagal
17           Madeye           Madeye
18        Voldemort        Voldemort
19         Flitwick         Flitwick
\end{verbatim}

Now when you make this a graph object R does something different.

\begin{Shaded}
\begin{Highlighting}[]
\NormalTok{Crush\_correct\_net }\OtherTok{\textless{}{-}}  \FunctionTok{graph\_from\_data\_frame}\NormalTok{(hog\_crush\_correct, }\AttributeTok{directed =} \ConstantTok{TRUE}\NormalTok{)}
\FunctionTok{plot}\NormalTok{(Crush\_correct\_net)}
\end{Highlighting}
\end{Shaded}

\pandocbounded{\includegraphics[keepaspectratio]{Cleaning-Network-Data---Trimming-and-Adding_files/figure-pdf/unnamed-chunk-4-1.pdf}}

They have self looped edges!!! These do not look great. To Remove them,
you can use the delete\_edges() command and select the edges that are
looped by using the E() command coupled with the is.loop() option.This
is also something you will need to remember to do every time you bring
in an edgelist with isolates.

\begin{Shaded}
\begin{Highlighting}[]
\NormalTok{Crush\_correct\_net  }\OtherTok{\textless{}{-}} \FunctionTok{delete\_edges}\NormalTok{(Crush\_correct\_net , }\FunctionTok{E}\NormalTok{(Crush\_correct\_net )[}\FunctionTok{which\_loop}\NormalTok{(Crush\_correct\_net )])}
\FunctionTok{plot}\NormalTok{(Crush\_correct\_net)}
\end{Highlighting}
\end{Shaded}

\pandocbounded{\includegraphics[keepaspectratio]{Cleaning-Network-Data---Trimming-and-Adding_files/figure-pdf/unnamed-chunk-5-1.pdf}}

Another way to deal with isolates from an edgelist is to list noone in
the ``to'' column. In other words, you list the name of the person in
your network but leave the cell next to them blank. However, this
approach also has additional steps to take before it is clean and ready
to go.

\begin{Shaded}
\begin{Highlighting}[]
\NormalTok{hog\_crush\_wrong }\OtherTok{\textless{}{-}} \FunctionTok{read.csv}\NormalTok{(}\FunctionTok{file.choose}\NormalTok{(), }\AttributeTok{header=}\ConstantTok{TRUE}\NormalTok{) }\CommentTok{\# select Crushes Edgelist\_INCORRECT.csv}
\end{Highlighting}
\end{Shaded}

Take a look at the edgeist now it is in and you will see I added a few
more characters to this group: Madeye, Flitwick, McGonagal, and
Voldemort. They are all listed in the ``Crusher'' (from) column but have
no connection to anyone in the ``crush'' column. This makes sense, since
we know little about their romances from the Harry Potter Saga.

\begin{Shaded}
\begin{Highlighting}[]
\NormalTok{hog\_crush\_wrong}
\end{Highlighting}
\end{Shaded}

\begin{verbatim}
            Crusher            Crush
1      Harry Potter    Ginny Weasley
2      Harry Potter        Cho Chang
3       Ron Weasley Hermione Granger
4  Hermione Granger      Ron Weasley
5       Ron Weasley   Lavender Brown
6     Ginny Weasley     Harry Potter
7       Lily Potter     James Potter
8      James Potter      Lily Potter
9     Severus Snape      Lily Potter
10 Nymphadora Tonks      Remus Lupin
11      Remus Lupin Nymphadora Tonks
12   Lavender Brown      Ron Weasley
13        Cho Chang   Cedric Diggory
14        Cho Chang     Harry Potter
15   Cedric Diggory        Cho Chang
16        McGonagal                 
17           Madeye                 
18        Voldemort                 
19         Flitwick                 
\end{verbatim}

When we make this a graph object, R does something funky.

The new characters are all connected to a nameless node and it looks, on
visual inspection, that they all have a crush on the same person.

I have highlighted that node in the visualization below. The red node is
nameless because the edgelist has empty (nameless) cells.

\begin{Shaded}
\begin{Highlighting}[]
\NormalTok{crush\_wrong\_net }\OtherTok{\textless{}{-}} \FunctionTok{graph\_from\_data\_frame}\NormalTok{(hog\_crush\_wrong, }\AttributeTok{directed =} \ConstantTok{TRUE}\NormalTok{)}
\FunctionTok{plot}\NormalTok{(crush\_wrong\_net)}
\end{Highlighting}
\end{Shaded}

\pandocbounded{\includegraphics[keepaspectratio]{Cleaning-Network-Data---Trimming-and-Adding_files/figure-pdf/unnamed-chunk-8-1.pdf}}

\begin{Shaded}
\begin{Highlighting}[]
\FunctionTok{V}\NormalTok{(crush\_wrong\_net)}\SpecialCharTok{$}\NormalTok{wrong }\OtherTok{\textless{}{-}} \FunctionTok{ifelse}\NormalTok{(}\FunctionTok{V}\NormalTok{(crush\_wrong\_net)}\SpecialCharTok{$}\NormalTok{name }\SpecialCharTok{\%in\%} \FunctionTok{c}\NormalTok{(}\StringTok{""}\NormalTok{), }\StringTok{"red"}\NormalTok{, }\StringTok{"white"}\NormalTok{)}
\FunctionTok{plot}\NormalTok{(crush\_wrong\_net, }\AttributeTok{vertex.color =} \FunctionTok{V}\NormalTok{(crush\_wrong\_net)}\SpecialCharTok{$}\NormalTok{wrong)}
\end{Highlighting}
\end{Shaded}

\pandocbounded{\includegraphics[keepaspectratio]{Cleaning-Network-Data---Trimming-and-Adding_files/figure-pdf/unnamed-chunk-8-2.pdf}}

One way to deal with this is to delete the superfluous node. You do this
using the delete\_vertex() function. \#\#This fixes the issue once you
have the data in Rstudio, but the issue still exists in your dataset. If
you choose to structure your network data this way, you will have to
remember to remove this node every time. This may be harder to
do/realise when dealing with large dense networks.

\begin{Shaded}
\begin{Highlighting}[]
\NormalTok{crush\_wrong\_net }\OtherTok{\textless{}{-}} \FunctionTok{delete\_vertices}\NormalTok{(crush\_wrong\_net, }\StringTok{""}\NormalTok{)}
\FunctionTok{plot}\NormalTok{(crush\_wrong\_net)}
\end{Highlighting}
\end{Shaded}

\pandocbounded{\includegraphics[keepaspectratio]{Cleaning-Network-Data---Trimming-and-Adding_files/figure-pdf/unnamed-chunk-9-1.pdf}}

Other things to do to clean a network object once in Rstudio.

You may want to add or remove nodes and vertices (nodes) from your
network. Only do this if you have legitimate reason to.

Deleting Nodes. You might decide to remove one or more nodes from your
network. For example, in this hogwarts dataset, we may want to remove
those who are not students at Hogwarts (i.e.~remove teachers or adults).
To do this, you would use the delete\_vertices() option

Basic - You can delete them one-by-one.

\begin{Shaded}
\begin{Highlighting}[]
\NormalTok{hog\_crush\_students }\OtherTok{\textless{}{-}} \FunctionTok{delete\_vertices}\NormalTok{(Crush\_correct\_net, }\StringTok{"Voldemort"}\NormalTok{)}
\FunctionTok{plot}\NormalTok{(hog\_crush\_students)}
\end{Highlighting}
\end{Shaded}

\pandocbounded{\includegraphics[keepaspectratio]{Cleaning-Network-Data---Trimming-and-Adding_files/figure-pdf/unnamed-chunk-10-1.pdf}}

Pro tip - if you are deleting multiple, it is worth making a vector with
all the names of those you want to remove, then use the
delete\_vertices() command

\begin{Shaded}
\begin{Highlighting}[]
\NormalTok{hog\_adults }\OtherTok{\textless{}{-}} \FunctionTok{c}\NormalTok{(}\StringTok{"Severus Snape"}\NormalTok{, }\StringTok{"Lily Potter"}\NormalTok{, }\StringTok{"James Potter"}\NormalTok{, }\StringTok{"Nymphadora Tonks"}\NormalTok{, }\StringTok{"Remus Lupin"}\NormalTok{, }\StringTok{"Voldemort"}\NormalTok{, }\StringTok{"Flitwick"}\NormalTok{, }\StringTok{"McGonagal"}\NormalTok{, }\StringTok{"Madeye"}\NormalTok{)}
\NormalTok{hog\_crush\_students }\OtherTok{\textless{}{-}} \FunctionTok{delete\_vertices}\NormalTok{(Crush\_correct\_net, hog\_adults)}
\FunctionTok{plot}\NormalTok{(hog\_crush\_students)}
\end{Highlighting}
\end{Shaded}

\pandocbounded{\includegraphics[keepaspectratio]{Cleaning-Network-Data---Trimming-and-Adding_files/figure-pdf/unnamed-chunk-11-1.pdf}}

This new version removed all unwanted nodes at once.

\section{Deleting isolates.}\label{deleting-isolates.}

Sometimes, you want to remove the isolated nodes from your network
because you only care about those who have connections to others. To do
this, you identify those with no connections (degree = 0) and them
remove them from your network. I suggest making a new object with this
sub network.

\begin{Shaded}
\begin{Highlighting}[]
\NormalTok{hog\_crush\_isol }\OtherTok{\textless{}{-}} \FunctionTok{which}\NormalTok{(}\FunctionTok{degree}\NormalTok{(Crush\_correct\_net)}\SpecialCharTok{==}\DecValTok{0}\NormalTok{)}
\end{Highlighting}
\end{Shaded}

Now you use the delete\_vertices() command and remove those in the
vector you just created (those with degree = 0)

\begin{Shaded}
\begin{Highlighting}[]
\NormalTok{Crush\_no\_isol }\OtherTok{\textless{}{-}}\FunctionTok{delete\_vertices}\NormalTok{(Crush\_correct\_net, hog\_crush\_isol)}
\FunctionTok{plot}\NormalTok{(Crush\_no\_isol)}
\end{Highlighting}
\end{Shaded}

\pandocbounded{\includegraphics[keepaspectratio]{Cleaning-Network-Data---Trimming-and-Adding_files/figure-pdf/unnamed-chunk-13-1.pdf}}

Now this new object has only those nodes with ties to others in the
network.

\section{Adding Nodes}\label{adding-nodes}

Use add.vertices(graph name, number of additional vertices, attribute =
)

\begin{Shaded}
\begin{Highlighting}[]
\NormalTok{crush\_added }\OtherTok{\textless{}{-}} \FunctionTok{add.vertices}\NormalTok{(Crush\_correct\_net, }\DecValTok{1}\NormalTok{, }\AttributeTok{name =} \StringTok{"Michael Corner"}\NormalTok{)}
\FunctionTok{plot}\NormalTok{(crush\_added)}
\end{Highlighting}
\end{Shaded}

\pandocbounded{\includegraphics[keepaspectratio]{Cleaning-Network-Data---Trimming-and-Adding_files/figure-pdf/unnamed-chunk-14-1.pdf}}

\section{Deleting edges}\label{deleting-edges}

You may want to delete edges between two nodes.

\begin{Shaded}
\begin{Highlighting}[]
\NormalTok{edges\_to\_delete }\OtherTok{\textless{}{-}} \FunctionTok{E}\NormalTok{(Crush\_correct\_net)[(}\FunctionTok{.from}\NormalTok{(}\StringTok{"Remus Lupin"}\NormalTok{) }\SpecialCharTok{\&} \FunctionTok{.to}\NormalTok{(}\StringTok{"Nymphadora Tonks"}\NormalTok{))]}
\NormalTok{Crush\_edge\_delete }\OtherTok{\textless{}{-}} \FunctionTok{delete\_edges}\NormalTok{(Crush\_correct\_net, edges\_to\_delete)}
\FunctionTok{plot}\NormalTok{(Crush\_edge\_delete)}
\end{Highlighting}
\end{Shaded}

\pandocbounded{\includegraphics[keepaspectratio]{Cleaning-Network-Data---Trimming-and-Adding_files/figure-pdf/unnamed-chunk-15-1.pdf}}

To delete all edges between two nodes

\begin{Shaded}
\begin{Highlighting}[]
\NormalTok{edges\_to\_delete2 }\OtherTok{\textless{}{-}} \FunctionTok{E}\NormalTok{(Crush\_correct\_net)[(}\FunctionTok{.from}\NormalTok{(}\StringTok{"Remus Lupin"}\NormalTok{) }\SpecialCharTok{\&} \FunctionTok{.to}\NormalTok{(}\StringTok{"Nymphadora Tonks"}\NormalTok{)) }\SpecialCharTok{|} \FunctionTok{.from}\NormalTok{(}\StringTok{"Nymphadora Tonks"}\NormalTok{) }\SpecialCharTok{\&} \FunctionTok{.to}\NormalTok{(}\StringTok{"Remus Lupin"}\NormalTok{)]}
\NormalTok{Crush\_edge\_delete }\OtherTok{\textless{}{-}} \FunctionTok{delete\_edges}\NormalTok{(Crush\_correct\_net, edges\_to\_delete2)}
\FunctionTok{plot}\NormalTok{(Crush\_edge\_delete)}
\end{Highlighting}
\end{Shaded}

\pandocbounded{\includegraphics[keepaspectratio]{Cleaning-Network-Data---Trimming-and-Adding_files/figure-pdf/unnamed-chunk-16-1.pdf}}

\section{Add Edges}\label{add-edges}

Use add.edges().

\begin{Shaded}
\begin{Highlighting}[]
\NormalTok{crush\_added }\OtherTok{\textless{}{-}} \FunctionTok{add.edges}\NormalTok{(crush\_added, }\AttributeTok{edges =} \FunctionTok{c}\NormalTok{(}\StringTok{"Michael Corner"}\NormalTok{, }\StringTok{"Ginny Weasley"}\NormalTok{))}
\FunctionTok{plot}\NormalTok{(crush\_added)}
\end{Highlighting}
\end{Shaded}

\pandocbounded{\includegraphics[keepaspectratio]{Cleaning-Network-Data---Trimming-and-Adding_files/figure-pdf/unnamed-chunk-17-1.pdf}}

Now to add the reciprocated tie

\begin{Shaded}
\begin{Highlighting}[]
\NormalTok{crush\_added }\OtherTok{\textless{}{-}} \FunctionTok{add.edges}\NormalTok{(crush\_added, }\AttributeTok{edges =} \FunctionTok{c}\NormalTok{(}\StringTok{"Ginny Weasley"}\NormalTok{, }\StringTok{"Michael Corner"}\NormalTok{))}
\FunctionTok{plot}\NormalTok{(crush\_added)}
\end{Highlighting}
\end{Shaded}

\pandocbounded{\includegraphics[keepaspectratio]{Cleaning-Network-Data---Trimming-and-Adding_files/figure-pdf/unnamed-chunk-18-1.pdf}}

\chapter{Cleaning Network Data -
Subgraphs}\label{cleaning-network-data---subgraphs}

\begin{Shaded}
\begin{Highlighting}[]
\FunctionTok{library}\NormalTok{(igraph)}
\end{Highlighting}
\end{Shaded}

Bringing in the data and cleaning out the self loops.

\begin{Shaded}
\begin{Highlighting}[]
\NormalTok{grime\_edge\_list }\OtherTok{\textless{}{-}} \FunctionTok{read.csv}\NormalTok{(}\FunctionTok{file.choose}\NormalTok{(), }\AttributeTok{header =} \ConstantTok{TRUE}\NormalTok{)}

\NormalTok{grime\_08 }\OtherTok{\textless{}{-}} \FunctionTok{graph\_from\_data\_frame}\NormalTok{(}\AttributeTok{d=}\NormalTok{ grime\_edge\_list, }\AttributeTok{directed =} \ConstantTok{TRUE}\NormalTok{)}
\end{Highlighting}
\end{Shaded}

\begin{Shaded}
\begin{Highlighting}[]
\NormalTok{grime\_08\_clean }\OtherTok{\textless{}{-}} \FunctionTok{delete\_edges}\NormalTok{(grime\_08, }\FunctionTok{E}\NormalTok{(grime\_08)[}\FunctionTok{which\_loop}\NormalTok{(grime\_08)])}
\end{Highlighting}
\end{Shaded}

You may want to create subgraphs of the network that you have. There are
two basic ways that you can think about this. You may be interested in a
specific group of people and how they relate to each other, or you may
be interested in a specific person and find out who they are connected
to.

\section{Specific Subgraphs}\label{specific-subgraphs}

First, a subgraph to see a specific set of people and how/whether they
are connected

You may have a list of individual nodes that you are interested in and
you want to see how they related to each other. You can do this by
creating a vector with the names of thise nodes, then use the subgraph
function().

\begin{Shaded}
\begin{Highlighting}[]
\NormalTok{sub\_people }\OtherTok{\textless{}{-}} \FunctionTok{c}\NormalTok{(}\StringTok{\textquotesingle{}Wiley\textquotesingle{}}\NormalTok{, }\StringTok{\textquotesingle{}Jammer\textquotesingle{}}\NormalTok{, }\StringTok{\textquotesingle{}Flowdan\textquotesingle{}}\NormalTok{, }\StringTok{\textquotesingle{}Ice Kid\textquotesingle{}}\NormalTok{)}
\NormalTok{sub\_net }\OtherTok{\textless{}{-}} \FunctionTok{subgraph}\NormalTok{(grime\_08\_clean, sub\_people) }
\FunctionTok{par}\NormalTok{(}\AttributeTok{mar =} \FunctionTok{c}\NormalTok{(}\DecValTok{0}\NormalTok{,}\DecValTok{0}\NormalTok{,}\DecValTok{0}\NormalTok{,}\DecValTok{0}\NormalTok{))}
\FunctionTok{plot}\NormalTok{(sub\_net)}
\end{Highlighting}
\end{Shaded}

\pandocbounded{\includegraphics[keepaspectratio]{Cleaning-Network-Data---Subgraphs_files/figure-pdf/unnamed-chunk-4-1.pdf}}

\section{Ego Graphs}\label{ego-graphs}

Next, you may want to see ego networks from those in your network. In
other words, smaller networks showing only the connections of each
individual artist. To do this, you can use the make\_ego\_graph()
argument. This creates a list of ego graphs from your entire network.
Note, the order = 1 argument refers to the number of steps away from the
ego (focal node). Since mine is set to 1, this only caputres the ego's
immidiate neighbours (i.e.~only those directly connected to ego).

\begin{Shaded}
\begin{Highlighting}[]
\NormalTok{ego\_graphs }\OtherTok{\textless{}{-}} \FunctionTok{make\_ego\_graph}\NormalTok{(grime\_08\_clean, }\AttributeTok{order =} \DecValTok{1}\NormalTok{)}
\FunctionTok{head}\NormalTok{(ego\_graphs)}
\end{Highlighting}
\end{Shaded}

\begin{verbatim}
[[1]]
IGRAPH 7895cad DN-- 2 1 -- 
+ attr: name (v/c), collab_weight (e/n)
+ edge from 7895cad (vertex names):
[1] Asher D->Wiley

[[2]]
IGRAPH 7895cd1 DN-- 1 0 -- 
+ attr: name (v/c), collab_weight (e/n)
+ edges from 7895cd1 (vertex names):

[[3]]
IGRAPH 7895ce4 DN-- 1 0 -- 
+ attr: name (v/c), collab_weight (e/n)
+ edges from 7895ce4 (vertex names):

[[4]]
IGRAPH 7895cf5 DN-- 2 1 -- 
+ attr: name (v/c), collab_weight (e/n)
+ edge from 7895cf5 (vertex names):
[1] Scorcher->Wiley

[[5]]
IGRAPH 7895d06 DN-- 3 2 -- 
+ attr: name (v/c), collab_weight (e/n)
+ edges from 7895d06 (vertex names):
[1] Bless Beats->Wiley     Bless Beats->Roll Deep

[[6]]
IGRAPH 7895d16 DN-- 3 2 -- 
+ attr: name (v/c), collab_weight (e/n)
+ edges from 7895d16 (vertex names):
[1] Flowdan->Wiley  Flowdan->Jammer
\end{verbatim}

You can also specify exactly which node's network you want to see. Let's
say there was a person of interest in your network that you specifically
want to see. To do this, you can do the folliwing using the node's name
to single them out.

This chunk returns a lsit of edges connected to Wiley (the name of my
node of interest).

\begin{Shaded}
\begin{Highlighting}[]
\FunctionTok{E}\NormalTok{(grime\_08\_clean)[[}\FunctionTok{.inc}\NormalTok{(}\StringTok{\textquotesingle{}Wiley\textquotesingle{}}\NormalTok{)]]}
\end{Highlighting}
\end{Shaded}

\begin{verbatim}
+ 8/28 edges from 788c71c (vertex names):
             tail         head tid hid collab_weight
1         Asher D        Wiley   1  29             1
2        Scorcher        Wiley   4  29             4
3     Bless Beats        Wiley   5  29             1
4         Flowdan        Wiley   6  29             3
5  Tinchy Stryder        Wiley   7  29             2
6          Frisco        Wiley   8  29             1
7            Kano        Wiley   9  29             1
27          Wiley Lauren Mason  29  39             1
\end{verbatim}

I can also plot these. To do so, I make an object with the name `Wiley'
and then make an ego graph based on that name only. The {[}{[}1{]}{]}
simply tells R to get only the first one in the list that
make\_ego\_graph() creates. In this case, Wiley.

\begin{Shaded}
\begin{Highlighting}[]
\NormalTok{Wiley }\OtherTok{\textless{}{-}} \StringTok{"Wiley"}
\NormalTok{ego\_wiley }\OtherTok{\textless{}{-}} \FunctionTok{make\_ego\_graph}\NormalTok{(grime\_08\_clean, }\AttributeTok{order =} \DecValTok{1}\NormalTok{, }\AttributeTok{nodes =}\NormalTok{ Wiley)[[}\DecValTok{1}\NormalTok{]]}

\FunctionTok{par}\NormalTok{(}\AttributeTok{mar =} \FunctionTok{c}\NormalTok{(}\DecValTok{0}\NormalTok{,}\DecValTok{0}\NormalTok{,}\DecValTok{0}\NormalTok{,}\DecValTok{0}\NormalTok{))}
\FunctionTok{plot}\NormalTok{(ego\_wiley)}
\end{Highlighting}
\end{Shaded}

\pandocbounded{\includegraphics[keepaspectratio]{Cleaning-Network-Data---Subgraphs_files/figure-pdf/unnamed-chunk-7-1.pdf}}

You can also select an ego network with more information. The second
order ego network includes the connections of Wiley's neighbours.

\begin{Shaded}
\begin{Highlighting}[]
\NormalTok{second\_order\_wiley }\OtherTok{\textless{}{-}} \FunctionTok{make\_ego\_graph}\NormalTok{(grime\_08\_clean, }\AttributeTok{order =} \DecValTok{2}\NormalTok{, }\AttributeTok{nodes =}\NormalTok{ Wiley)[[}\DecValTok{1}\NormalTok{]]}

\FunctionTok{par}\NormalTok{(}\AttributeTok{mar =} \FunctionTok{c}\NormalTok{(}\DecValTok{0}\NormalTok{,}\DecValTok{0}\NormalTok{,}\DecValTok{0}\NormalTok{,}\DecValTok{0}\NormalTok{))}
\FunctionTok{plot}\NormalTok{(second\_order\_wiley)}
\end{Highlighting}
\end{Shaded}

\pandocbounded{\includegraphics[keepaspectratio]{Cleaning-Network-Data---Subgraphs_files/figure-pdf/unnamed-chunk-8-1.pdf}}

Pro tip: If you are working with ego networks like this, especially when
you get passed the first order network (including friends of friends) it
is good practice to do something to differentiate the ego from their
neighbours. Once simple way it to change their colour.

\begin{Shaded}
\begin{Highlighting}[]
\FunctionTok{V}\NormalTok{(second\_order\_wiley)}\SpecialCharTok{$}\NormalTok{ego }\OtherTok{\textless{}{-}} \FunctionTok{ifelse}\NormalTok{(}\FunctionTok{V}\NormalTok{(second\_order\_wiley)}\SpecialCharTok{$}\NormalTok{name }\SpecialCharTok{\%in\%} \FunctionTok{c}\NormalTok{(}\StringTok{"Wiley"}\NormalTok{), }\StringTok{"red"}\NormalTok{, }\StringTok{"white"}\NormalTok{)}

\FunctionTok{par}\NormalTok{(}\AttributeTok{mar =} \FunctionTok{c}\NormalTok{(}\DecValTok{0}\NormalTok{,}\DecValTok{0}\NormalTok{,}\DecValTok{3}\NormalTok{,}\DecValTok{0}\NormalTok{))}
\FunctionTok{plot}\NormalTok{(second\_order\_wiley, }\AttributeTok{vertex.color =} \FunctionTok{V}\NormalTok{(second\_order\_wiley)}\SpecialCharTok{$}\NormalTok{ego, }\AttributeTok{main =} \StringTok{"Wiley\textquotesingle{}s Second Order Ego Network"}\NormalTok{)}
\end{Highlighting}
\end{Shaded}

\pandocbounded{\includegraphics[keepaspectratio]{Cleaning-Network-Data---Subgraphs_files/figure-pdf/unnamed-chunk-9-1.pdf}}

Finally, one other way to can subset a network is by a set parameter you
may have. For example, you may want to see a network of frequent
collaborators (more than 1 collab).

The following returns a vector with collaborators who work together more
than once.

\begin{Shaded}
\begin{Highlighting}[]
\NormalTok{frequent\_collabors }\OtherTok{\textless{}{-}} \FunctionTok{E}\NormalTok{(grime\_08\_clean)[[collab\_weight }\SpecialCharTok{\textgreater{}} \DecValTok{1}\NormalTok{]]}
\NormalTok{frequent\_collabors}
\end{Highlighting}
\end{Shaded}

\begin{verbatim}
+ 8/28 edges from 788c71c (vertex names):
             tail   head tid hid collab_weight
2        Scorcher  Wiley   4  29             4
4         Flowdan  Wiley   6  29             3
5  Tinchy Stryder  Wiley   7  29             2
8          Blacks Jammer  12  35             4
9         Badness Jammer  13  35             5
11        Tempa T Jammer  15  35             2
14         Skepta Jammer  17  35             5
16         Frisco Jammer   8  35             3
\end{verbatim}

You can then turn this vector of edges into a igraph object to plot

\begin{Shaded}
\begin{Highlighting}[]
\NormalTok{frequent\_collabors\_graph }\OtherTok{\textless{}{-}} \FunctionTok{induced\_subgraph}\NormalTok{(grime\_08\_clean, }\AttributeTok{vids =} \FunctionTok{unique}\NormalTok{(}\FunctionTok{c}\NormalTok{(}\FunctionTok{ends}\NormalTok{(grime\_08\_clean, frequent\_collabors)[, }\DecValTok{1}\NormalTok{], }\FunctionTok{ends}\NormalTok{(grime\_08\_clean, frequent\_collabors)[, }\DecValTok{2}\NormalTok{])))}
\FunctionTok{plot}\NormalTok{(frequent\_collabors\_graph)}
\end{Highlighting}
\end{Shaded}

\pandocbounded{\includegraphics[keepaspectratio]{Cleaning-Network-Data---Subgraphs_files/figure-pdf/unnamed-chunk-11-1.pdf}}

\chapter{Node and Edge Attributes}\label{node-and-edge-attributes}

\begin{Shaded}
\begin{Highlighting}[]
\FunctionTok{library}\NormalTok{(igraph)}
\FunctionTok{library}\NormalTok{(readxl)}
\end{Highlighting}
\end{Shaded}

Your network may have some edge characteristics. What this means is that
the network has qualitative or quantitative information regarding the
connections between nodes. These could be things that denote a certain
type of connection between individuals in the network (romantic
vs.~friend, positive vs.~negative, kinship vs.~friend). These
qualitatively different categories tell us more about the types of
relationships that there are between the nodes in our network.
Meanwhile, quantitative information can also tells us more about the
information. Quantitative information could include things like
frequency of communication. Such information is termed the edges
``weight'' indicating that there are substantively meaningful
differences between the levels of connection (for example interacting
only once compared to 10 times).

Additionally, your network may have some information about the nodes.
Such information could include categorical information (e.g demographics
or other categories) or numeric information (e.g.~age). This means we
can inform our visualisation to portray the information about the nodes.

All of this information can be attached to our edgelist. You can also do
this on an adjacency matrix but it is not as straightforward. To learn
the process, let's stick with edgelists.

\section{Getting to know these data}\label{getting-to-know-these-data}

This dataset has a lot more information about these individuals and
their relationships than others we have used this far. We have one excel
spreadsheet (not .csv this time) with two separate sheets, one called
vertices and the other called edges. The vertices sheet has information
about each individual node in the network while the edges contains both
the edges that exist as well as information about them. Let's read them
in and take a look at them

We are using a little more specific code on this one using the readxl
package to read in a specific sheet from our spreadsheet. Then we can
read in each sheet using the followin method.

\begin{Shaded}
\begin{Highlighting}[]
\NormalTok{vertices.df }\OtherTok{\textless{}{-}} \FunctionTok{read\_excel}\NormalTok{(}\StringTok{"C:/Users/trleppar/Downloads/node and vertex characteristics.xlsx"}\NormalTok{, }\AttributeTok{sheet =} \StringTok{"vertices"}\NormalTok{)}
\NormalTok{edges.df }\OtherTok{\textless{}{-}} \FunctionTok{read\_excel}\NormalTok{(}\StringTok{"C:/Users/trleppar/Downloads/node and vertex characteristics.xlsx"}\NormalTok{, }\AttributeTok{sheet =} \StringTok{"edges"}\NormalTok{)}
\end{Highlighting}
\end{Shaded}

Let's take a look at these one at a tine to get an idea of what
information we have.

\begin{Shaded}
\begin{Highlighting}[]
\FunctionTok{head}\NormalTok{(edges.df)}
\end{Highlighting}
\end{Shaded}

\begin{verbatim}
# A tibble: 6 x 4
  from  to     freq affinity
  <chr> <chr> <dbl> <chr>   
1 A     B         2 pos     
2 A     C         1 neg     
3 A     D         1 pos     
4 A     E         1 neg     
5 A     F         3 neg     
6 E     F         2 pos     
\end{verbatim}

We have an edgelist between A-F people. In this edgelist we have the
frequency of interaction and the affinity (i.e.~if they are positive or
negative interactions).

\begin{Shaded}
\begin{Highlighting}[]
\FunctionTok{head}\NormalTok{(vertices.df)}
\end{Highlighting}
\end{Shaded}

\begin{verbatim}
# A tibble: 6 x 4
  name    age role  gender
  <chr> <dbl> <chr> <chr> 
1 A        20 DJ    F     
2 B        25 MC    M     
3 C        21 DJ    F     
4 D        23 crew  M     
5 E        24 MC    M     
6 F        23 MC    F     
\end{verbatim}

We have their name, age and two categorical variables about them, their
role (if they are a DJ or something else) and their sex (Male/Female).

Now I can create a network object in igraph using the familiar method
you are used to - graph\_from\_data\_frame(). However, We want this
network to have all the information possible. For this, we don't just
want the edge information, but also the node level information. TO do
this, we tell R that the data = the edgelist.df (a familiar step to you
pros now!), and the vertex characteristics are stored in the object we
created earlier, vertices.df

\begin{Shaded}
\begin{Highlighting}[]
\NormalTok{graph }\OtherTok{\textless{}{-}} \FunctionTok{graph\_from\_data\_frame}\NormalTok{(}\AttributeTok{d =}\NormalTok{ edges.df, }\AttributeTok{vertices =}\NormalTok{ vertices.df , }\AttributeTok{directed =} \ConstantTok{FALSE}\NormalTok{)}

\NormalTok{graph}
\end{Highlighting}
\end{Shaded}

\begin{verbatim}
IGRAPH 7bbf6ab UN-- 7 7 -- 
+ attr: name (v/c), age (v/n), role (v/c), gender (v/c), freq (e/n),
| affinity (e/c)
+ edges from 7bbf6ab (vertex names):
[1] A--B A--C A--D A--E A--F E--F F--G
\end{verbatim}

Nice work! You can see the vertex information is stores as v
characteristics (name, age, role and gender. The edge characteristics
are stored as e characteristics - freq, affinity. Now lets visualise the
network and see what we have.

\begin{Shaded}
\begin{Highlighting}[]
\FunctionTok{plot}\NormalTok{(graph)}
\end{Highlighting}
\end{Shaded}

\pandocbounded{\includegraphics[keepaspectratio]{Node-and-Edge-Attributes_files/figure-pdf/unnamed-chunk-6-1.pdf}}

Here comes the fun part. First, let's start with the edge
characteristics. Rapid fire, we can visualise these in may different
ways.

\section{Exploring Edge
Characteristics}\label{exploring-edge-characteristics}

We will create a few visuals to demonstrate the information about these
edges.

Let's start with the numeric information we have about the edges. First,
we will change the width of the lines between nodes to reflect the
frequency of interactions using the edge.width argument and the freq
edge characteristic.

\begin{Shaded}
\begin{Highlighting}[]
\FunctionTok{plot}\NormalTok{(graph, }\AttributeTok{edge.width =} \FunctionTok{E}\NormalTok{(graph)}\SpecialCharTok{$}\NormalTok{freq)}
\end{Highlighting}
\end{Shaded}

\pandocbounded{\includegraphics[keepaspectratio]{Node-and-Edge-Attributes_files/figure-pdf/unnamed-chunk-7-1.pdf}}

Or, we can label the nodes with the frequency to tell a similar story.
We do this using the edge.labelargument.

\begin{Shaded}
\begin{Highlighting}[]
\FunctionTok{plot}\NormalTok{(graph, }\AttributeTok{edge.label =} \FunctionTok{E}\NormalTok{(graph)}\SpecialCharTok{$}\NormalTok{freq)}
\end{Highlighting}
\end{Shaded}

\pandocbounded{\includegraphics[keepaspectratio]{Node-and-Edge-Attributes_files/figure-pdf/unnamed-chunk-8-1.pdf}}

What do these visuals tell you about their relationships compared to the
first one?

Now let's use the categorical information to tell a slightly different
story. Let's see what we can do to demonstrate the levels of affinity
between these individuals. First, we will change the line type to
reflect the different levels. To do this, we first create a logical
comparison using an ifelse statement. This checks if the affinity
attribute of each edge is equal to ``pos''. This will return a logical
vector (TRUE or FALSE for each edge). If the edge is ``pos'' then it
will return an item of the vector ``solid'' if it is false
(i.e.~``neg''), then it will return ``dotdash''. We can then visualise
this in the network using the edge.lty argument.

\begin{Shaded}
\begin{Highlighting}[]
\CommentTok{\#change the line type using edge.lty to match the affinity}
\NormalTok{type\_affinity }\OtherTok{\textless{}{-}} \FunctionTok{ifelse}\NormalTok{(}\FunctionTok{E}\NormalTok{(graph)}\SpecialCharTok{$}\NormalTok{affinity }\SpecialCharTok{==} \StringTok{"pos"}\NormalTok{, }\StringTok{"solid"}\NormalTok{, }\StringTok{"dotdash"}\NormalTok{)}
\CommentTok{\# Plot plus colour}
\FunctionTok{plot}\NormalTok{(graph, }\AttributeTok{edge.lty =}\NormalTok{ type\_affinity)}
\end{Highlighting}
\end{Shaded}

\pandocbounded{\includegraphics[keepaspectratio]{Node-and-Edge-Attributes_files/figure-pdf/unnamed-chunk-9-1.pdf}}

Now, let's combine a few approaches. We will use the same
ifelsestatement but will apply it to the colours of the edges. We will
also change the edge labels to reflect the affinity label alongside the
line type.

\begin{Shaded}
\begin{Highlighting}[]
\NormalTok{affinity }\OtherTok{\textless{}{-}} \FunctionTok{ifelse}\NormalTok{(}\FunctionTok{E}\NormalTok{(graph)}\SpecialCharTok{$}\NormalTok{affinity }\SpecialCharTok{==} \StringTok{"pos"}\NormalTok{, }\StringTok{"green"}\NormalTok{, }\StringTok{"red"}\NormalTok{)}
\FunctionTok{plot}\NormalTok{(graph, }\AttributeTok{edge.color =}\NormalTok{ affinity, }\AttributeTok{edge.label =} \FunctionTok{E}\NormalTok{(graph)}\SpecialCharTok{$}\NormalTok{affinity, }\AttributeTok{edge.lty =}\NormalTok{ type\_affinity)}
\end{Highlighting}
\end{Shaded}

\pandocbounded{\includegraphics[keepaspectratio]{Node-and-Edge-Attributes_files/figure-pdf/unnamed-chunk-10-1.pdf}}

\section{Exploring Vertex Attributes}\label{exploring-vertex-attributes}

Now let's turn to the rest of our data and explore the network's vertex
attributes.

We will start with the numerical characteristics of the attributes -
their age. First, let's change the labels to show their age.

\begin{Shaded}
\begin{Highlighting}[]
\FunctionTok{plot}\NormalTok{(graph, }\AttributeTok{vertex.label =} \FunctionTok{V}\NormalTok{(graph)}\SpecialCharTok{$}\NormalTok{age)}
\end{Highlighting}
\end{Shaded}

\pandocbounded{\includegraphics[keepaspectratio]{Node-and-Edge-Attributes_files/figure-pdf/unnamed-chunk-11-1.pdf}}

Now, let's change the colours based on certain parameters that we set
using an ifelse() statement.

\begin{Shaded}
\begin{Highlighting}[]
\NormalTok{over\_22 }\OtherTok{\textless{}{-}} \FunctionTok{ifelse}\NormalTok{(}\FunctionTok{V}\NormalTok{(graph)}\SpecialCharTok{$}\NormalTok{age }\SpecialCharTok{\textgreater{}} \DecValTok{22}\NormalTok{, }\StringTok{"red"}\NormalTok{, }\StringTok{"white"}\NormalTok{) }
\FunctionTok{plot}\NormalTok{(graph, }\AttributeTok{vertex.color =}\NormalTok{ over\_22, }\AttributeTok{veterx.label.color =} \StringTok{"Black"}\NormalTok{)}
\end{Highlighting}
\end{Shaded}

\pandocbounded{\includegraphics[keepaspectratio]{Node-and-Edge-Attributes_files/figure-pdf/unnamed-chunk-12-1.pdf}}

Next, let's work with the categorical variables. First we can change the
labels to show these, and then change the colours. See if you can follow
the following code chunks and think about what these new networks tell
us.

\begin{Shaded}
\begin{Highlighting}[]
\FunctionTok{plot}\NormalTok{(graph, }\AttributeTok{vertex.label =}\FunctionTok{V}\NormalTok{(graph)}\SpecialCharTok{$}\NormalTok{gender)}
\end{Highlighting}
\end{Shaded}

\pandocbounded{\includegraphics[keepaspectratio]{Node-and-Edge-Attributes_files/figure-pdf/unnamed-chunk-13-1.pdf}}

\begin{Shaded}
\begin{Highlighting}[]
\FunctionTok{plot}\NormalTok{(graph, }\AttributeTok{vertex.label =} \FunctionTok{V}\NormalTok{(graph)}\SpecialCharTok{$}\NormalTok{role)}
\end{Highlighting}
\end{Shaded}

\pandocbounded{\includegraphics[keepaspectratio]{Node-and-Edge-Attributes_files/figure-pdf/unnamed-chunk-14-1.pdf}}

\begin{Shaded}
\begin{Highlighting}[]
\NormalTok{gender }\OtherTok{\textless{}{-}} \FunctionTok{ifelse}\NormalTok{(}\FunctionTok{V}\NormalTok{(graph)}\SpecialCharTok{$}\NormalTok{gender }\SpecialCharTok{==} \StringTok{"M"}\NormalTok{, }\StringTok{"orange"}\NormalTok{, }\StringTok{"blue"}\NormalTok{)}
\FunctionTok{plot}\NormalTok{(graph, }\AttributeTok{vertex.color =}\NormalTok{ gender, }\AttributeTok{vertex.label.color =} \StringTok{"white"}\NormalTok{)}
\end{Highlighting}
\end{Shaded}

\pandocbounded{\includegraphics[keepaspectratio]{Node-and-Edge-Attributes_files/figure-pdf/unnamed-chunk-15-1.pdf}}

We have done a lot with ifelse statements here. This are great for
setting direct parameters or for working with dichotomous categories
(i.e.~the male/female one we have). However, we may want to create
colours for categories that have more than one and then visualise it. We
are going to use a differet package, called dplyr to manipulate what we
have to create a vertex attribute that reflect colours based on a
categorical variable (their role).

\begin{Shaded}
\begin{Highlighting}[]
\FunctionTok{library}\NormalTok{(dplyr)}
\end{Highlighting}
\end{Shaded}

To do this, we will return to the original dataframe storing information
about the vertex characteristics. Then, we will use the mutate()
function to create a new variable that reflect a colour for each role.
See if you can follow the logic and look at what we end up with.

\begin{Shaded}
\begin{Highlighting}[]
\NormalTok{vertices.df }\OtherTok{\textless{}{-}}\NormalTok{ vertices.df }\SpecialCharTok{\%\textgreater{}\%}
  \FunctionTok{mutate}\NormalTok{(}\AttributeTok{role\_colour =} \FunctionTok{ifelse}\NormalTok{(role }\SpecialCharTok{==} \StringTok{"DJ"}\NormalTok{, }\StringTok{"blue"}\NormalTok{, role)) }
\NormalTok{vertices.df }\OtherTok{\textless{}{-}}\NormalTok{ vertices.df }\SpecialCharTok{\%\textgreater{}\%}
  \FunctionTok{mutate}\NormalTok{(}\AttributeTok{role\_colour =} \FunctionTok{ifelse}\NormalTok{(role }\SpecialCharTok{==} \StringTok{"MC"}\NormalTok{, }\StringTok{"red"}\NormalTok{, role\_colour))}
\NormalTok{vertices.df }\OtherTok{\textless{}{-}}\NormalTok{ vertices.df }\SpecialCharTok{\%\textgreater{}\%}
  \FunctionTok{mutate}\NormalTok{(}\AttributeTok{role\_colour =} \FunctionTok{ifelse}\NormalTok{(role }\SpecialCharTok{==} \StringTok{"crew"}\NormalTok{, }\StringTok{"green"}\NormalTok{, role\_colour))}

\FunctionTok{head}\NormalTok{(vertices.df)}
\end{Highlighting}
\end{Shaded}

\begin{verbatim}
# A tibble: 6 x 5
  name    age role  gender role_colour
  <chr> <dbl> <chr> <chr>  <chr>      
1 A        20 DJ    F      blue       
2 B        25 MC    M      red        
3 C        21 DJ    F      blue       
4 D        23 crew  M      green      
5 E        24 MC    M      red        
6 F        23 MC    F      red        
\end{verbatim}

Now, let's recreate our network object following the above method.

\begin{Shaded}
\begin{Highlighting}[]
\NormalTok{graph }\OtherTok{\textless{}{-}} \FunctionTok{graph\_from\_data\_frame}\NormalTok{(}\AttributeTok{d =}\NormalTok{ edges.df, }\AttributeTok{vertices =}\NormalTok{ vertices.df , }\AttributeTok{directed =} \ConstantTok{FALSE}\NormalTok{)}

\NormalTok{graph}
\end{Highlighting}
\end{Shaded}

\begin{verbatim}
IGRAPH 7bf708c UN-- 7 7 -- 
+ attr: name (v/c), age (v/n), role (v/c), gender (v/c), role_colour
| (v/c), freq (e/n), affinity (e/c)
+ edges from 7bf708c (vertex names):
[1] A--B A--C A--D A--E A--F E--F F--G
\end{verbatim}

The network has the new v characteristic that we created - role\_colour

Now we can visualise this network with the different colours for the
roles all represented on the visual.

\begin{Shaded}
\begin{Highlighting}[]
\FunctionTok{plot}\NormalTok{(graph, }\AttributeTok{vertex.color =} \FunctionTok{V}\NormalTok{(graph)}\SpecialCharTok{$}\NormalTok{role\_colour, }\AttributeTok{vertex.label.color =} \StringTok{"white"}\NormalTok{)}
\end{Highlighting}
\end{Shaded}

\pandocbounded{\includegraphics[keepaspectratio]{Node-and-Edge-Attributes_files/figure-pdf/unnamed-chunk-19-1.pdf}}

GREAT WORK!

\chapter{Two Mode Networks - Adjacency
Matrix}\label{two-mode-networks---adjacency-matrix}

\begin{Shaded}
\begin{Highlighting}[]
\FunctionTok{library}\NormalTok{(igraph)}
\end{Highlighting}
\end{Shaded}

This script is intended for learning how to work with two-mode network
data. It uses some data I put together about Characters from the Harry
Potter movies.

\section{Getting to Know the Data}\label{getting-to-know-the-data}

Most of the info came from this site -
https://harrypotter.fandom.com/wiki/Dumbledore\%27s\_Army I.e. I googled
``list of prefects at hogwarts'' then deferred to that above wiki. Same
for phoenix, death eaters etc.

I then checked each person on google to see what house they were in -
some are missing and NA because they are fake characters from the movies
or the wiki page.

\begin{Shaded}
\begin{Highlighting}[]
\NormalTok{hp }\OtherTok{\textless{}{-}} \FunctionTok{read.csv}\NormalTok{(}\FunctionTok{file.choose}\NormalTok{(),}\AttributeTok{header=}\ConstantTok{TRUE}\NormalTok{,}\AttributeTok{row.names=}\DecValTok{1}\NormalTok{) }\DocumentationTok{\#\# select harry\_potter\_two\_mode.csv}

\NormalTok{hp\_mat }\OtherTok{\textless{}{-}} \FunctionTok{as.matrix}\NormalTok{(hp)}

\FunctionTok{head}\NormalTok{(hp\_mat)}
\end{Highlighting}
\end{Shaded}

\begin{verbatim}
                 Phoenix Dumbeldore.s.Army Death.Eaters Inquisitorial.Squad
Albus Dumbledore       1                 0            0                   0
Remus Lupin            1                 0            0                   0
Molly Weasley          1                 0            0                   0
Siruis Black           1                 0            0                   0
Severus Snape          1                 0            1                   0
Alastor Moody          1                 0            0                   0
                 Prefect Gryffindor Ravenclaw Hufflepuff Slytherin
Albus Dumbledore       1          1         0          0         0
Remus Lupin            1          1         0          0         0
Molly Weasley          0          1         0          0         0
Siruis Black           0          1         0          0         0
Severus Snape          0          0         0          0         1
Alastor Moody          0          0         0          1         0
\end{verbatim}

\subsection{Two Mode Adjacency
Matrices}\label{two-mode-adjacency-matrices}

So, here we have a two mode adjacency matrix. You will notice some
things that are similar to you, perhaps the 1/0 nature of a matrix. But
this is slightly different. The columns no longer reflect the same names
as the rows. Now, instead of an i,j matrix we have an i,g (group)
matrix. This means that there is no diagonal. Why? Well, because the
names at the top of the matrix (columns) are different from the side
(rows). Here then, I is sending to the group. Rather, we talk about
this, usually, in terms of affiliation. So, i is affiliated with the
group (or not).

For R to understand this is a two mode network matrix, we use a slightly
different command than a regular matrix.
graph\_from\_biadjacency\_matrix() is the current function where R
recognises the separate column names as one type of node and the row
names as another. For this to truly be a two mode network, they have to
be distinct.

\begin{Shaded}
\begin{Highlighting}[]
\NormalTok{hp\_aff }\OtherTok{\textless{}{-}} \FunctionTok{graph\_from\_biadjacency\_matrix}\NormalTok{(hp\_mat)}


\FunctionTok{plot}\NormalTok{(hp\_aff)}
\end{Highlighting}
\end{Shaded}

\pandocbounded{\includegraphics[keepaspectratio]{Two-Mode-Networks---AM_files/figure-pdf/unnamed-chunk-3-1.pdf}}

\section{Visualising Two Mode
Networks}\label{visualising-two-mode-networks}

Let's make the visualisation much clearer between the two types of
nodes.

I do this by changing the shape and the colour of each type of node. I
set a vector with the colours and shapes I want

\begin{Shaded}
\begin{Highlighting}[]
\NormalTok{shapes }\OtherTok{\textless{}{-}} \FunctionTok{c}\NormalTok{(}\StringTok{"circle"}\NormalTok{, }\StringTok{"square"}\NormalTok{)}
\NormalTok{colors }\OtherTok{\textless{}{-}}\FunctionTok{c}\NormalTok{(}\StringTok{"green"}\NormalTok{, }\StringTok{"orange"}\NormalTok{)}
\end{Highlighting}
\end{Shaded}

Then we can plot them based on these design parameters

\begin{Shaded}
\begin{Highlighting}[]
\FunctionTok{par}\NormalTok{(}\AttributeTok{mar =}\FunctionTok{c}\NormalTok{(}\DecValTok{5}\NormalTok{,}\DecValTok{0}\NormalTok{,}\DecValTok{2}\NormalTok{,}\DecValTok{0}\NormalTok{))}
\FunctionTok{plot}\NormalTok{(hp\_aff, }\AttributeTok{vertex.color=}\NormalTok{colors[}\FunctionTok{V}\NormalTok{(hp\_aff)}\SpecialCharTok{$}\NormalTok{type}\SpecialCharTok{+}\DecValTok{1}\NormalTok{],}
     \AttributeTok{vertex.shape=}\NormalTok{shapes[}\FunctionTok{V}\NormalTok{(hp\_aff)}\SpecialCharTok{$}\NormalTok{type}\SpecialCharTok{+}\DecValTok{1}\NormalTok{], }\AttributeTok{vertex.label.cex =} \FloatTok{0.5}\NormalTok{, }\AttributeTok{vertex.size =} \DecValTok{7}\NormalTok{, }\AttributeTok{main =} \StringTok{"Harry Potter"}\NormalTok{, }\AttributeTok{sub =} \StringTok{"Characters Connected to Groups"}\NormalTok{)}
\end{Highlighting}
\end{Shaded}

\pandocbounded{\includegraphics[keepaspectratio]{Two-Mode-Networks---AM_files/figure-pdf/unnamed-chunk-5-1.pdf}}

Here, I tell R to use the indices I've defined in the shapes and colors
vectors and apply those to the network using the vertex.shape and
vertex.color arguments. Notice that I need to state type + 1 in both of
these arguments. It might look a bit unusual at first, but it makes
sense once we take a closer look at what R does behind the scenes.

The ``type'' vertex characteristic is stored as TRUE or FALSE. You can
verify this by running the code V(hp\_aff)\$type. This will display a
long list of TRUE and FALSE values, which are stored in R as logical
values: TRUE is equivalent to 1, and FALSE is equivalent to 0.
Meanwhile, the index values for our shapes and colors vectors are stored
differently. R always starts indexing at 1. In our colors vector,
``green'' is stored at index 1 and ``orange'' at index 2. For the shapes
vector, ``circle'' is stored at index 1 and ``square'' at index 2.

Thus, there's a mismatch between how the ``type'' characteristic is
stored (as 1/0) and the way the shapes and colors vectors are indexed
(which start from 1). To fix this, we add +1 to the type values, so that
FALSE (which is stored as 0) becomes 1, and TRUE (which is stored as 1)
becomes 2.

In this network, the second mode (which represents the ``groups'' in the
bipartite network) is always considered to be the ``TRUE'' type. So,
this means that the ``characters'' (the first mode, FALSE or 0) are
displayed with green circles, and the ``groups'' (the second mode, TRUE
or 1) are displayed with orange squares.

Let's visualise different centrality measures on this two-mode network

\begin{Shaded}
\begin{Highlighting}[]
\FunctionTok{par}\NormalTok{(}\AttributeTok{mar =}\FunctionTok{c}\NormalTok{(}\DecValTok{0}\NormalTok{,}\DecValTok{0}\NormalTok{,}\DecValTok{0}\NormalTok{,}\DecValTok{0}\NormalTok{))}
\FunctionTok{plot}\NormalTok{(hp\_aff, }\AttributeTok{vertex.color=}\NormalTok{colors[}\FunctionTok{V}\NormalTok{(hp\_aff)}\SpecialCharTok{$}\NormalTok{type}\SpecialCharTok{+}\DecValTok{1}\NormalTok{],}
     \AttributeTok{vertex.shape=}\NormalTok{shapes[}\FunctionTok{V}\NormalTok{(hp\_aff)}\SpecialCharTok{$}\NormalTok{type}\SpecialCharTok{+}\DecValTok{1}\NormalTok{], }\AttributeTok{vertex.size =} \FunctionTok{betweenness}\NormalTok{(hp\_aff)}\SpecialCharTok{/}\DecValTok{100}\NormalTok{, }\AttributeTok{vertex.label =} \ConstantTok{NA}\NormalTok{)}
\end{Highlighting}
\end{Shaded}

\pandocbounded{\includegraphics[keepaspectratio]{Two-Mode-Networks---AM_files/figure-pdf/unnamed-chunk-6-1.pdf}}

\begin{Shaded}
\begin{Highlighting}[]
\FunctionTok{plot}\NormalTok{(hp\_aff, }\AttributeTok{vertex.color=}\NormalTok{colors[}\FunctionTok{V}\NormalTok{(hp\_aff)}\SpecialCharTok{$}\NormalTok{type}\SpecialCharTok{+}\DecValTok{1}\NormalTok{],}
     \AttributeTok{vertex.shape=}\NormalTok{shapes[}\FunctionTok{V}\NormalTok{(hp\_aff)}\SpecialCharTok{$}\NormalTok{type}\SpecialCharTok{+}\DecValTok{1}\NormalTok{], }\AttributeTok{vertex.label =} \FunctionTok{degree}\NormalTok{(hp\_aff), }\AttributeTok{vertex.label.cex =} \FloatTok{0.75}\NormalTok{)}
\end{Highlighting}
\end{Shaded}

\pandocbounded{\includegraphics[keepaspectratio]{Two-Mode-Networks---AM_files/figure-pdf/unnamed-chunk-6-2.pdf}}

These are great, but do not quite capture the bipartite (two-mode)
nature of the network. Igraph has a specific layout option that can help
emphasise this.

\begin{Shaded}
\begin{Highlighting}[]
\FunctionTok{par}\NormalTok{(}\AttributeTok{mar =}\FunctionTok{c}\NormalTok{(}\DecValTok{0}\NormalTok{,}\DecValTok{0}\NormalTok{,}\DecValTok{0}\NormalTok{,}\DecValTok{0}\NormalTok{))}
\FunctionTok{plot}\NormalTok{(hp\_aff, }\AttributeTok{vertex.color=}\NormalTok{colors[}\FunctionTok{V}\NormalTok{(hp\_aff)}\SpecialCharTok{$}\NormalTok{type}\SpecialCharTok{+}\DecValTok{1}\NormalTok{],}
     \AttributeTok{vertex.shape=}\NormalTok{shapes[}\FunctionTok{V}\NormalTok{(hp\_aff)}\SpecialCharTok{$}\NormalTok{type}\SpecialCharTok{+}\DecValTok{1}\NormalTok{], }\AttributeTok{vertex.label =} \ConstantTok{NA}\NormalTok{, }\AttributeTok{layout =} \FunctionTok{layout\_as\_bipartite}\NormalTok{(hp\_aff))}
\end{Highlighting}
\end{Shaded}

\pandocbounded{\includegraphics[keepaspectratio]{Two-Mode-Networks---AM_files/figure-pdf/unnamed-chunk-7-1.pdf}}

\chapter{Two Mode Networks -
Edgelists}\label{two-mode-networks---edgelists}

\begin{Shaded}
\begin{Highlighting}[]
\FunctionTok{library}\NormalTok{(igraph)}
\end{Highlighting}
\end{Shaded}

There is a slightly different approach to bringing in Two mode network
data from an Edgelist than from an Adjacency matrix. Instead of a two
mode matrix, you may have edgelist data in two mode format. \#\# Two
Mode Edgelists First, bring it in and make it a network object.

\begin{Shaded}
\begin{Highlighting}[]
\NormalTok{hp\_tm\_edgelist }\OtherTok{\textless{}{-}} \FunctionTok{read.csv}\NormalTok{(}\FunctionTok{file.choose}\NormalTok{())}
\FunctionTok{head}\NormalTok{(hp\_tm\_edgelist)}
\end{Highlighting}
\end{Shaded}

\begin{verbatim}
         character      group
1 Albus Dumbledore    Phoenix
2 Albus Dumbledore    Prefect
3 Albus Dumbledore Gryffindor
4      Remus Lupin    Phoenix
5      Remus Lupin    Prefect
6      Remus Lupin Gryffindor
\end{verbatim}

Different from the adjacency matrix, this edge list has one type of node
in one column and the other type in the second column.

This is the same process as any other network. Directed is set to FALSE

\begin{Shaded}
\begin{Highlighting}[]
\NormalTok{hp\_tm\_net }\OtherTok{\textless{}{-}} \FunctionTok{graph\_from\_data\_frame}\NormalTok{(hp\_tm\_edgelist, }\AttributeTok{directed =} \ConstantTok{FALSE}\NormalTok{)}
\end{Highlighting}
\end{Shaded}

Let's check to see if this is actually a two mode network using
bipartite\_mapping. This function goes through the edgelist an ensures
that the columns have distinct nodes in them (i.e.~it is truly a
bipartite or two mod network).

\begin{Shaded}
\begin{Highlighting}[]
\FunctionTok{bipartite\_mapping}\NormalTok{(hp\_tm\_net)}
\end{Highlighting}
\end{Shaded}

\begin{verbatim}
$res
[1] TRUE

$type
      Albus Dumbledore            Remus Lupin          Molly Weasley 
                 FALSE                  FALSE                  FALSE 
          Siruis Black          Severus Snape          Alastor Moody 
                 FALSE                  FALSE                  FALSE 
    Minerva McGonagall          Rubeus Hagrid   Kingsley Shacklebolt 
                 FALSE                  FALSE                  FALSE 
      Nymphadora Tonks     Mundungus Fletcher         Dedalus Diggle 
                 FALSE                  FALSE                  FALSE 
          Elphias Dode   Aberforth Dumbledore          Arabella Figg 
                 FALSE                  FALSE                  FALSE 
        Emmeline Vance        Sturgis Podmore           Hestia Jones 
                 FALSE                  FALSE                  FALSE 
       Aurthur Weasley           Bill Weasley        Charlie Weasley 
                 FALSE                  FALSE                  FALSE 
      Hermione Granger           Harry Potter              Cho Chang 
                 FALSE                  FALSE                  FALSE 
           Ron Weasley         Lavendar Brown         George Weasley 
                 FALSE                  FALSE                  FALSE 
          Fred Weasley     Neville Longbottom          Colin Creevey 
                 FALSE                  FALSE                  FALSE 
         Luna Lovegood            Dean Thomas             Katie Bell 
                 FALSE                  FALSE                  FALSE 
      Angelina Johnson          Hannah Abbott             Lee Jordon 
                 FALSE                  FALSE                  FALSE 
     Anthony Goldstein         Ernie Macmilan Justin Finch-Fletchley 
                 FALSE                  FALSE                  FALSE 
           Padma Patil        Seamus Finnigan            Susan Bones 
                 FALSE                  FALSE                  FALSE 
    Marietta Edgecombe         Alicia Spinnet         Dennis Creevey 
                 FALSE                  FALSE                  FALSE 
         Ginny Weasley          Parvati Patil          Nigel Wolpert 
                 FALSE                  FALSE                  FALSE 
       Cormac McLaggen           Romilda Vane         Michael Corner 
                 FALSE                  FALSE                  FALSE 
            Terry Boot         Maisy Reynolds                 Leanne 
                 FALSE                  FALSE                  FALSE 
       Zacharias Smith            Luca Caruso           Alice Toplin 
                 FALSE                  FALSE                  FALSE 
          James Potter           Lilly Potter         Peter Petigrew 
                 FALSE                  FALSE                  FALSE 
        Fabian Prewett         Gideon Prewett       Frank Longbottom 
                 FALSE                  FALSE                  FALSE 
      Alice Longbottom            Edgar Bones          Benjy Fenwick 
                 FALSE                  FALSE                  FALSE 
      Caradoc Dearborn        Dorcas Meadowes       Marlene McKinnon 
                 FALSE                  FALSE                  FALSE 
        Fleur Delacour    Bellatrix Lestrange          Lucius Malfoy 
                 FALSE                  FALSE                  FALSE 
        Igor Karkaroff          Regulus Black        Barty Crouch Jr 
                 FALSE                  FALSE                  FALSE 
       Antonin Dolohov         Thorfinn Rowle      Augustus Rookwood 
                 FALSE                  FALSE                  FALSE 
           Evan Rosier         Walden Macnair          Alecto Carrow 
                 FALSE                  FALSE                  FALSE 
         Amycus Carrow              Avery Jnr          Corban Yaxley 
                 FALSE                  FALSE                  FALSE 
            Crabbe Snr           Draco Malfoy                 Gibbon 
                 FALSE                  FALSE                  FALSE 
             Goyle Snr                 Jugson           Mulciber Snr 
                 FALSE                  FALSE                  FALSE 
          Mulciber Jnr               Nott Snr     Rabastan Lestrange 
                 FALSE                  FALSE                  FALSE 
   Rodolphus Lestrange                 Tavers        Pansy Parkinson 
                 FALSE                  FALSE                  FALSE 
   Millicent Bulstrode         Vincent Crabbe          Gregory Goyle 
                 FALSE                  FALSE                  FALSE 
       Graham Montague     Cassius Warrington            Argus Filch 
                 FALSE                  FALSE                  FALSE 
      Dolores Umbridge             Jane Court         Gabriel Truman 
                 FALSE                  FALSE                  FALSE 
        Cedric Diggory    Constance Pickering        Natalie Kathryn 
                 FALSE                  FALSE                  FALSE 
            Tom Riddle           Felix Rosier           Gemma Farley 
                 FALSE                  FALSE                  FALSE 
      Penelope Padgett           Rodrick Lyme          Annalena Murk 
                 FALSE                  FALSE                  FALSE 
         Angelica Cole          Percy Weasley       Freddie Clemmons 
                 FALSE                  FALSE                  FALSE 
           Dani Caroll          Marcus Turner    Penelope Clearwater 
                 FALSE                  FALSE                  FALSE 
        Robert Hillard         Chester Davies                Phoenix 
                 FALSE                  FALSE                   TRUE 
               Prefect             Gryffindor           Death.Eaters 
                  TRUE                   TRUE                   TRUE 
             Slytherin             Hufflepuff              Ravenclaw 
                  TRUE                   TRUE                   TRUE 
     Dumbeldore.s.Army    Inquisitorial.Squad 
                  TRUE                   TRUE 
\end{verbatim}

It recognises that there are two types of node in this object, so we can
set that as a vertex characteristic.

\begin{Shaded}
\begin{Highlighting}[]
\FunctionTok{V}\NormalTok{(hp\_tm\_net)}\SpecialCharTok{$}\NormalTok{type }\OtherTok{\textless{}{-}} \FunctionTok{bipartite\_mapping}\NormalTok{(hp\_tm\_net)}\SpecialCharTok{$}\NormalTok{type}
\end{Highlighting}
\end{Shaded}

Now we have changed it into a two mode network and added the
characteristic ``type'' that we are familiar with from wokring on the
adjacecny matrix version.

\begin{Shaded}
\begin{Highlighting}[]
\FunctionTok{V}\NormalTok{(hp\_tm\_net)}\SpecialCharTok{$}\NormalTok{type }
\end{Highlighting}
\end{Shaded}

\begin{verbatim}
  [1] FALSE FALSE FALSE FALSE FALSE FALSE FALSE FALSE FALSE FALSE FALSE FALSE
 [13] FALSE FALSE FALSE FALSE FALSE FALSE FALSE FALSE FALSE FALSE FALSE FALSE
 [25] FALSE FALSE FALSE FALSE FALSE FALSE FALSE FALSE FALSE FALSE FALSE FALSE
 [37] FALSE FALSE FALSE FALSE FALSE FALSE FALSE FALSE FALSE FALSE FALSE FALSE
 [49] FALSE FALSE FALSE FALSE FALSE FALSE FALSE FALSE FALSE FALSE FALSE FALSE
 [61] FALSE FALSE FALSE FALSE FALSE FALSE FALSE FALSE FALSE FALSE FALSE FALSE
 [73] FALSE FALSE FALSE FALSE FALSE FALSE FALSE FALSE FALSE FALSE FALSE FALSE
 [85] FALSE FALSE FALSE FALSE FALSE FALSE FALSE FALSE FALSE FALSE FALSE FALSE
 [97] FALSE FALSE FALSE FALSE FALSE FALSE FALSE FALSE FALSE FALSE FALSE FALSE
[109] FALSE FALSE FALSE FALSE FALSE FALSE FALSE FALSE FALSE FALSE FALSE FALSE
[121] FALSE FALSE  TRUE  TRUE  TRUE  TRUE  TRUE  TRUE  TRUE  TRUE  TRUE
\end{verbatim}

You see the true and false statements, as we expect to see. Since this
is the case, we will need to use the same +1 alteration to the arguments
in our visualisation.

When we plot it, it looks how we expect!

\begin{Shaded}
\begin{Highlighting}[]
\NormalTok{shapes }\OtherTok{\textless{}{-}} \FunctionTok{c}\NormalTok{(}\StringTok{"circle"}\NormalTok{, }\StringTok{"square"}\NormalTok{)}
\NormalTok{colors }\OtherTok{\textless{}{-}}\FunctionTok{c}\NormalTok{(}\StringTok{"green"}\NormalTok{, }\StringTok{"orange"}\NormalTok{)}
\FunctionTok{plot}\NormalTok{(hp\_tm\_net, }\AttributeTok{vertex.color=}\NormalTok{colors[}\FunctionTok{V}\NormalTok{(hp\_tm\_net)}\SpecialCharTok{$}\NormalTok{type}\SpecialCharTok{+}\DecValTok{1}\NormalTok{],}
     \AttributeTok{vertex.shape=}\NormalTok{shapes[}\FunctionTok{V}\NormalTok{(hp\_tm\_net)}\SpecialCharTok{$}\NormalTok{type}\SpecialCharTok{+}\DecValTok{1}\NormalTok{], }\AttributeTok{vertex.label =} \ConstantTok{NA}\NormalTok{)}
\end{Highlighting}
\end{Shaded}

\pandocbounded{\includegraphics[keepaspectratio]{Two-Mode-Networks---Edgelists_files/figure-pdf/unnamed-chunk-7-1.pdf}}

\part{Unit 2: Visualising Network Data}

\chapter{Unit 2}\label{unit-2}

In this unit, we will cover how to visualise network data. By the end of
this unit you will:

\begin{enumerate}
\def\labelenumi{\arabic{enumi}.}
\item
  Produce basic visualisations
\item
  Produce intermediate visualisations that tell a story about your
  network
\item
  Produce interactive or dynamic network visualisations
\end{enumerate}

\section{Project Milestones}\label{project-milestones-1}

\begin{longtable}[]{@{}
  >{\raggedright\arraybackslash}p{(\linewidth - 2\tabcolsep) * \real{0.5278}}
  >{\raggedright\arraybackslash}p{(\linewidth - 2\tabcolsep) * \real{0.4722}}@{}}
\caption{Enjoy!}\tabularnewline
\toprule\noalign{}
\begin{minipage}[b]{\linewidth}\raggedright
Milestone (assignments linked)
\end{minipage} & \begin{minipage}[b]{\linewidth}\raggedright
Explanation
\end{minipage} \\
\midrule\noalign{}
\endfirsthead
\toprule\noalign{}
\begin{minipage}[b]{\linewidth}\raggedright
Milestone (assignments linked)
\end{minipage} & \begin{minipage}[b]{\linewidth}\raggedright
Explanation
\end{minipage} \\
\midrule\noalign{}
\endhead
\bottomrule\noalign{}
\endlastfoot
\href{A3_Data\%20Exploration.qmd}{Data Exploration} & Students will give
descriptions of the network data they are using. They will discuss
possible transformations to the data that they will need to perform
before analysing. They may also provide basic visualisations to
demonstrate this. \\
\href{A4_Visualisations.qmd}{Visualisations} & Students will provide
multiple visualisations of the network they are studying. This
assignment requires students to demonstrate they have learned how to
make basic visualization, clean visualization, and more advanced
visualisations (i.e.~at least one dynamic or interactive version of
their network). \\
\end{longtable}

\chapter{Network Visualisation -
Basic}\label{network-visualisation---basic}

\begin{Shaded}
\begin{Highlighting}[]
\FunctionTok{library}\NormalTok{(igraph)}

\NormalTok{grime\_edge\_list }\OtherTok{\textless{}{-}} \FunctionTok{read.csv}\NormalTok{(}\FunctionTok{file.choose}\NormalTok{(), }\AttributeTok{header =} \ConstantTok{TRUE}\NormalTok{)}

\NormalTok{grime\_08 }\OtherTok{\textless{}{-}} \FunctionTok{graph\_from\_data\_frame}\NormalTok{(}\AttributeTok{d=}\NormalTok{ grime\_edge\_list, }\AttributeTok{directed =} \ConstantTok{TRUE}\NormalTok{)}
\FunctionTok{plot}\NormalTok{(grime\_08)}
\end{Highlighting}
\end{Shaded}

\pandocbounded{\includegraphics[keepaspectratio]{Basic-Visualisation_files/figure-pdf/unnamed-chunk-1-1.pdf}}

\begin{Shaded}
\begin{Highlighting}[]
\NormalTok{grime\_08\_clean }\OtherTok{\textless{}{-}} \FunctionTok{delete.edges}\NormalTok{(grime\_08, }\FunctionTok{E}\NormalTok{(grime\_08)[}\FunctionTok{which\_loop}\NormalTok{(grime\_08)])}

\FunctionTok{plot}\NormalTok{(grime\_08\_clean)}
\end{Highlighting}
\end{Shaded}

\pandocbounded{\includegraphics[keepaspectratio]{Basic-Visualisation_files/figure-pdf/unnamed-chunk-1-2.pdf}}

\section{Layout}\label{layout}

Random

\begin{Shaded}
\begin{Highlighting}[]
\FunctionTok{plot}\NormalTok{(grime\_08\_clean, }\AttributeTok{layout =}\NormalTok{ layout.random)}
\end{Highlighting}
\end{Shaded}

\pandocbounded{\includegraphics[keepaspectratio]{Basic-Visualisation_files/figure-pdf/unnamed-chunk-2-1.pdf}}

Grid

\begin{Shaded}
\begin{Highlighting}[]
\FunctionTok{plot}\NormalTok{(grime\_08\_clean, }\AttributeTok{layout =}\NormalTok{ layout.grid)}
\end{Highlighting}
\end{Shaded}

\pandocbounded{\includegraphics[keepaspectratio]{Basic-Visualisation_files/figure-pdf/unnamed-chunk-3-1.pdf}}

Small world/Circle

\begin{Shaded}
\begin{Highlighting}[]
\FunctionTok{plot}\NormalTok{(grime\_08\_clean, }\AttributeTok{layout =}\NormalTok{ layout.circle)}
\end{Highlighting}
\end{Shaded}

\pandocbounded{\includegraphics[keepaspectratio]{Basic-Visualisation_files/figure-pdf/unnamed-chunk-4-1.pdf}}

Layout by algorithms to demonstrate the relationships a bit better

\begin{Shaded}
\begin{Highlighting}[]
\FunctionTok{plot}\NormalTok{(grime\_08\_clean, }\AttributeTok{layout =} \FunctionTok{layout\_with\_dh}\NormalTok{(grime\_08\_clean))}
\end{Highlighting}
\end{Shaded}

\pandocbounded{\includegraphics[keepaspectratio]{Basic-Visualisation_files/figure-pdf/unnamed-chunk-5-1.pdf}}

\begin{Shaded}
\begin{Highlighting}[]
\FunctionTok{plot}\NormalTok{(grime\_08\_clean, }\AttributeTok{layout =} \FunctionTok{layout\_with\_fr}\NormalTok{(grime\_08\_clean))}
\end{Highlighting}
\end{Shaded}

\pandocbounded{\includegraphics[keepaspectratio]{Basic-Visualisation_files/figure-pdf/unnamed-chunk-5-2.pdf}}

\begin{Shaded}
\begin{Highlighting}[]
\FunctionTok{plot}\NormalTok{(grime\_08\_clean, }\AttributeTok{layout =} \FunctionTok{layout\_with\_kk}\NormalTok{(grime\_08\_clean))}
\end{Highlighting}
\end{Shaded}

\pandocbounded{\includegraphics[keepaspectratio]{Basic-Visualisation_files/figure-pdf/unnamed-chunk-5-3.pdf}}

\section{Changing colours of the
graph}\label{changing-colours-of-the-graph}

Vertex colours with verterx.color()

\begin{Shaded}
\begin{Highlighting}[]
\FunctionTok{plot}\NormalTok{(grime\_08\_clean, }\AttributeTok{vertex.color =} \StringTok{"pink"}\NormalTok{)}
\end{Highlighting}
\end{Shaded}

\pandocbounded{\includegraphics[keepaspectratio]{Basic-Visualisation_files/figure-pdf/unnamed-chunk-6-1.pdf}}

Edge colour with edge.color()

\begin{Shaded}
\begin{Highlighting}[]
\FunctionTok{plot}\NormalTok{(grime\_08\_clean, }\AttributeTok{edge.color =} \StringTok{"pink"}\NormalTok{)}
\end{Highlighting}
\end{Shaded}

\pandocbounded{\includegraphics[keepaspectratio]{Basic-Visualisation_files/figure-pdf/unnamed-chunk-7-1.pdf}}

\section{Changing the Sizes}\label{changing-the-sizes}

Edge and vertex size

\begin{Shaded}
\begin{Highlighting}[]
\FunctionTok{plot}\NormalTok{(grime\_08\_clean, }\AttributeTok{vertex.size =} \DecValTok{5}\NormalTok{, }\AttributeTok{edge.arrow.size =} \FloatTok{0.5}\NormalTok{)}
\end{Highlighting}
\end{Shaded}

\pandocbounded{\includegraphics[keepaspectratio]{Basic-Visualisation_files/figure-pdf/unnamed-chunk-8-1.pdf}}

Edge thickness

\begin{Shaded}
\begin{Highlighting}[]
\FunctionTok{plot}\NormalTok{(grime\_08\_clean, }\AttributeTok{vertex.size =} \DecValTok{5}\NormalTok{, }\AttributeTok{edge.width =} \DecValTok{10}\NormalTok{)}
\end{Highlighting}
\end{Shaded}

\pandocbounded{\includegraphics[keepaspectratio]{Basic-Visualisation_files/figure-pdf/unnamed-chunk-9-1.pdf}}

\section{Titles}\label{titles}

Main Title

\begin{Shaded}
\begin{Highlighting}[]
\FunctionTok{plot}\NormalTok{(grime\_08\_clean, }\AttributeTok{vertex.size =} \DecValTok{5}\NormalTok{, }\AttributeTok{edge.arrow.size =} \FloatTok{0.5}\NormalTok{, }\AttributeTok{main =} \StringTok{"Grime Collaborations 2008"}\NormalTok{)}
\end{Highlighting}
\end{Shaded}

\pandocbounded{\includegraphics[keepaspectratio]{Basic-Visualisation_files/figure-pdf/unnamed-chunk-10-1.pdf}}

Sub title

\begin{Shaded}
\begin{Highlighting}[]
\FunctionTok{plot}\NormalTok{(grime\_08\_clean, }\AttributeTok{vertex.size =} \DecValTok{5}\NormalTok{, }\AttributeTok{edge.arrow.size =} \FloatTok{0.5}\NormalTok{, }\AttributeTok{main =} \StringTok{"Grime Collaborations"}\NormalTok{, }\AttributeTok{sub =} \StringTok{"2008"}\NormalTok{)}
\end{Highlighting}
\end{Shaded}

\pandocbounded{\includegraphics[keepaspectratio]{Basic-Visualisation_files/figure-pdf/unnamed-chunk-11-1.pdf}}

\section{Labels}\label{labels}

Toggle

\begin{Shaded}
\begin{Highlighting}[]
\FunctionTok{plot}\NormalTok{(grime\_08\_clean, }\AttributeTok{vertex.label =} \ConstantTok{NA}\NormalTok{)}
\end{Highlighting}
\end{Shaded}

\pandocbounded{\includegraphics[keepaspectratio]{Basic-Visualisation_files/figure-pdf/unnamed-chunk-12-1.pdf}}

Size and Offset

\begin{Shaded}
\begin{Highlighting}[]
\FunctionTok{plot}\NormalTok{(grime\_08\_clean, }\AttributeTok{vertex.label.cex =} \FloatTok{0.5}\NormalTok{, }\AttributeTok{vertex.label.dist =} \DecValTok{2}\NormalTok{)}
\end{Highlighting}
\end{Shaded}

\pandocbounded{\includegraphics[keepaspectratio]{Basic-Visualisation_files/figure-pdf/unnamed-chunk-13-1.pdf}}

\section{General considerations for
visualizations}\label{general-considerations-for-visualizations}

Changing the aspect ratio by changing the margings of the plot window.

\begin{Shaded}
\begin{Highlighting}[]
\FunctionTok{plot}\NormalTok{(grime\_08\_clean, }\AttributeTok{vertex.label =} \ConstantTok{NA}\NormalTok{)}
\end{Highlighting}
\end{Shaded}

\pandocbounded{\includegraphics[keepaspectratio]{Basic-Visualisation_files/figure-pdf/unnamed-chunk-14-1.pdf}}

\begin{Shaded}
\begin{Highlighting}[]
\FunctionTok{par}\NormalTok{(}\AttributeTok{mar =}\FunctionTok{c}\NormalTok{(}\DecValTok{0}\NormalTok{,}\DecValTok{0}\NormalTok{,}\DecValTok{0}\NormalTok{,}\DecValTok{0}\NormalTok{))}
\FunctionTok{plot}\NormalTok{(grime\_08\_clean, }\AttributeTok{vertex.label =} \ConstantTok{NA}\NormalTok{)}
\end{Highlighting}
\end{Shaded}

\pandocbounded{\includegraphics[keepaspectratio]{Basic-Visualisation_files/figure-pdf/unnamed-chunk-14-2.pdf}}

We have reduced the margins all down to zero.

Set a seed to replicate the same layout every time

\begin{Shaded}
\begin{Highlighting}[]
\FunctionTok{par}\NormalTok{(}\AttributeTok{mar =}\FunctionTok{c}\NormalTok{(}\DecValTok{0}\NormalTok{,}\DecValTok{0}\NormalTok{,}\DecValTok{0}\NormalTok{,}\DecValTok{0}\NormalTok{))}
\FunctionTok{set.seed}\NormalTok{(}\DecValTok{123}\NormalTok{)}
\FunctionTok{plot}\NormalTok{(grime\_08\_clean, }\AttributeTok{vertex.label =} \ConstantTok{NA}\NormalTok{)}
\end{Highlighting}
\end{Shaded}

\pandocbounded{\includegraphics[keepaspectratio]{Basic-Visualisation_files/figure-pdf/unnamed-chunk-15-1.pdf}}

\begin{Shaded}
\begin{Highlighting}[]
\FunctionTok{par}\NormalTok{(}\AttributeTok{mar =}\FunctionTok{c}\NormalTok{(}\DecValTok{0}\NormalTok{,}\DecValTok{0}\NormalTok{,}\DecValTok{0}\NormalTok{,}\DecValTok{0}\NormalTok{))}
\FunctionTok{set.seed}\NormalTok{(}\DecValTok{123}\NormalTok{)}
\FunctionTok{plot}\NormalTok{(grime\_08\_clean, }\AttributeTok{vertex.label =} \ConstantTok{NA}\NormalTok{)}
\end{Highlighting}
\end{Shaded}

\pandocbounded{\includegraphics[keepaspectratio]{Basic-Visualisation_files/figure-pdf/unnamed-chunk-16-1.pdf}}

The layout is replicated - useful for recreating visuals you use.

\section{Plotting visuals next to each
other}\label{plotting-visuals-next-to-each-other}

Vertical

\begin{Shaded}
\begin{Highlighting}[]
\FunctionTok{par}\NormalTok{(}\AttributeTok{mfrow =} \FunctionTok{c}\NormalTok{(}\DecValTok{2}\NormalTok{, }\DecValTok{1}\NormalTok{))}
\FunctionTok{par}\NormalTok{(}\AttributeTok{mar =}\FunctionTok{c}\NormalTok{(}\DecValTok{0}\NormalTok{,}\DecValTok{0}\NormalTok{,}\DecValTok{0}\NormalTok{,}\DecValTok{0}\NormalTok{))}
\FunctionTok{set.seed}\NormalTok{(}\DecValTok{123}\NormalTok{)}
\FunctionTok{plot}\NormalTok{(grime\_08\_clean, }\AttributeTok{vertex.label =} \ConstantTok{NA}\NormalTok{)}
\FunctionTok{set.seed}\NormalTok{(}\DecValTok{123}\NormalTok{)}
\FunctionTok{par}\NormalTok{(}\AttributeTok{mar =}\FunctionTok{c}\NormalTok{(}\DecValTok{0}\NormalTok{,}\DecValTok{0}\NormalTok{,}\DecValTok{0}\NormalTok{,}\DecValTok{0}\NormalTok{))}
\FunctionTok{plot}\NormalTok{(grime\_08\_clean, }\AttributeTok{vertex.label =} \ConstantTok{NA}\NormalTok{, }\AttributeTok{vertex.size =} \FunctionTok{degree}\NormalTok{(grime\_08\_clean))}
\end{Highlighting}
\end{Shaded}

\pandocbounded{\includegraphics[keepaspectratio]{Basic-Visualisation_files/figure-pdf/unnamed-chunk-17-1.pdf}}

Horizontal

\begin{Shaded}
\begin{Highlighting}[]
\FunctionTok{par}\NormalTok{(}\AttributeTok{mfrow =} \FunctionTok{c}\NormalTok{(}\DecValTok{1}\NormalTok{, }\DecValTok{2}\NormalTok{))}
\FunctionTok{set.seed}\NormalTok{(}\DecValTok{123}\NormalTok{)}
\FunctionTok{par}\NormalTok{(}\AttributeTok{mar =}\FunctionTok{c}\NormalTok{(}\DecValTok{0}\NormalTok{,}\DecValTok{0}\NormalTok{,}\DecValTok{0}\NormalTok{,}\DecValTok{0}\NormalTok{))}
\FunctionTok{plot}\NormalTok{(grime\_08\_clean, }\AttributeTok{vertex.label =} \ConstantTok{NA}\NormalTok{)}
\FunctionTok{set.seed}\NormalTok{(}\DecValTok{123}\NormalTok{)}
\FunctionTok{par}\NormalTok{(}\AttributeTok{mar =}\FunctionTok{c}\NormalTok{(}\DecValTok{0}\NormalTok{,}\DecValTok{0}\NormalTok{,}\DecValTok{0}\NormalTok{,}\DecValTok{0}\NormalTok{))}
\FunctionTok{plot}\NormalTok{(grime\_08\_clean, }\AttributeTok{vertex.label =} \ConstantTok{NA}\NormalTok{, }\AttributeTok{vertex.size =} \FunctionTok{degree}\NormalTok{(grime\_08\_clean))}
\end{Highlighting}
\end{Shaded}

\pandocbounded{\includegraphics[keepaspectratio]{Basic-Visualisation_files/figure-pdf/unnamed-chunk-18-1.pdf}}

\chapter{Visualisations -
Intermediate}\label{visualisations---intermediate}

\begin{Shaded}
\begin{Highlighting}[]
\FunctionTok{library}\NormalTok{(igraph)}
\end{Highlighting}
\end{Shaded}

First, I bring in some data on Grime musicians and their collaborations
with each other in 2008. Then I clean the network a little bit by
deleting the self-loops.

\begin{Shaded}
\begin{Highlighting}[]
\NormalTok{grime\_edge\_list }\OtherTok{\textless{}{-}} \FunctionTok{read.csv}\NormalTok{(}\FunctionTok{file.choose}\NormalTok{(), }\AttributeTok{header =} \ConstantTok{TRUE}\NormalTok{)}

\NormalTok{grime\_08 }\OtherTok{\textless{}{-}} \FunctionTok{graph\_from\_data\_frame}\NormalTok{(}\AttributeTok{d=}\NormalTok{ grime\_edge\_list, }\AttributeTok{directed =} \ConstantTok{TRUE}\NormalTok{)}
\FunctionTok{plot}\NormalTok{(grime\_08)}
\end{Highlighting}
\end{Shaded}

\pandocbounded{\includegraphics[keepaspectratio]{Intermediate-Visualisation_files/figure-pdf/unnamed-chunk-2-1.pdf}}

\begin{Shaded}
\begin{Highlighting}[]
\NormalTok{grime\_08\_clean }\OtherTok{\textless{}{-}} \FunctionTok{delete.edges}\NormalTok{(grime\_08, }\FunctionTok{E}\NormalTok{(grime\_08)[}\FunctionTok{which\_loop}\NormalTok{(grime\_08)])}

\FunctionTok{plot}\NormalTok{(grime\_08\_clean)}
\end{Highlighting}
\end{Shaded}

\pandocbounded{\includegraphics[keepaspectratio]{Intermediate-Visualisation_files/figure-pdf/unnamed-chunk-2-2.pdf}}

\section{Visualising Centrality - Node
Size}\label{visualising-centrality---node-size}

Now, using data internal to your graph object you can visualise central
people in the network using the vertex.size() argument.

\subsection{Degree Centrality}\label{degree-centrality}

\begin{Shaded}
\begin{Highlighting}[]
\FunctionTok{par}\NormalTok{(}\AttributeTok{mar =} \FunctionTok{c}\NormalTok{(}\DecValTok{0}\NormalTok{,}\DecValTok{0}\NormalTok{,}\DecValTok{0}\NormalTok{,}\DecValTok{0}\NormalTok{))}
\FunctionTok{plot}\NormalTok{(grime\_08\_clean, }\AttributeTok{vertex.size =} \FunctionTok{degree}\NormalTok{(grime\_08\_clean)}\SpecialCharTok{*}\DecValTok{2}\NormalTok{, }\AttributeTok{edge.arrow.size =} \FloatTok{0.5}\NormalTok{)}
\end{Highlighting}
\end{Shaded}

\pandocbounded{\includegraphics[keepaspectratio]{Intermediate-Visualisation_files/figure-pdf/unnamed-chunk-3-1.pdf}}

\subsection{Betweenness Centrality}\label{betweenness-centrality}

\begin{Shaded}
\begin{Highlighting}[]
\FunctionTok{par}\NormalTok{(}\AttributeTok{mar =} \FunctionTok{c}\NormalTok{(}\DecValTok{0}\NormalTok{,}\DecValTok{0}\NormalTok{,}\DecValTok{0}\NormalTok{,}\DecValTok{0}\NormalTok{))}
\FunctionTok{plot}\NormalTok{(grime\_08\_clean, }\AttributeTok{vertex.size =} \FunctionTok{betweenness}\NormalTok{(grime\_08\_clean)}\SpecialCharTok{*}\DecValTok{2}\NormalTok{, }\AttributeTok{edge.arrow.size =} \FloatTok{0.5}\NormalTok{)}
\end{Highlighting}
\end{Shaded}

\pandocbounded{\includegraphics[keepaspectratio]{Intermediate-Visualisation_files/figure-pdf/unnamed-chunk-4-1.pdf}}

Alternatively, you can use the labels to demonstrate centrality

\section{Visualising Centrality - Node
Labels}\label{visualising-centrality---node-labels}

\subsection{Degree}\label{degree}

\begin{Shaded}
\begin{Highlighting}[]
\FunctionTok{par}\NormalTok{(}\AttributeTok{mar =} \FunctionTok{c}\NormalTok{(}\DecValTok{0}\NormalTok{,}\DecValTok{0}\NormalTok{,}\DecValTok{0}\NormalTok{,}\DecValTok{0}\NormalTok{))}
\FunctionTok{plot}\NormalTok{(grime\_08\_clean, }\AttributeTok{vertex.label =} \FunctionTok{degree}\NormalTok{(grime\_08\_clean))}
\end{Highlighting}
\end{Shaded}

\pandocbounded{\includegraphics[keepaspectratio]{Intermediate-Visualisation_files/figure-pdf/unnamed-chunk-5-1.pdf}}

\subsection{Betweenness}\label{betweenness}

\begin{Shaded}
\begin{Highlighting}[]
\FunctionTok{par}\NormalTok{(}\AttributeTok{mar =} \FunctionTok{c}\NormalTok{(}\DecValTok{0}\NormalTok{,}\DecValTok{0}\NormalTok{,}\DecValTok{0}\NormalTok{,}\DecValTok{0}\NormalTok{))}
\FunctionTok{plot}\NormalTok{(grime\_08\_clean, }\AttributeTok{vertex.label =} \FunctionTok{betweenness}\NormalTok{(grime\_08\_clean))}
\end{Highlighting}
\end{Shaded}

\pandocbounded{\includegraphics[keepaspectratio]{Intermediate-Visualisation_files/figure-pdf/unnamed-chunk-6-1.pdf}}

\section{Visualising Relationships}\label{visualising-relationships}

What about visualising the nature of the relationships? The edges?

\begin{Shaded}
\begin{Highlighting}[]
\FunctionTok{par}\NormalTok{(}\AttributeTok{mar =} \FunctionTok{c}\NormalTok{(}\DecValTok{0}\NormalTok{,}\DecValTok{0}\NormalTok{,}\DecValTok{0}\NormalTok{,}\DecValTok{0}\NormalTok{))}
\FunctionTok{plot}\NormalTok{(grime\_08\_clean, }\AttributeTok{edge.width =} \FunctionTok{E}\NormalTok{(grime\_08\_clean)}\SpecialCharTok{$}\NormalTok{collab\_weight, }\AttributeTok{edge.arrow.size =} \FloatTok{0.5}\NormalTok{, }\AttributeTok{vertex.size =} \DecValTok{6}\NormalTok{, }\AttributeTok{vertex.label =} \ConstantTok{NA}\NormalTok{)}
\end{Highlighting}
\end{Shaded}

\pandocbounded{\includegraphics[keepaspectratio]{Intermediate-Visualisation_files/figure-pdf/unnamed-chunk-7-1.pdf}}

\begin{Shaded}
\begin{Highlighting}[]
\FunctionTok{par}\NormalTok{(}\AttributeTok{mar =} \FunctionTok{c}\NormalTok{(}\DecValTok{0}\NormalTok{,}\DecValTok{0}\NormalTok{,}\DecValTok{0}\NormalTok{,}\DecValTok{0}\NormalTok{))}
\FunctionTok{plot}\NormalTok{(grime\_08\_clean, }\AttributeTok{edge.width =} \FunctionTok{E}\NormalTok{(grime\_08\_clean)}\SpecialCharTok{$}\NormalTok{collab\_weight, }\AttributeTok{edge.label =} \FunctionTok{E}\NormalTok{(grime\_08\_clean)}\SpecialCharTok{$}\NormalTok{collab\_weight, }\AttributeTok{edge.arrow.size =} \FloatTok{0.5}\NormalTok{, }\AttributeTok{vertex.size =} \DecValTok{6}\NormalTok{, }\AttributeTok{vertex.label =} \ConstantTok{NA}\NormalTok{) }
\end{Highlighting}
\end{Shaded}

\pandocbounded{\includegraphics[keepaspectratio]{Intermediate-Visualisation_files/figure-pdf/unnamed-chunk-7-2.pdf}}

\section{Node Level Characteristics}\label{node-level-characteristics}

Next, you can attach data external to the network (i.e.~node
characteristics) and visualise those. I bring in a separate .csv file
that has various variables that pertain to the nodes. These are fake
characteristics that I made up.

\begin{Shaded}
\begin{Highlighting}[]
\NormalTok{grime\_nodes }\OtherTok{\textless{}{-}} \FunctionTok{read.csv}\NormalTok{(}\FunctionTok{file.choose}\NormalTok{()) }
\FunctionTok{head}\NormalTok{(grime\_nodes)}
\end{Highlighting}
\end{Shaded}

\begin{verbatim}
           node fake_sales sex
1       Asher D         10   m
2 Dizzee Rascal         20   m
3 Lethal Bizzle         50   m
4         Wiley         70   m
5   Treble Clef        100   m
6       Shystie         95   f
\end{verbatim}

Take a look at this to see what we have available. 2 variables - fake
sales (continuous) and the artist's sex (categorical - dichotomous)

Now we can create a network object that has both the network data and
the node characteristics. This section uses the vertices = argument
which tells R that there are edges and node characteristics as part of
the network. I also clean the selfloops from this edgelist.

\begin{Shaded}
\begin{Highlighting}[]
\NormalTok{grime\_full }\OtherTok{\textless{}{-}} \FunctionTok{graph\_from\_data\_frame}\NormalTok{(grime\_edge\_list, }\AttributeTok{vertices =}\NormalTok{ grime\_nodes, }\AttributeTok{directed =} \ConstantTok{TRUE}\NormalTok{)}
\NormalTok{grime\_full\_clean }\OtherTok{\textless{}{-}} \FunctionTok{delete.edges}\NormalTok{(grime\_full, }\FunctionTok{E}\NormalTok{(grime\_full)[}\FunctionTok{which\_loop}\NormalTok{(grime\_full)])}
\end{Highlighting}
\end{Shaded}

We can use these node attributes to visualise more about the network

\begin{Shaded}
\begin{Highlighting}[]
\NormalTok{sex }\OtherTok{\textless{}{-}}\FunctionTok{ifelse}\NormalTok{(}\FunctionTok{V}\NormalTok{(grime\_full\_clean)}\SpecialCharTok{$}\NormalTok{sex }\SpecialCharTok{==} \StringTok{"f"}\NormalTok{, }\StringTok{"red"}\NormalTok{, }\StringTok{"white"}\NormalTok{)}

\FunctionTok{par}\NormalTok{(}\AttributeTok{mar =} \FunctionTok{c}\NormalTok{(}\DecValTok{0}\NormalTok{,}\DecValTok{0}\NormalTok{,}\DecValTok{0}\NormalTok{,}\DecValTok{0}\NormalTok{))}
\FunctionTok{plot}\NormalTok{(grime\_full\_clean, }\AttributeTok{vertex.color =}\NormalTok{ sex)}
\end{Highlighting}
\end{Shaded}

\pandocbounded{\includegraphics[keepaspectratio]{Intermediate-Visualisation_files/figure-pdf/unnamed-chunk-10-1.pdf}}

We can also set the vertex characteristics to reflect the continuous
variable. In this case, the artists' fake sales.

\begin{Shaded}
\begin{Highlighting}[]
\FunctionTok{par}\NormalTok{(}\AttributeTok{mar =} \FunctionTok{c}\NormalTok{(}\DecValTok{0}\NormalTok{,}\DecValTok{0}\NormalTok{,}\DecValTok{0}\NormalTok{,}\DecValTok{0}\NormalTok{))}
\FunctionTok{plot}\NormalTok{(grime\_full\_clean, }\AttributeTok{vertex.size =} \FunctionTok{V}\NormalTok{(grime\_full\_clean)}\SpecialCharTok{$}\NormalTok{fake\_sales}\SpecialCharTok{/}\DecValTok{100}\NormalTok{)}
\end{Highlighting}
\end{Shaded}

\pandocbounded{\includegraphics[keepaspectratio]{Intermediate-Visualisation_files/figure-pdf/unnamed-chunk-11-1.pdf}}

\chapter{Interactive Networks
Visualisations}\label{interactive-networks-visualisations}

\begin{Shaded}
\begin{Highlighting}[]
\FunctionTok{library}\NormalTok{(igraph)}
\FunctionTok{library}\NormalTok{(readxl)}
\FunctionTok{library}\NormalTok{(dplyr) }
\FunctionTok{library}\NormalTok{(htmlwidgets) }\CommentTok{\# this helps us work with widgets in markdown}
\FunctionTok{library}\NormalTok{(htmltools) }\CommentTok{\#this allows the widgets to present in the markdown file}
\FunctionTok{library}\NormalTok{(visNetwork) }\CommentTok{\# This is the first interactive network package}
\FunctionTok{library}\NormalTok{(threejs) }\CommentTok{\# This is the second interactive network package}
\end{Highlighting}
\end{Shaded}

So far, we have worked only on static visualisations but there are other
ways you can present network data that are a little more fun and can be
more instructive. In this chapter we are going to work on two
interactive networks and discuss the utility of both. There are multiple
packages in R that help you put together interactive visualisations
using the visnetwork and threejs packages. For a more complete tutorial
on some others, take a look at Katya Ognyanova's website
(https://kateto.net/network-visualization).

\section{Interactive networks using
visNetwork}\label{interactive-networks-using-visnetwork}

First, we will be using the visnetwork to create a network visualisation
that you can click on, move and view labels one at a time. To do this,
we are going to use the data I put together previously that we have used
when working on node and edge characteristics. Like in that tutorial,
you will need to set your own working directory for the below code to
run and pull in each sheet.

\begin{Shaded}
\begin{Highlighting}[]
\NormalTok{vertices.df }\OtherTok{\textless{}{-}} \FunctionTok{read\_excel}\NormalTok{(}\StringTok{"C:/Users/trleppar/Downloads/node and vertex characteristics.xlsx"}\NormalTok{, }\AttributeTok{sheet =} \StringTok{"vertices"}\NormalTok{)}
\NormalTok{edges.df }\OtherTok{\textless{}{-}} \FunctionTok{read\_excel}\NormalTok{(}\StringTok{"C:/Users/trleppar/Downloads/node and vertex characteristics.xlsx"}\NormalTok{, }\AttributeTok{sheet =} \StringTok{"edges"}\NormalTok{)}
\end{Highlighting}
\end{Shaded}

Let's take a look at these objects so you can see how these data are
structured. You will see that the vertices.df object has a few node
characteristics that help describe those in the network. The edges.df
object contains the connections between the nodes in the network and
some information about those connections. These will all come into play
as we construct our interactive networks based on it.

\begin{Shaded}
\begin{Highlighting}[]
\FunctionTok{head}\NormalTok{(vertices.df)}
\end{Highlighting}
\end{Shaded}

\begin{verbatim}
# A tibble: 6 x 4
  name    age role  gender
  <chr> <dbl> <chr> <chr> 
1 A        20 DJ    F     
2 B        25 MC    M     
3 C        21 DJ    F     
4 D        23 crew  M     
5 E        24 MC    M     
6 F        23 MC    F     
\end{verbatim}

\begin{Shaded}
\begin{Highlighting}[]
\FunctionTok{head}\NormalTok{(edges.df)}
\end{Highlighting}
\end{Shaded}

\begin{verbatim}
# A tibble: 6 x 4
  from  to     freq affinity
  <chr> <chr> <dbl> <chr>   
1 A     B         2 pos     
2 A     C         1 neg     
3 A     D         1 pos     
4 A     E         1 neg     
5 A     F         3 neg     
6 E     F         2 pos     
\end{verbatim}

The visNetwork package requires that our each node has a unique ID. This
ID is not separate from the network (i.e.~a row number in a dataframe),
rather, this unique ID MUST MATCH the names of the vertices. THIS IS
IMPORTAT. If you are to create this type of network, you will need both
the edge information and the node information. Remember, the latter, the
node information, can simply consist of just an id column. So, if you
are working with a network that only has a an adjacency matrix or a
network, you can create a dataframe that has the other half you need - a
dataframe with the node id (names).

In our case, we have a network that I have created that has both of
these separate dataframes that we can use to create our network. First,
we need the aforementioned ID for each node. In this case, we can use
the rename() function from the dplyr package to rename the column `name'
to id. The vistNetwork package can now detect the unique ids that define
each node.

\begin{Shaded}
\begin{Highlighting}[]
\NormalTok{vertices.df }\OtherTok{\textless{}{-}}\NormalTok{ vertices.df }\SpecialCharTok{\%\textgreater{}\%}
  \FunctionTok{rename}\NormalTok{(}\AttributeTok{id =}\NormalTok{ name)}
\end{Highlighting}
\end{Shaded}

Now that our data is structured in the correct format, we can use the
visnNetwork() argument to create our interactive network! Outside of
markdown, if you are working in a script file, this will appear in the
viewer window of your Rstudio. Here, however, it will appear in a widget
below the code chunk. Play around with the network. You can scroll in
and out. Select and move a node around in the network.

\begin{Shaded}
\begin{Highlighting}[]
\FunctionTok{visNetwork}\NormalTok{(vertices.df, edges.df)}
\end{Highlighting}
\end{Shaded}

\pandocbounded{\includegraphics[keepaspectratio]{Advanced-Visualisation---Interactive_files/figure-pdf/unnamed-chunk-5-1.pdf}}

The package visNetwork requires that you have certain variables
available to it in order for the visualisation to reflect the
information you have. Below, I demonstrate the label, title, and shadow
options. Label does what you expect, it labels the nodes. The above
visualisation does not have labels because visNetwork did not find a
column in the vertices.df that is called label. Title is an option that
enables you to click on the node and get more information about it.
Shadow shows a small shadow behind the node - you can set this as TRUE
or FALSE.

Below, I set the label as the id. To do this, I have to ensure that R
recognises these as strings or characters. Hense I use the
as.character() function. Then, I set the title of the node to reflect
the node's gender. Then, for fun, I like the shadows!!

\begin{Shaded}
\begin{Highlighting}[]
\NormalTok{vertices.df}\SpecialCharTok{$}\NormalTok{label }\OtherTok{\textless{}{-}} \FunctionTok{as.character}\NormalTok{(vertices.df}\SpecialCharTok{$}\NormalTok{id)}
\NormalTok{vertices.df}\SpecialCharTok{$}\NormalTok{title }\OtherTok{\textless{}{-}}\NormalTok{ vertices.df}\SpecialCharTok{$}\NormalTok{gender}
\NormalTok{vertices.df}\SpecialCharTok{$}\NormalTok{shadow }\OtherTok{\textless{}{-}} \ConstantTok{TRUE}
\FunctionTok{head}\NormalTok{(vertices.df)}
\end{Highlighting}
\end{Shaded}

\begin{verbatim}
# A tibble: 6 x 7
  id      age role  gender label title shadow
  <chr> <dbl> <chr> <chr>  <chr> <chr> <lgl> 
1 A        20 DJ    F      A     F     TRUE  
2 B        25 MC    M      B     M     TRUE  
3 C        21 DJ    F      C     F     TRUE  
4 D        23 crew  M      D     M     TRUE  
5 E        24 MC    M      E     M     TRUE  
6 F        23 MC    F      F     F     TRUE  
\end{verbatim}

The object now has these columns in it! Let's see the visual that it
creates!

\begin{Shaded}
\begin{Highlighting}[]
\FunctionTok{visNetwork}\NormalTok{(vertices.df, edges.df) }
\end{Highlighting}
\end{Shaded}

\pandocbounded{\includegraphics[keepaspectratio]{Advanced-Visualisation---Interactive_files/figure-pdf/unnamed-chunk-7-1.pdf}}

Here is a more complete visualisation using other options available to
you with the visNetwork() function. The width = 100\% ensures that the
visualisation fills the space in the widget. The height option also does
something similar. The background, main, submain, and footer options
show other ways you can alter the visualisation. Fun, right??

\begin{Shaded}
\begin{Highlighting}[]
\FunctionTok{visNetwork}\NormalTok{(vertices.df, edges.df, }\AttributeTok{background=}\StringTok{"orange"}\NormalTok{,}
           \AttributeTok{main=}\StringTok{"TITLE HERE"}\NormalTok{, }\AttributeTok{submain=}\StringTok{"SUB HERE!"}\NormalTok{)}
\end{Highlighting}
\end{Shaded}

\pandocbounded{\includegraphics[keepaspectratio]{Advanced-Visualisation---Interactive_files/figure-pdf/unnamed-chunk-8-1.pdf}}

One final thing that we can do is change the colours of the nodes to
represent the communities they are a part of. I use the loivain
algorithm with the igraph package. To do this, I create a igraph object
from both data frames, run the clustering. Then I designate the
``group'' characteristic of the vertices data frame using the membership
from the clustering object. The new visual

\begin{Shaded}
\begin{Highlighting}[]
\NormalTok{edges }\OtherTok{\textless{}{-}} \FunctionTok{graph\_from\_data\_frame}\NormalTok{(}\AttributeTok{d =}\NormalTok{ edges.df, }\AttributeTok{vertices =}\NormalTok{ vertices.df, }\AttributeTok{directed =}\NormalTok{ F)}
\NormalTok{clust }\OtherTok{\textless{}{-}} \FunctionTok{cluster\_louvain}\NormalTok{(edges)}
\NormalTok{vertices.df}\SpecialCharTok{$}\NormalTok{group }\OtherTok{\textless{}{-}} \FunctionTok{as.factor}\NormalTok{(clust}\SpecialCharTok{$}\NormalTok{membership)}


\FunctionTok{visNetwork}\NormalTok{(vertices.df, edges.df)}\SpecialCharTok{\%\textgreater{}\%}
  \FunctionTok{visOptions}\NormalTok{(}\AttributeTok{highlightNearest =}\NormalTok{ T, }\AttributeTok{nodesIdSelection =}\NormalTok{ T)}
\end{Highlighting}
\end{Shaded}

\pandocbounded{\includegraphics[keepaspectratio]{Advanced-Visualisation---Interactive_files/figure-pdf/unnamed-chunk-9-1.pdf}}

\section{3D interctive networks with
threejs}\label{d-interctive-networks-with-threejs}

If you thought the visNetwork visualisation was cool\ldots{} wait till
you see these ones!!

Here we will cover a second package, threejs and create some slightly
different interactive visualisations. For this one, let's use different
network data. You need an network object created in igraph. We could use
the data we have been using so far by creating an igraph object the way
we have been doing so ar. However, to mix things up, let's use our
familiar Grime network.

Here I bring in the Grime 2008 edgelist, clean it and store it in an
object called grime\_08\_clean. This process if familiar to you now. You
are pros!

\begin{Shaded}
\begin{Highlighting}[]
\NormalTok{grime\_edge\_list }\OtherTok{\textless{}{-}} \FunctionTok{read.csv}\NormalTok{(}\FunctionTok{file.choose}\NormalTok{(), }\AttributeTok{header =} \ConstantTok{TRUE}\NormalTok{)}

\NormalTok{grime\_08 }\OtherTok{\textless{}{-}} \FunctionTok{graph\_from\_data\_frame}\NormalTok{(}\AttributeTok{d=}\NormalTok{ grime\_edge\_list, }\AttributeTok{directed =} \ConstantTok{TRUE}\NormalTok{)}
\FunctionTok{plot}\NormalTok{(grime\_08)}
\end{Highlighting}
\end{Shaded}

\pandocbounded{\includegraphics[keepaspectratio]{Advanced-Visualisation---Interactive_files/figure-pdf/unnamed-chunk-10-1.pdf}}

\begin{Shaded}
\begin{Highlighting}[]
\NormalTok{grime\_08\_clean }\OtherTok{\textless{}{-}} \FunctionTok{delete.edges}\NormalTok{(grime\_08, }\FunctionTok{E}\NormalTok{(grime\_08)[}\FunctionTok{which\_loop}\NormalTok{(grime\_08)])}
\end{Highlighting}
\end{Shaded}

Here we can create an objext that threejs recognises. The function we
need to use is the graphjs() that will convert your network into a 3D
interactive network a bit like a molecular or planetary model. You
should be able to do this with any one-mode network. In this chunk I
create an object that has the 3D network called grime\_08\_3d. Then, I
visualise it.

\begin{Shaded}
\begin{Highlighting}[]
\NormalTok{grime\_08\_3d }\OtherTok{\textless{}{-}} \FunctionTok{graphjs}\NormalTok{(grime\_08\_clean)}

\NormalTok{grime\_08\_3d}
\end{Highlighting}
\end{Shaded}

\pandocbounded{\includegraphics[keepaspectratio]{Advanced-Visualisation---Interactive_files/figure-pdf/unnamed-chunk-11-1.pdf}}

We can make this visualisation a little cleaner by deleting the
isolates. To do this, I use the delete\_vertices() argument from the
igraph package. Then, I recreate an object, now called
grime\_08\_isol\_3d. I also add a title to our visualisation.

\begin{Shaded}
\begin{Highlighting}[]
\NormalTok{grime\_isol }\OtherTok{\textless{}{-}} \FunctionTok{delete\_vertices}\NormalTok{(grime\_08\_clean,  }\FunctionTok{which}\NormalTok{(}\FunctionTok{degree}\NormalTok{(grime\_08\_clean)}\SpecialCharTok{==}\DecValTok{0}\NormalTok{))}

\NormalTok{grime\_08\_isol\_3d }\OtherTok{\textless{}{-}} \FunctionTok{graphjs}\NormalTok{(grime\_isol, }\AttributeTok{main=}\StringTok{"Grime 2008!"}\NormalTok{)}
\NormalTok{grime\_08\_isol\_3d}
\end{Highlighting}
\end{Shaded}

\pandocbounded{\includegraphics[keepaspectratio]{Advanced-Visualisation---Interactive_files/figure-pdf/unnamed-chunk-12-1.pdf}}

Okay, this is great, but we can tell a bit more of a story here. Let's
use something we are familiar with, a node attribute, to help us pull
out a bit more of a story from this visualisation. I want to highlight
highly central nodes in this network and change their colour if they are
highly central (let's say a degree above 3). Once again, the package
threejs looks for specific characteristics of your network to visualise.
One of these is the characteristic ``color''. In the chunk below, I use
the set\_vertex\_attr() function from igraph to create an attribute
called color that threejs can recognise. Then, I use an ifelse()
statement to set the colour of highly central attributes to red and
others white. For sake of comparison, I copy the grime\_isol object we
worked on above, to a new object called grime\_isol\_colour. Although,
you could set the vertex attribute to the object directly. I can also
set the size of the node here to further its readability. The package
threejs recognises the option `size' as being the size of the node in
the visualisation.

\begin{Shaded}
\begin{Highlighting}[]
\NormalTok{grime\_isol\_colour }\OtherTok{\textless{}{-}}\NormalTok{ grime\_isol}

\NormalTok{grime\_isol\_colour }\OtherTok{\textless{}{-}} \FunctionTok{set\_vertex\_attr}\NormalTok{(grime\_isol, }\StringTok{"color"}\NormalTok{, }\AttributeTok{value =} \FunctionTok{ifelse}\NormalTok{(}\FunctionTok{degree}\NormalTok{(grime\_isol) }\SpecialCharTok{\textgreater{}} \DecValTok{3}\NormalTok{, }\StringTok{"red"}\NormalTok{, }\StringTok{"white"}\NormalTok{))}

\FunctionTok{V}\NormalTok{(grime\_isol\_colour)}\SpecialCharTok{$}\NormalTok{size }\OtherTok{\textless{}{-}} \DecValTok{5}

\NormalTok{grime\_08\_isol\_3d\_col }\OtherTok{\textless{}{-}} \FunctionTok{graphjs}\NormalTok{(grime\_isol\_colour, }\AttributeTok{main=}\StringTok{"Major Collaborators in Grime 2008!"}\NormalTok{, }\AttributeTok{bg =} \StringTok{"black"}\NormalTok{)}
\NormalTok{grime\_08\_isol\_3d\_col}
\end{Highlighting}
\end{Shaded}

\pandocbounded{\includegraphics[keepaspectratio]{Advanced-Visualisation---Interactive_files/figure-pdf/unnamed-chunk-13-1.pdf}}

If you would like to save this widget, you can do so using the
savewidget() function from the htmltools package. Then you can call upon
it using the browseURL function. However, in order to make this work
well in a html format, you will need to use the browsable() function on
the object. The chunk below should open an html page with your network
(remove the \#).

\begin{Shaded}
\begin{Highlighting}[]
\NormalTok{grime\_08\_isol\_3d\_col }\OtherTok{\textless{}{-}} \FunctionTok{browsable}\NormalTok{(grime\_08\_isol\_3d\_col)}

\CommentTok{\# saveWidget(grime\_08\_isol\_3d\_col, file="grime\_2008\_JS.html")}
\CommentTok{\# browseURL("grime\_2008\_JS.html")}
\end{Highlighting}
\end{Shaded}

Finally, you can further represent elements of your graph in the 3D
network. For example, you can change the colours of the nodes to reflect
the membership of which community they are in. In this chunk, I use the
infomap community detection algorithm to identify which communities the
nodes are in. Then, I create a new node level characteristic called
``color'' that captures which community they are in. If you have many
communities, you may need to use a colour palette package like vidris or
RColorBrewer,

\begin{Shaded}
\begin{Highlighting}[]
\NormalTok{grime\_coms }\OtherTok{\textless{}{-}}\NormalTok{ grime\_isol}
\NormalTok{cl }\OtherTok{\textless{}{-}} \FunctionTok{cluster\_infomap}\NormalTok{(grime\_coms)}
\FunctionTok{V}\NormalTok{(grime\_coms)}\SpecialCharTok{$}\NormalTok{color }\OtherTok{\textless{}{-}}\NormalTok{ cl}\SpecialCharTok{$}\NormalTok{membership}

\NormalTok{grime\_coms3D }\OtherTok{\textless{}{-}} \FunctionTok{graphjs}\NormalTok{(grime\_coms,   }\AttributeTok{edge.color =} \StringTok{"gray"}\NormalTok{,  }\AttributeTok{bg =} \StringTok{"white"}\NormalTok{, }\AttributeTok{main =} \StringTok{"Communities in Grime 2008"}\NormalTok{)}
\NormalTok{grime\_coms3D}
\end{Highlighting}
\end{Shaded}

\pandocbounded{\includegraphics[keepaspectratio]{Advanced-Visualisation---Interactive_files/figure-pdf/unnamed-chunk-15-1.pdf}}

\section{Closing Thoughts on Interactive
Networks}\label{closing-thoughts-on-interactive-networks}

I think these networks are cool. However, I really think they are a bit
of a gimmick! Their utility is limited at best. For example, they only
really work in online spaces while are completely useless in print. The
click functions and maneuverable attributes of these graphs are fun to
play around with, perhaps useful in grabbing the imagination of readers.
However, many academic uses for network analysis is much easier to
present using static graphs. Still\ldots{} they are loads of fun!

\chapter{Dynamic Network
Visualisations}\label{dynamic-network-visualisations}

\begin{Shaded}
\begin{Highlighting}[]
\FunctionTok{library}\NormalTok{(igraph)}
\end{Highlighting}
\end{Shaded}

You might be interested in networks that change over time. Rather than a
cross-section of relationships, you may have multiple networks that are
taken at different time points. These are called discrete longitudinal
networks. They are discrete because each network represents a discrete,
distinct, point in time. For example, you may have monthly phone call
data between people and their family. You may have yearly romantic
affiliations between a group of people. The point is, the time stamps
are distinct and standardised (yearly, monthly, daily etc.) and the
relationships may change over time. There is such a thing as continuous
longitudinal network data, but we will just focus on discrete networks
for now.

In other tutorials we have been using collaboration data between Grime
musicians. Well, we can measure this over time! This next chunk should
seem familiar to you, they bring in and clean up edgelists of
collaborations every two years from 2008-2014 (four time stamps).

\begin{Shaded}
\begin{Highlighting}[]
\CommentTok{\# 2008}
\NormalTok{edge\_list }\OtherTok{\textless{}{-}} \FunctionTok{read.csv}\NormalTok{(}\FunctionTok{file.choose}\NormalTok{(), }\AttributeTok{header =} \ConstantTok{TRUE}\NormalTok{)}
\NormalTok{grime\_08}\OtherTok{\textless{}{-}} \FunctionTok{graph\_from\_data\_frame}\NormalTok{(}\AttributeTok{d=}\NormalTok{edge\_list, }\AttributeTok{directed =} \ConstantTok{TRUE}\NormalTok{)}
\DocumentationTok{\#\#Removing Loops}
\NormalTok{grime\_08 }\OtherTok{\textless{}{-}} \FunctionTok{delete\_edges}\NormalTok{(grime\_08, }\FunctionTok{E}\NormalTok{(grime\_08)[}\FunctionTok{is.loop}\NormalTok{(grime\_08)])}


\CommentTok{\# 2010}
\NormalTok{edge\_list\_10 }\OtherTok{\textless{}{-}} \FunctionTok{read.csv}\NormalTok{(}\FunctionTok{file.choose}\NormalTok{(), }\AttributeTok{header =} \ConstantTok{TRUE}\NormalTok{)}
\NormalTok{grime\_10}\OtherTok{\textless{}{-}} \FunctionTok{graph\_from\_data\_frame}\NormalTok{(}\AttributeTok{d=}\NormalTok{edge\_list\_10, }\AttributeTok{directed =} \ConstantTok{TRUE}\NormalTok{)}
\DocumentationTok{\#\#Removing Loops}
\NormalTok{grime\_10 }\OtherTok{\textless{}{-}} \FunctionTok{delete\_edges}\NormalTok{(grime\_10, }\FunctionTok{E}\NormalTok{(grime\_10)[}\FunctionTok{is.loop}\NormalTok{(grime\_10)])}

\CommentTok{\# 2012}
\NormalTok{edge\_list\_12 }\OtherTok{\textless{}{-}} \FunctionTok{read.csv}\NormalTok{(}\FunctionTok{file.choose}\NormalTok{(), }\AttributeTok{header =} \ConstantTok{TRUE}\NormalTok{)}
\NormalTok{grime\_12}\OtherTok{\textless{}{-}} \FunctionTok{graph\_from\_data\_frame}\NormalTok{(}\AttributeTok{d=}\NormalTok{edge\_list\_12, }\AttributeTok{directed =} \ConstantTok{TRUE}\NormalTok{)}
\DocumentationTok{\#\#Removing Loops}
\NormalTok{grime\_12 }\OtherTok{\textless{}{-}} \FunctionTok{delete\_edges}\NormalTok{(grime\_12, }\FunctionTok{E}\NormalTok{(grime\_12)[}\FunctionTok{is.loop}\NormalTok{(grime\_12)])}

\CommentTok{\# 2014}
\NormalTok{edge\_list\_14 }\OtherTok{\textless{}{-}} \FunctionTok{read.csv}\NormalTok{(}\FunctionTok{file.choose}\NormalTok{(), }\AttributeTok{header =} \ConstantTok{TRUE}\NormalTok{)}
\NormalTok{grime\_14}\OtherTok{\textless{}{-}} \FunctionTok{graph\_from\_data\_frame}\NormalTok{(}\AttributeTok{d=}\NormalTok{edge\_list\_14, }\AttributeTok{directed =} \ConstantTok{TRUE}\NormalTok{)}
\DocumentationTok{\#\#Removing Loops}
\NormalTok{grime\_14 }\OtherTok{\textless{}{-}} \FunctionTok{delete\_edges}\NormalTok{(grime\_14, }\FunctionTok{E}\NormalTok{(grime\_14)[}\FunctionTok{is.loop}\NormalTok{(grime\_14)])}
\end{Highlighting}
\end{Shaded}

Now we have the networks in, we need to switch gears a little over to
another package that allows us to you create an object known as a
network list. Like it sounds, it is a list of networks. We use the
package intergraph to swap from igraph over to an object type
``Network'' that the packages Networkdynamic and ndtv recognise as they
are the ones we will use for our visualisaitons. We also detach igraph
since we are no longer using that package. We want to detach it so R
does not get confused if there are functions that are similarly worded
across packages.

\begin{Shaded}
\begin{Highlighting}[]
\FunctionTok{library}\NormalTok{(intergraph)}
\NormalTok{grime\_net\_1 }\OtherTok{\textless{}{-}} \FunctionTok{asNetwork}\NormalTok{(grime\_08)}
\NormalTok{grime\_net\_2 }\OtherTok{\textless{}{-}} \FunctionTok{asNetwork}\NormalTok{(grime\_10)}
\NormalTok{grime\_net\_3 }\OtherTok{\textless{}{-}} \FunctionTok{asNetwork}\NormalTok{(grime\_12)}
\NormalTok{grime\_net\_4 }\OtherTok{\textless{}{-}} \FunctionTok{asNetwork}\NormalTok{(grime\_14)}
\end{Highlighting}
\end{Shaded}

If you take a look at these networks, the appear a little different from
our usual igraph object but have the same information stored.

\begin{Shaded}
\begin{Highlighting}[]
\NormalTok{grime\_net\_1}
\end{Highlighting}
\end{Shaded}

\begin{verbatim}
 Network attributes:
  vertices = 40 
  directed = TRUE 
  hyper = FALSE 
  loops = FALSE 
  multiple = FALSE 
  bipartite = FALSE 
  total edges= 28 
    missing edges= 0 
    non-missing edges= 28 

 Vertex attribute names: 
    vertex.names 

 Edge attribute names: 
    collab_weight 
\end{verbatim}

Right, now the fun stuff. We need to bring in our new packages and then
create our dynamic network object using the networkdynamic() function.

\begin{Shaded}
\begin{Highlighting}[]
\FunctionTok{detach}\NormalTok{(}\StringTok{"package:igraph"}\NormalTok{, }\AttributeTok{unload=}\ConstantTok{TRUE}\NormalTok{)}
\FunctionTok{library}\NormalTok{(networkDynamic)}
\FunctionTok{library}\NormalTok{(ndtv)}

\NormalTok{net\_dynamic\_4periods }\OtherTok{\textless{}{-}} \FunctionTok{networkDynamic}\NormalTok{(}\AttributeTok{network.list =} \FunctionTok{list}\NormalTok{(grime\_net\_1, }
\NormalTok{                                                           grime\_net\_2, }
\NormalTok{                                                           grime\_net\_3,}
\NormalTok{                                                           grime\_net\_4), }\AttributeTok{vertex.pid =} \StringTok{"vertex.names"}\NormalTok{)}
\end{Highlighting}
\end{Shaded}

\begin{verbatim}
Neither start or onsets specified, assuming start=0
Onsets and termini not specified, assuming each network in network.list should have a discrete spell of length 1
Argument base.net not specified, using first element of network.list instead
Initialized network of size 91 inferred from number of unique vertex.pids
Created net.obs.period to describe network
 Network observation period info:
  Number of observation spells: 1 
  Maximal time range observed: 0 until 4 
  Temporal mode: discrete 
  Time unit: step 
  Suggested time increment: 1 
\end{verbatim}

Note that you have to state the vertex.names as the id. There are
artists in these networks that come and go (known as joiners or
leavers). setting the vertex.pid ensures that R recognises all artists
based on these unique identifiers.

Let's take a look at this new dynamic network object.

\begin{Shaded}
\begin{Highlighting}[]
\NormalTok{net\_dynamic\_4periods}
\end{Highlighting}
\end{Shaded}

\begin{verbatim}
NetworkDynamic properties:
  distinct change times: 5 
  maximal time range: 0 until  4 

Includes optional net.obs.period attribute:
 Network observation period info:
  Number of observation spells: 1 
  Maximal time range observed: 0 until 4 
  Temporal mode: discrete 
  Time unit: step 
  Suggested time increment: 1 

 Network attributes:
  vertices = 91 
  directed = TRUE 
  hyper = FALSE 
  loops = FALSE 
  multiple = FALSE 
  bipartite = FALSE 
  vertex.pid = vertex.names 
  net.obs.period: (not shown)
  total edges= 188 
    missing edges= 0 
    non-missing edges= 188 

 Vertex attribute names: 
    active vertex.names 

 Edge attribute names: 
    active 
\end{verbatim}

This object tells us all about the network over time. Pay special
attention to the time range. We have four networks and this ojbect
confirms that there are four time periods (0 - 4).

You can make this a data frame to observe changes in tie formation. Pay
attention to the onset and terminus columns. These indicate when ties
form and dissolve. Head and tail columns indicate a ``to'' and ``from''
logic.

\begin{Shaded}
\begin{Highlighting}[]
\NormalTok{net\_dynamic\_4periods\_dat }\OtherTok{\textless{}{-}} \FunctionTok{as.data.frame}\NormalTok{(net\_dynamic\_4periods)}
\FunctionTok{head}\NormalTok{(net\_dynamic\_4periods\_dat)}
\end{Highlighting}
\end{Shaded}

\begin{verbatim}
  onset terminus tail head onset.censored terminus.censored duration edge.id
1     0        1    1   89          FALSE             FALSE        1       1
2     0        1   69   89          FALSE             FALSE        1       2
3     0        1    9   89          FALSE             FALSE        1       3
4     0        1   24   89          FALSE             FALSE        1       4
5     0        1   81   89          FALSE             FALSE        1       5
6     0        1   26   89          FALSE             FALSE        1       6
\end{verbatim}

Now we can start making some visualisations. First, we can create a time
prism of the networks.

\begin{Shaded}
\begin{Highlighting}[]
\FunctionTok{compute.animation}\NormalTok{(net\_dynamic\_4periods)}
\end{Highlighting}
\end{Shaded}

\begin{verbatim}
slice parameters:
  start:0
  end:4
  interval:1
  aggregate.dur:1
  rule:latest
\end{verbatim}

\begin{Shaded}
\begin{Highlighting}[]
\FunctionTok{timePrism}\NormalTok{(net\_dynamic\_4periods,}\AttributeTok{at=}\FunctionTok{c}\NormalTok{(}\DecValTok{0}\NormalTok{,}\DecValTok{1}\NormalTok{, }\DecValTok{2}\NormalTok{,}\DecValTok{3}\NormalTok{),}
          \AttributeTok{displaylabels=}\ConstantTok{FALSE}\NormalTok{,}\AttributeTok{planes =} \ConstantTok{TRUE}\NormalTok{,}
          \AttributeTok{label.cex=}\FloatTok{0.5}\NormalTok{)}
\end{Highlighting}
\end{Shaded}

\pandocbounded{\includegraphics[keepaspectratio]{Advanced-Visualisation---Dynamic_files/figure-pdf/unnamed-chunk-8-1.pdf}}

With this network object, we are ready to look at the changes over time
and present a movie using the render.d3movie() function from ndtv. I
suggest using the output.mode = `htmlWidget' option so it keeps the
video in your rstudio environment. Alternatively, you could use the
launchBrowser= T option to open up an internet page with your video. To
do this, you need to specify the filename. For example,
filename=``Grime-Network.html''.

\begin{Shaded}
\begin{Highlighting}[]
\FunctionTok{render.d3movie}\NormalTok{(net\_dynamic\_4periods, }\AttributeTok{output.mode =} \StringTok{\textquotesingle{}htmlWidget\textquotesingle{}}\NormalTok{)}
\end{Highlighting}
\end{Shaded}

\pandocbounded{\includegraphics[keepaspectratio]{Advanced-Visualisation---Dynamic_files/figure-pdf/unnamed-chunk-9-1.pdf}}

The play buttons on the bottom right operate the video. You can alter
the speed of the transitions between the time points by using the
options menu on the top right.

You can alter the appearance of the movie in a similar way as you can
change the colours and other elements of the network.

\begin{Shaded}
\begin{Highlighting}[]
\FunctionTok{render.d3movie}\NormalTok{(net\_dynamic\_4periods, }\AttributeTok{usearrows =}\NormalTok{ F, }\AttributeTok{displaylabels =}\NormalTok{ F, }\AttributeTok{bg=}\StringTok{"black"}\NormalTok{, }\AttributeTok{vertex.border=}\StringTok{"white"}\NormalTok{, }\AttributeTok{vertex.col =}  \StringTok{"blue"}\NormalTok{, }\AttributeTok{edge.col =} \StringTok{"orange"}\NormalTok{, }\AttributeTok{output.mode =} \StringTok{\textquotesingle{}htmlWidget\textquotesingle{}}\NormalTok{) }
\end{Highlighting}
\end{Shaded}

\pandocbounded{\includegraphics[keepaspectratio]{Advanced-Visualisation---Dynamic_files/figure-pdf/unnamed-chunk-10-1.pdf}}

\part{Unit 3: Analysing Network Data}

\chapter{Unit 3}\label{unit-3}

In this unit, we will cover how to analyse network data. By the end of
this unit you will:

\begin{enumerate}
\def\labelenumi{\arabic{enumi}.}
\item
  Study people in networks
\item
  Identify communities within networks
\item
  Analyse networks of people
\item
  Measure symbolic connections between individuals and groups
\end{enumerate}

\section{Project Milestones}\label{project-milestones-2}

\begin{longtable}[]{@{}
  >{\raggedright\arraybackslash}p{(\linewidth - 2\tabcolsep) * \real{0.5000}}
  >{\raggedright\arraybackslash}p{(\linewidth - 2\tabcolsep) * \real{0.5000}}@{}}
\toprule\noalign{}
\begin{minipage}[b]{\linewidth}\raggedright
Milestone (assignments linked)
\end{minipage} & \begin{minipage}[b]{\linewidth}\raggedright
Explanation
\end{minipage} \\
\midrule\noalign{}
\endhead
\bottomrule\noalign{}
\endlastfoot
\href{Final\%20Project\%20Instructions.qmd}{Final Project} & Students
will perform an exploratory analysis using the data they have been
working on all semester long. The aim of this project is for you to do
an in-depth exploration which includes descriptions, visualisations, and
some primary analysis. ~ \\
\end{longtable}

Enjoy!

\chapter{People in Networks}\label{people-in-networks}

One way to analyse a social network is to analyse the people in those
networks. In general terms,we can report on individuals based on the
structural position they hold within a network. The concept and various
measures of centrality allows us to do so. Ultimately, as the scientist,
you can tell stories about the individuals in your network based on how
central of peripheral they are in the group. In other words, measures of
centrality tell stories about the individuals in your network.

I encourage you to think about these measures in terms of an individuals
prominence and influence over the network. In this chapter, we will be
using four measures, degree centrality, betweenness centrality,
closeness centrality, and constraint. First, let's go through these
one-by-one showing how to measure these in igraph. I won't go too deep
into the maths behind the algorithms, but will focus more on
substantively what it means. Then, I will show you how to construct a
data frame with all of this information so you can then use these to run
further analysis (i.e.~perhaps regressing centrality on career success).
You may have multiple years worth of networks and could create line
graphs of centrality for each node over time to see their personal
change overtime.

Before we begin, let's bring in the Grime data we have been using so
far. We know by now we need to remove self loops in this edgelist.

\begin{Shaded}
\begin{Highlighting}[]
\FunctionTok{library}\NormalTok{(igraph)}

\NormalTok{grime\_edge\_list }\OtherTok{\textless{}{-}} \FunctionTok{read.csv}\NormalTok{(}\FunctionTok{file.choose}\NormalTok{(), }\AttributeTok{header =} \ConstantTok{TRUE}\NormalTok{)}

\NormalTok{grime\_08 }\OtherTok{\textless{}{-}} \FunctionTok{graph\_from\_data\_frame}\NormalTok{(}\AttributeTok{d=}\NormalTok{ grime\_edge\_list, }\AttributeTok{directed =} \ConstantTok{TRUE}\NormalTok{)}
\FunctionTok{plot}\NormalTok{(grime\_08)}
\end{Highlighting}
\end{Shaded}

\pandocbounded{\includegraphics[keepaspectratio]{People-In-Networks_files/figure-pdf/unnamed-chunk-1-1.pdf}}

\begin{Shaded}
\begin{Highlighting}[]
\NormalTok{grime\_08\_clean }\OtherTok{\textless{}{-}} \FunctionTok{delete.edges}\NormalTok{(grime\_08, }\FunctionTok{E}\NormalTok{(grime\_08)[}\FunctionTok{which\_loop}\NormalTok{(grime\_08)])}

\FunctionTok{plot}\NormalTok{(grime\_08\_clean)}
\end{Highlighting}
\end{Shaded}

\pandocbounded{\includegraphics[keepaspectratio]{People-In-Networks_files/figure-pdf/unnamed-chunk-1-2.pdf}}

\section{Measures of Centrality}\label{measures-of-centrality}

You may be interested in finding out who are some prominent individuals
in your network. When I say prominent, I really mean those what are
``well connected'' or perhaps ``popular''. Someone who is very prominent
in the network is likely positioned very central compared to others.

\subsection{Degree Centrality and
Strength}\label{degree-centrality-and-strength}

Degree centrality is simply a count of the number of ties that a node
has with others in the network. Put simply, it is a count of how many
neighbours any given node has.

\begin{Shaded}
\begin{Highlighting}[]
\FunctionTok{degree}\NormalTok{(grime\_08\_clean)}
\end{Highlighting}
\end{Shaded}

\begin{verbatim}
        Asher D   Dizzee Rascal   Lethal Bizzle        Scorcher     Bless Beats 
              1               0               0               1               2 
        Flowdan  Tinchy Stryder          Frisco            Kano     Treble Clef 
              2               1               2               1               0 
        Shystie          Blacks         Badness         Chronik         Tempa T 
              0               1               1               1               1 
Newham Generals          Skepta             JME            Chip             BBK 
              1               1               1               4               1 
Virus Syndicate          Ghetts        Mercston        Double S        Griminal 
              0               1               0               1               3 
        Ice Kid Nu Brand Flexxx       Wretch 32           Wiley  Bossman Birdie 
              2               0               2               8               1 
    The Streets            Sway    Tinie Tempah           Giggs          Jammer 
              0               0               0               0              10 
      Roll Deep          Devlin        P Money     Lauren Mason     Milli Major 
              2               1               1               1               1 
\end{verbatim}

There is a variation of degree centrality called strength(), which
accounts for the weighted nature of the edges. Instead of just counting
the number of neighbours, it sums the weights of the connections to each
neighbor, reflecting both the number of connections and their strength
(or whatever the edge weight represents).

\begin{Shaded}
\begin{Highlighting}[]
\FunctionTok{strength}\NormalTok{(grime\_08\_clean)}
\end{Highlighting}
\end{Shaded}

\begin{verbatim}
        Asher D   Dizzee Rascal   Lethal Bizzle        Scorcher     Bless Beats 
              1               0               0               1               2 
        Flowdan  Tinchy Stryder          Frisco            Kano     Treble Clef 
              2               1               2               1               0 
        Shystie          Blacks         Badness         Chronik         Tempa T 
              0               1               1               1               1 
Newham Generals          Skepta             JME            Chip             BBK 
              1               1               1               4               1 
Virus Syndicate          Ghetts        Mercston        Double S        Griminal 
              0               1               0               1               3 
        Ice Kid Nu Brand Flexxx       Wretch 32           Wiley  Bossman Birdie 
              2               0               2               8               1 
    The Streets            Sway    Tinie Tempah           Giggs          Jammer 
              0               0               0               0              10 
      Roll Deep          Devlin        P Money     Lauren Mason     Milli Major 
              2               1               1               1               1 
\end{verbatim}

So, we see that Wiley and Jammer are very popular in this network.
However, since this is a directed network, there is even more that we
can tease out from this in terms of their outgoing and incoming
connections. Put simply, we can get the individual's strength (i.e.~the
number of connections and edge weight). We do this by using the `mode ='
argument from the strength() function. ``in'' takes the incoming ties
and ``out'' gets the outgoing ties.

\begin{Shaded}
\begin{Highlighting}[]
\FunctionTok{strength}\NormalTok{(grime\_08\_clean, }\AttributeTok{mode =} \StringTok{"in"}\NormalTok{, }\AttributeTok{loops =} \ConstantTok{FALSE}\NormalTok{)}
\end{Highlighting}
\end{Shaded}

\begin{verbatim}
        Asher D   Dizzee Rascal   Lethal Bizzle        Scorcher     Bless Beats 
              0               0               0               0               0 
        Flowdan  Tinchy Stryder          Frisco            Kano     Treble Clef 
              0               0               0               0               0 
        Shystie          Blacks         Badness         Chronik         Tempa T 
              0               0               0               0               0 
Newham Generals          Skepta             JME            Chip             BBK 
              0               0               0               3               0 
Virus Syndicate          Ghetts        Mercston        Double S        Griminal 
              0               0               0               0               1 
        Ice Kid Nu Brand Flexxx       Wretch 32           Wiley  Bossman Birdie 
              0               0               1               7               0 
    The Streets            Sway    Tinie Tempah           Giggs          Jammer 
              0               0               0               0              10 
      Roll Deep          Devlin        P Money     Lauren Mason     Milli Major 
              2               1               1               1               1 
\end{verbatim}

\begin{Shaded}
\begin{Highlighting}[]
\FunctionTok{strength}\NormalTok{(grime\_08\_clean, }\AttributeTok{mode =} \StringTok{"out"}\NormalTok{, }\AttributeTok{loops =} \ConstantTok{FALSE}\NormalTok{)}
\end{Highlighting}
\end{Shaded}

\begin{verbatim}
        Asher D   Dizzee Rascal   Lethal Bizzle        Scorcher     Bless Beats 
              1               0               0               1               2 
        Flowdan  Tinchy Stryder          Frisco            Kano     Treble Clef 
              2               1               2               1               0 
        Shystie          Blacks         Badness         Chronik         Tempa T 
              0               1               1               1               1 
Newham Generals          Skepta             JME            Chip             BBK 
              1               1               1               1               1 
Virus Syndicate          Ghetts        Mercston        Double S        Griminal 
              0               1               0               1               2 
        Ice Kid Nu Brand Flexxx       Wretch 32           Wiley  Bossman Birdie 
              2               0               1               1               1 
    The Streets            Sway    Tinie Tempah           Giggs          Jammer 
              0               0               0               0               0 
      Roll Deep          Devlin        P Money     Lauren Mason     Milli Major 
              0               0               0               0               0 
\end{verbatim}

\subsection{Closeness Centrality}\label{closeness-centrality}

Another measurement of centrality that captures the prominence of a node
is closeness centrality. The closeness centrality measure uses the
average length of the shortest path from one node to all other nodes in
the network. Simply, it measures how close a node is to others in the
network. A node with a high closeness centrality is considered more
prominent in the network because they can reach all others in the
network quickly.

\begin{Shaded}
\begin{Highlighting}[]
\FunctionTok{closeness}\NormalTok{(grime\_08\_clean)}
\end{Highlighting}
\end{Shaded}

\begin{verbatim}
        Asher D   Dizzee Rascal   Lethal Bizzle        Scorcher     Bless Beats 
      0.3333333             NaN             NaN       0.3333333       0.2500000 
        Flowdan  Tinchy Stryder          Frisco            Kano     Treble Clef 
      0.2500000       0.3333333       0.2500000       0.3333333             NaN 
        Shystie          Blacks         Badness         Chronik         Tempa T 
            NaN       1.0000000       1.0000000       1.0000000       1.0000000 
Newham Generals          Skepta             JME            Chip             BBK 
      1.0000000       1.0000000       1.0000000       1.0000000       1.0000000 
Virus Syndicate          Ghetts        Mercston        Double S        Griminal 
            NaN       1.0000000             NaN       0.3333333       0.2500000 
        Ice Kid Nu Brand Flexxx       Wretch 32           Wiley  Bossman Birdie 
      0.2500000             NaN       0.1666667       1.0000000       1.0000000 
    The Streets            Sway    Tinie Tempah           Giggs          Jammer 
            NaN             NaN             NaN             NaN             NaN 
      Roll Deep          Devlin        P Money     Lauren Mason     Milli Major 
            NaN             NaN             NaN             NaN             NaN 
\end{verbatim}

Closeness is a bit of a strange one. You have to take into consideration
the direction of the network. At its core, the maths behind this measure
identifies the possible paths that lead one node to another in the
network. This takes into consideration the direction of the network. You
have to identify the number of ``steps'' it takes one node to reach all
the possible nodes connected to them across the network. Then, you
divide that score by 1 and this gives the closeness score.

Take Frisco for example (this is one of the nodes that sits in between
the two big hubs). Because of the direction of the network, the furthest
node across the network that Frisco can reach is Lauren Mason through
Wiley. Going the other way, the furthest Frisco can go is to Jammer. So,
this node can reach 3 alters. It takes one step to Jammer, 1 step to
Wiley, and 1 step to Lauren Mason. This is a total distance of 4 steps.
1/4 = 0.25 closeness centrality score.

Some of you noted that certain nodes had a score of 1 - like Tempa T. If
you look at Tempa T, this node appears to be connected to a big hub of
the network (Jammer's). However, pay attention to the direction of the
ties. Jammer is as far as Tempa T can get. So, the maths for his
closeness score is 1 step. 1/1 = 1.

This is a weakness of closeness centrality! Because a node with a
closeness of 1 may only have 1 connection. But a node with a closeness
of .25 (say Frisco) has more! This is why we use multiple scores to tell
our story. Mainly, I tend to use degree (or directed strength) and
betweenness since they are more robust.

\subsection{Betweenness Centrality}\label{betweenness-centrality-1}

Next, let's talk about influential people in a network. Influence has to
do with controlling the flow of resources (e.g.~information etc.) within
a network or rather, being involved with the transition of resources.
The most influential individuals, perhaps you could think of them as
powerful, are highly ``central'' to the ongoings of the network.
Betweenness centrality is a measure that can help us capture this type
of information.

Betweenness centrality measures the number of times an individual node
lies on the shortest path (geodesic) between any two other nodes in the
network. A geodesic is the shortest possible path connecting two nodes.
Imagine a network as a set of people connected by relationships, where
you are trying to travel from one person to another. The shortest path
represents the quickest way to get from one node to another in the
network. Betweenness centrality counts how often a specific node appears
as a part of these shortest paths, reflecting its role in connecting
different parts of the network.

For this measure, whether the network is directed is important. Consider
ties as roads: if there are two lanes (one for going and one for
returning), information can flow in both directions. However, if there
is only a one-way road, information can only travel in one direction. In
a directed network, a node may appear on the geodesic, but if it only
has outgoing connections, it might not have much influence on others, as
it cannot receive information from other nodes.

\begin{Shaded}
\begin{Highlighting}[]
\FunctionTok{betweenness}\NormalTok{(grime\_08\_clean)}
\end{Highlighting}
\end{Shaded}

\begin{verbatim}
        Asher D   Dizzee Rascal   Lethal Bizzle        Scorcher     Bless Beats 
              0               0               0               0               0 
        Flowdan  Tinchy Stryder          Frisco            Kano     Treble Clef 
              0               0               0               0               0 
        Shystie          Blacks         Badness         Chronik         Tempa T 
              0               0               0               0               0 
Newham Generals          Skepta             JME            Chip             BBK 
              0               0               0               4               0 
Virus Syndicate          Ghetts        Mercston        Double S        Griminal 
              0               0               0               0               2 
        Ice Kid Nu Brand Flexxx       Wretch 32           Wiley  Bossman Birdie 
              0               0               0               7               0 
    The Streets            Sway    Tinie Tempah           Giggs          Jammer 
              0               0               0               0               0 
      Roll Deep          Devlin        P Money     Lauren Mason     Milli Major 
              0               0               0               0               0 
\end{verbatim}

Compare these numbers (the betweenness) with the artist's degree
centrality. What do you notice? You may notice that those who are highly
prominent aren't necessarily the most influential. Take Wiley, for
example. He is among the most central (a high degree) but, he has low
influence. Why is this? Well, look at the direction of the ties! He has
a lot of incoming ties and not many outgoing meaning that he cannot (if
we stick with the directedness of the graph) influence others.

\subsection{Constraint}\label{constraint}

Constraint is a measure of a node's embeddedness in a specific portion
of the network, indicating how much its connections are limited to a
small set of other nodes. A node with high constraint has fewer
opportunities to interact with a broader range of nodes, as its ties are
concentrated within a small cluster. This measure assesses the diversity
of connections, not in terms of the types of ties (such as strong or
weak edges), but in terms of the variety of other nodes it connects to.

Constraint is considered a measure of influence because, those who have
a low constraint have access to diverse resources within the network.
They may bridge across subgroups in the network which means that they
may have access to nonredundant (new) resources.

\begin{Shaded}
\begin{Highlighting}[]
\FunctionTok{constraint}\NormalTok{(grime\_08\_clean)}
\end{Highlighting}
\end{Shaded}

\begin{verbatim}
        Asher D   Dizzee Rascal   Lethal Bizzle        Scorcher     Bless Beats 
      1.0000000             NaN             NaN       1.0000000       0.5000000 
        Flowdan  Tinchy Stryder          Frisco            Kano     Treble Clef 
      0.5000000       1.0000000       0.5000000       1.0000000             NaN 
        Shystie          Blacks         Badness         Chronik         Tempa T 
            NaN       1.0000000       1.0000000       1.0000000       1.0000000 
Newham Generals          Skepta             JME            Chip             BBK 
      1.0000000       1.0000000       1.0000000       0.2500000       1.0000000 
Virus Syndicate          Ghetts        Mercston        Double S        Griminal 
            NaN       1.0000000             NaN       1.0000000       0.5555556 
        Ice Kid Nu Brand Flexxx       Wretch 32           Wiley  Bossman Birdie 
      0.5000000             NaN       1.0000000       0.1250000       1.0000000 
    The Streets            Sway    Tinie Tempah           Giggs          Jammer 
            NaN             NaN             NaN             NaN       0.1000000 
      Roll Deep          Devlin        P Money     Lauren Mason     Milli Major 
      0.5000000       1.0000000       1.0000000       1.0000000       1.0000000 
\end{verbatim}

\subsection{Visualising central nodes}\label{visualising-central-nodes}

The vertex.size argument that accompanies the plot() function in igraph
gives you the option to visualise those in the network based on their
centrality scores.

Let's say we want to visualise the degee centrality in order to
demonstrate who is prominent in the network. Use the chunk below and see
how we combine the plot() with the degree() functions from igraph.

\begin{Shaded}
\begin{Highlighting}[]
\FunctionTok{plot}\NormalTok{(grime\_08\_clean, }\AttributeTok{vertex.size =} \FunctionTok{degree}\NormalTok{(grime\_08\_clean))}
\end{Highlighting}
\end{Shaded}

\pandocbounded{\includegraphics[keepaspectratio]{People-In-Networks_files/figure-pdf/unnamed-chunk-9-1.pdf}}

Pretty cool, right? Bear in mind that this literally assigned the node
size to equal the value of the degree centrality. So, if a node has a
degree value of 100, the their size will equal 100. This will likely
crowd out other nodes in your visualisation. If this occurs, you may try
dividing the degree scores by a constant (for example. 10). This will
divide every node's degree score, so the visual will still represent the
relative differences between their degree. To do this simply add a
division after the degree() argument. If the opposite occurs (i.e.~the
node's degree centrality is too small to visualise) you can multiply it
by a constant. See the examples below.

\begin{Shaded}
\begin{Highlighting}[]
\FunctionTok{par}\NormalTok{(}\AttributeTok{mfrow =} \FunctionTok{c}\NormalTok{(}\DecValTok{1}\NormalTok{, }\DecValTok{2}\NormalTok{))}
\FunctionTok{set.seed}\NormalTok{(}\DecValTok{123}\NormalTok{)}
\FunctionTok{plot}\NormalTok{(grime\_08\_clean, }\AttributeTok{vertex.size =} \FunctionTok{degree}\NormalTok{(grime\_08\_clean)}\SpecialCharTok{/}\DecValTok{2}\NormalTok{, }\AttributeTok{sub =} \StringTok{"Node Size = Degree/2"}\NormalTok{)}
\FunctionTok{set.seed}\NormalTok{(}\DecValTok{123}\NormalTok{)}
\FunctionTok{plot}\NormalTok{(grime\_08\_clean, }\AttributeTok{vertex.size =} \FunctionTok{degree}\NormalTok{(grime\_08\_clean)}\SpecialCharTok{*}\DecValTok{2}\NormalTok{, }\AttributeTok{sub =} \StringTok{"Node Size = Degree*2"}\NormalTok{)}
\end{Highlighting}
\end{Shaded}

\pandocbounded{\includegraphics[keepaspectratio]{People-In-Networks_files/figure-pdf/unnamed-chunk-10-1.pdf}}

\subsection{Final thoughts on
measures}\label{final-thoughts-on-measures}

There are many measures that you can use to tell a story about
individuals in your network. They all rely on sightly different maths
and algorithms. As such, I encourage you to learn, at least on a
fundamental level what each measure captures so you can understant what
the measure is really telling you.

Furthermore, it may make sense to remove isolates from the network and
only keep the main component of the network. Depending on what measures
you are using, isolates are often left out. For example, you can't get
the constraint measure for an isolate. So, you may consider only
including those with connections in your analysis. This depends on what
you are trying to do. I encourage you to be mindful of what you are
asking and then consider if you need everyone in the network
(i.e.~including isolates) or if your question only involves those with
connections to others.

\section{Tabulating Centrality
Measures}\label{tabulating-centrality-measures}

We can now make this a dataframe with all the information that we have
created. To do this, we need to create objects (vectors) with all of the
measures we care for. Then we can assign these to a data frame that we
can use.

So far, you have been just using the arguments directly and seeing them
populate in the console. A good practice is to make them objects so you
can use them in the future. This way, you do not have to re run the
function every time. To do this, simply use the assignment operator.

\begin{Shaded}
\begin{Highlighting}[]
\NormalTok{degree }\OtherTok{\textless{}{-}} \FunctionTok{degree}\NormalTok{(grime\_08\_clean)}
\NormalTok{strength }\OtherTok{\textless{}{-}} \FunctionTok{strength}\NormalTok{(grime\_08\_clean)}
\NormalTok{close }\OtherTok{\textless{}{-}} \FunctionTok{closeness}\NormalTok{(grime\_08\_clean)}
\NormalTok{betweenness }\OtherTok{\textless{}{-}} \FunctionTok{betweenness}\NormalTok{(grime\_08\_clean)}
\NormalTok{constraint }\OtherTok{\textless{}{-}} \FunctionTok{constraint}\NormalTok{(grime\_08\_clean)}
\end{Highlighting}
\end{Shaded}

Note that these all create vector value objects in your environment from
which you can create a table with all of these. You ca combine these
into a table using the data.frame() function then list each column
setting them equal to the appropriate vectors. I also create a column
with the year that these data are pulled from. Finally, I name the
column that lists the node names.

\begin{Shaded}
\begin{Highlighting}[]
\NormalTok{node\_data }\OtherTok{\textless{}{-}} \FunctionTok{data.frame}\NormalTok{(}
  \AttributeTok{grime\_degree =}\NormalTok{ degree,}
  \AttributeTok{grime\_strength =}\NormalTok{ strength,}
  \AttributeTok{grime\_close =}\NormalTok{ close,}
  \AttributeTok{grime\_constraint =}\NormalTok{ constraint,}
  \AttributeTok{grime\_betweenness =}\NormalTok{ betweenness,}
  \AttributeTok{year =} \DecValTok{2008}\NormalTok{,}
  \AttributeTok{name =} \FunctionTok{V}\NormalTok{(grime\_08\_clean)}\SpecialCharTok{$}\NormalTok{name}
\NormalTok{)}

\FunctionTok{head}\NormalTok{(node\_data)}
\end{Highlighting}
\end{Shaded}

\begin{verbatim}
              grime_degree grime_strength grime_close grime_constraint
Asher D                  1              1   0.3333333              1.0
Dizzee Rascal            0              0         NaN              NaN
Lethal Bizzle            0              0         NaN              NaN
Scorcher                 1              1   0.3333333              1.0
Bless Beats              2              2   0.2500000              0.5
Flowdan                  2              2   0.2500000              0.5
              grime_betweenness year          name
Asher D                       0 2008       Asher D
Dizzee Rascal                 0 2008 Dizzee Rascal
Lethal Bizzle                 0 2008 Lethal Bizzle
Scorcher                      0 2008      Scorcher
Bless Beats                   0 2008   Bless Beats
Flowdan                       0 2008       Flowdan
\end{verbatim}

Now, you can use your imagination on the types of things you can do with
this information. What types of analyses can you think of? Maybe
centrality is associated with some other attribute or variable. This
network, for example, is of musicians. Perhaps centrality is associated
with success, popularity, or other accolades. Perhaps, central positions
are filled more by one type of artist than another (e.g.~male/female,
one form of music vs another). You could perform correlations,
differences of means, regressions etc. The possibilities are plenty!!

\chapter{Communities In Networks}\label{communities-in-networks}

\begin{Shaded}
\begin{Highlighting}[]
\FunctionTok{library}\NormalTok{(igraph)}
\FunctionTok{library}\NormalTok{(dplyr)}
\end{Highlighting}
\end{Shaded}

Within your network you may have subgroups. In a class of students,
there are often smaller homework groups, study groups, friendship
groups, etc.. What is the demographic composition of these groups? Do
subgroups appear more in some settings compared to others? What happens
to these groups over time? To answer these types of questions you need a
specific analytic tool, community detection.

At its core, community detection is a method used to analyse the sub
components(groups) of your graph. It literally detects whether there are
smaller groups within a network. Using mathematically informed
algorithms, various community detection approaches produce measurements
of how many sub groups there are in your graph and how cohesive those
subgroups are.

For this tutorial I am using the Grime collaboration network data that
we have been using a lot. I am going to bring in the data from 2008 and
clean up the graph a bit before we dive in. This network is directed,
but for the sake of the tutorial, I bring it in as an undirected graph
then swap. You will read why.

\begin{Shaded}
\begin{Highlighting}[]
\NormalTok{grime\_edge\_list }\OtherTok{\textless{}{-}} \FunctionTok{read.csv}\NormalTok{(}\FunctionTok{file.choose}\NormalTok{(), }\AttributeTok{header =} \ConstantTok{TRUE}\NormalTok{)}

\NormalTok{grime\_08 }\OtherTok{\textless{}{-}} \FunctionTok{graph\_from\_data\_frame}\NormalTok{(}\AttributeTok{d=}\NormalTok{ grime\_edge\_list, }\AttributeTok{directed =} \ConstantTok{FALSE}\NormalTok{)}
\NormalTok{grime\_08\_clean }\OtherTok{\textless{}{-}} \FunctionTok{delete.edges}\NormalTok{(grime\_08, }\FunctionTok{E}\NormalTok{(grime\_08)[}\FunctionTok{which\_loop}\NormalTok{(grime\_08)])}
\end{Highlighting}
\end{Shaded}

\section{The Process}\label{the-process}

Let's get familiar with how to perform community detection. In this
section, I will be using the Louvain algorithm since it is one of the
more commonly known and comprehensible. To do so, use the
cluster\_lourvain() command. Please note, your network may not work with
Louvain because it only works with undirected networks (hence why I
brought the Grime network in as undirected).

I strongly recommend putting that information into an object. Then we
can take a look at what is has in it.

\begin{Shaded}
\begin{Highlighting}[]
\NormalTok{louv }\OtherTok{\textless{}{-}} \FunctionTok{cluster\_louvain}\NormalTok{(grime\_08\_clean)}

\NormalTok{louv}
\end{Highlighting}
\end{Shaded}

\begin{verbatim}
IGRAPH clustering multi level, groups: 17, mod: 0.61
+ groups:
  $`1`
  [1] "Asher D"        "Scorcher"       "Flowdan"        "Tinchy Stryder"
  [5] "Frisco"         "Kano"           "Wiley"          "Lauren Mason"  
  
  $`2`
  [1] "Dizzee Rascal"
  
  $`3`
  [1] "Lethal Bizzle"
  
  + ... omitted several groups/vertices
\end{verbatim}

\subsection{Understanding the Metrics}\label{understanding-the-metrics}

The two main things you will want to take from these algorithms are the
modularity and the membership. Modularity is a score of how separated,
or modular, the network is indicating how cohesive the groups are
compared to the whole network. Put simply, modularity measures the
density of each group compared to the density inside the group. The
statistic is measured from -1/2 to +1 with metrics closer to 1
indicating higher modularity. It is a measurement that compares what we
expect to see if the graph were completely random with what is actually
observed. Use modularity().

Meanwhile, membership is a vector showing which group each node is
classed in. This will come in very handy for you if you want to export
this information or use it for visualisations. Use membership().

\begin{Shaded}
\begin{Highlighting}[]
\FunctionTok{modularity}\NormalTok{(louv)}
\end{Highlighting}
\end{Shaded}

\begin{verbatim}
[1] 0.6084184
\end{verbatim}

\begin{Shaded}
\begin{Highlighting}[]
\FunctionTok{membership}\NormalTok{(louv)}
\end{Highlighting}
\end{Shaded}

\begin{verbatim}
        Asher D   Dizzee Rascal   Lethal Bizzle        Scorcher     Bless Beats 
              1               2               3               1               4 
        Flowdan  Tinchy Stryder          Frisco            Kano     Treble Clef 
              1               1               1               1               5 
        Shystie          Blacks         Badness         Chronik         Tempa T 
              6               7               7               7               7 
Newham Generals          Skepta             JME            Chip             BBK 
              7               7               7               8               4 
Virus Syndicate          Ghetts        Mercston        Double S        Griminal 
              9              10              11               8               8 
        Ice Kid Nu Brand Flexxx       Wretch 32           Wiley  Bossman Birdie 
              8              12               8               1              13 
    The Streets            Sway    Tinie Tempah           Giggs          Jammer 
             14              15              16              17               7 
      Roll Deep          Devlin        P Money     Lauren Mason     Milli Major 
              4              10               8               1              13 
\end{verbatim}

\subsection{Visualisation Tips for community
detection}\label{visualisation-tips-for-community-detection}

There are two main ways to visualise communities in the network. First
is to change the colours of the nodes to match the community they are
in. To do this, you utilise the vertex.color() argument in the plot()
command

\begin{Shaded}
\begin{Highlighting}[]
\FunctionTok{par}\NormalTok{(}\AttributeTok{mar =} \FunctionTok{c}\NormalTok{(}\DecValTok{0}\NormalTok{,}\DecValTok{0}\NormalTok{,}\DecValTok{0}\NormalTok{,}\DecValTok{0}\NormalTok{))}
\FunctionTok{plot}\NormalTok{(grime\_08\_clean, }\AttributeTok{vertex.color =}\NormalTok{ louv}\SpecialCharTok{$}\NormalTok{membership, }\AttributeTok{vertex.label =} \ConstantTok{NA}\NormalTok{)}
\end{Highlighting}
\end{Shaded}

\pandocbounded{\includegraphics[keepaspectratio]{Communities-in-Networks_files/figure-pdf/unnamed-chunk-5-1.pdf}}

Second, you can use polygons to demonstrate the boundaries of the
communities. To do this, you plot the clustering object alongside the
graph object.

\begin{Shaded}
\begin{Highlighting}[]
\FunctionTok{par}\NormalTok{(}\AttributeTok{mar =} \FunctionTok{c}\NormalTok{(}\DecValTok{0}\NormalTok{,}\DecValTok{0}\NormalTok{,}\DecValTok{0}\NormalTok{,}\DecValTok{0}\NormalTok{))}
\FunctionTok{plot}\NormalTok{(louv, grime\_08\_clean, }\AttributeTok{vertex.label =} \ConstantTok{NA}\NormalTok{)}
\end{Highlighting}
\end{Shaded}

\pandocbounded{\includegraphics[keepaspectratio]{Communities-in-Networks_files/figure-pdf/unnamed-chunk-6-1.pdf}}

Notice the red edges? Good! This visualisation demonstrates nodes that
bridge across communities.

Another thing to notice is that some of the isolates share colours with
others in the network. This is because Rstudio only uses a set number of
colours by default. It may be misleading in your report if you include
isolates in your network since it appears as if they are in the same
community. This is false!!! You may want to clean your network a bit
more when presenting these visuals, then.

\subsection{Clean your graph}\label{clean-your-graph}

One main thing you need to think about when detecting communities in
your graph is its structure/composition. How your graph is structured
might strongly impact the findings you get from your community
detection.

IN 2008, there were a lot og Grime artists who did not collaborate with
anyone else (isolates). If we were to run a community detection
algorithm on the graph with all the isolates still in the graph, they
would be included in the algorithms mathematics.

For example, I am using going to re-run the analysis I did above using
the Louvain algorithm with and without the isolates and you will see
what a difference it makes visually. To do this, I will delete the
isolates from the network and the plot it.

\begin{Shaded}
\begin{Highlighting}[]
\NormalTok{grime\_isol }\OtherTok{\textless{}{-}}\FunctionTok{delete\_vertices}\NormalTok{(grime\_08\_clean, }\FunctionTok{which}\NormalTok{(}\FunctionTok{degree}\NormalTok{(grime\_08\_clean)}\SpecialCharTok{==}\DecValTok{0}\NormalTok{))}
\FunctionTok{par}\NormalTok{(}\AttributeTok{mar =} \FunctionTok{c}\NormalTok{(}\DecValTok{0}\NormalTok{,}\DecValTok{0}\NormalTok{,}\DecValTok{0}\NormalTok{,}\DecValTok{0}\NormalTok{))}
\FunctionTok{plot}\NormalTok{(grime\_isol)}
\end{Highlighting}
\end{Shaded}

\pandocbounded{\includegraphics[keepaspectratio]{Communities-in-Networks_files/figure-pdf/unnamed-chunk-7-1.pdf}}

Now take the louvain on this network and take a look at it.

\begin{Shaded}
\begin{Highlighting}[]
\NormalTok{louv\_isol }\OtherTok{\textless{}{-}} \FunctionTok{cluster\_louvain}\NormalTok{(grime\_isol)}

\NormalTok{louv\_isol}
\end{Highlighting}
\end{Shaded}

\begin{verbatim}
IGRAPH clustering multi level, groups: 6, mod: 0.61
+ groups:
  $`1`
  [1] "Asher D"        "Scorcher"       "Flowdan"        "Tinchy Stryder"
  [5] "Frisco"         "Kano"           "Wiley"          "Lauren Mason"  
  
  $`2`
  [1] "Bless Beats" "BBK"         "Roll Deep"  
  
  $`3`
  [1] "Blacks"          "Badness"         "Chronik"         "Tempa T"        
  [5] "Newham Generals" "Skepta"          "JME"             "Jammer"         
  + ... omitted several groups/vertices
\end{verbatim}

The number of groups has reduced from 17 to 6. This shows that the first
algorithm counted the isolates as groups. This becomes even more
apparent when we plot the network.

Notice, however, that modularity does not change.

\begin{Shaded}
\begin{Highlighting}[]
\FunctionTok{par}\NormalTok{(}\AttributeTok{mfrow =} \FunctionTok{c}\NormalTok{(}\DecValTok{1}\NormalTok{, }\DecValTok{2}\NormalTok{))}
\FunctionTok{par}\NormalTok{(}\AttributeTok{mar =}\FunctionTok{c}\NormalTok{(}\DecValTok{0}\NormalTok{,}\DecValTok{0}\NormalTok{,}\DecValTok{0}\NormalTok{,}\DecValTok{0}\NormalTok{))}
\FunctionTok{set.seed}\NormalTok{(}\DecValTok{123}\NormalTok{)}
\FunctionTok{plot}\NormalTok{(louv\_isol, grime\_isol, }\AttributeTok{vertex.label =} \ConstantTok{NA}\NormalTok{)}
\FunctionTok{set.seed}\NormalTok{(}\DecValTok{123}\NormalTok{)}
\FunctionTok{plot}\NormalTok{(grime\_isol, }\AttributeTok{vertex.color =}\NormalTok{ louv\_isol}\SpecialCharTok{$}\NormalTok{membership, }\AttributeTok{vertex.label =} \ConstantTok{NA}\NormalTok{)}
\end{Highlighting}
\end{Shaded}

\pandocbounded{\includegraphics[keepaspectratio]{Communities-in-Networks_files/figure-pdf/unnamed-chunk-9-1.pdf}}

Now we do not have the isolates in, it is a much less confusing (noisy)
visualisation.

\section{Comparing algorithms}\label{comparing-algorithms}

As a researcher, you may want to try different algorithms to determine
which tells the story of your network. This section shows you four
different algorithms. As the scientist, the onus is on you to ensure you
understand what the algorithms do and why they may produce slightly
different results.

\begin{Shaded}
\begin{Highlighting}[]
\NormalTok{wt }\OtherTok{\textless{}{-}} \FunctionTok{cluster\_walktrap}\NormalTok{(grime\_isol)}
\NormalTok{le }\OtherTok{\textless{}{-}} \FunctionTok{cluster\_leading\_eigen}\NormalTok{(grime\_isol)}
\NormalTok{edge }\OtherTok{\textless{}{-}} \FunctionTok{cluster\_edge\_betweenness}\NormalTok{(grime\_isol)}

\FunctionTok{par}\NormalTok{(}\AttributeTok{mfrow =} \FunctionTok{c}\NormalTok{(}\DecValTok{2}\NormalTok{, }\DecValTok{2}\NormalTok{))}
\FunctionTok{par}\NormalTok{(}\AttributeTok{mar =}\FunctionTok{c}\NormalTok{(}\DecValTok{0}\NormalTok{,}\DecValTok{0}\NormalTok{,}\DecValTok{3}\NormalTok{,}\DecValTok{0}\NormalTok{))}
\FunctionTok{set.seed}\NormalTok{(}\DecValTok{123}\NormalTok{)}
\FunctionTok{plot}\NormalTok{(louv\_isol, grime\_isol, }\AttributeTok{main =} \StringTok{"Louvain"}\NormalTok{, }\AttributeTok{vertex.label =} \ConstantTok{NA}\NormalTok{)}
\FunctionTok{set.seed}\NormalTok{(}\DecValTok{123}\NormalTok{)}
\FunctionTok{plot}\NormalTok{(edge, grime\_isol, }\AttributeTok{main =} \StringTok{"Edge Betweenness"}\NormalTok{, }\AttributeTok{vertex.label =} \ConstantTok{NA}\NormalTok{)}
\FunctionTok{set.seed}\NormalTok{(}\DecValTok{123}\NormalTok{)}
\FunctionTok{plot}\NormalTok{(wt, grime\_isol, }\AttributeTok{main =} \StringTok{"Walktrap"}\NormalTok{, }\AttributeTok{vertex.label =} \ConstantTok{NA}\NormalTok{)}
\FunctionTok{set.seed}\NormalTok{(}\DecValTok{123}\NormalTok{)}
\FunctionTok{plot}\NormalTok{(le, grime\_isol, }\AttributeTok{main =} \StringTok{"Leading Eigenvector"}\NormalTok{, }\AttributeTok{vertex.label =} \ConstantTok{NA}\NormalTok{)}
\end{Highlighting}
\end{Shaded}

\pandocbounded{\includegraphics[keepaspectratio]{Communities-in-Networks_files/figure-pdf/unnamed-chunk-10-1.pdf}}

In this case, we have consensus across multiple algorithms. This is
rare, but should build more confidence in your analysis.

\section{Analysing the communities.}\label{analysing-the-communities.}

One thing you can do with the community detection is to describe their
nature.A basic example is just to take a look at their characteristics
like average the nodal degree in each community to see if certain groups
have higher degree than others. In the following chunk, I make a data
frame in an object called node data. This data frame has the membership
from the louvain algorithm and the nodes' degree. Then, I present a
variable (not saved in the dataframe) called mean\_degree which presents
the mean degree of each community.

\begin{Shaded}
\begin{Highlighting}[]
\NormalTok{node\_data }\OtherTok{\textless{}{-}} \FunctionTok{data.frame}\NormalTok{(}
  \AttributeTok{deg =} \FunctionTok{degree}\NormalTok{(grime\_isol),}
  \AttributeTok{subgroup =}\NormalTok{ louv\_isol}\SpecialCharTok{$}\NormalTok{membership}
\NormalTok{)}

\NormalTok{node\_data }\SpecialCharTok{\%\textgreater{}\%}
     \FunctionTok{group\_by}\NormalTok{(subgroup) }\SpecialCharTok{\%\textgreater{}\%}
     \FunctionTok{summarise}\NormalTok{(}\AttributeTok{mean\_degree =} \FunctionTok{mean}\NormalTok{(deg, }\AttributeTok{na.rm =} \ConstantTok{TRUE}\NormalTok{))}
\end{Highlighting}
\end{Shaded}

\begin{verbatim}
# A tibble: 6 x 2
  subgroup mean_degree
     <dbl>       <dbl>
1        1        2.12
2        2        1.67
3        3        2.12
4        4        2.17
5        5        1   
6        6        1   
\end{verbatim}

What does this new table tell you about each community?

You can do many different descriptive analyses of these communities.
Let's say you have some node-level characteristics like their gender.
you can examine the percentage of men/women in each community to see if
gender may be associated with one group more than another.

\section{Final Thoughts}\label{final-thoughts}

You need to remember that there are a lot of algorithms that can be
used. Each algorithm identifies groups within your network based on a
certain characteristic. For example, some algorithms like Louvain seek
to maximise the modularity of groups finding the communities that are
more densely connected to each other compared to the network as a whole.
Meanwhile, Walktrap uses random walks across the graph to determine
which nodes occur more frequently together on each others random walk.
The co-occurrence of nodes across random walks indicates that they are
likely in the same community.

So, you need to be cautious when selecting what community detection
algorithm you are going to use and understand them. When reporting, you
will want to report your findings as they relate to the community
detection algorithm that you are using otherwise your results could be
misleading. For example, if you use the Walktrap but report that the
communities are more densely connected to each other than the whole
network (clearly a Louvain-related explanation) this may or may not
actually be true because Walktrap is not directly measuring community
vs.~network density.

The tendency for researchers is to try multiple algorithms and find one
that either produces the nicest visualisation, or produces the highest
metrics. I strongly recommend not doing this but rather thinking deeply
about why there might be differences across measurements. This in itself
could be a finding!

\chapter{Networks of People}\label{networks-of-people}

\begin{Shaded}
\begin{Highlighting}[]
\FunctionTok{library}\NormalTok{(igraph)}
\end{Highlighting}
\end{Shaded}

There are two elements to analyse networks, the nodes and the networks.
In this chapter, we will cover some of the fundamental network measures.
As usual, we won't get too technical into the mathematics behind each
measure, but focus on the substantive interpretations.

What is a network measure, then? Rather than describing an individual in
the group, we can describe characteristics of the group itself based on
the relationships that exist between the individuals in the group.
Network measures focus on different elements of the network and tell
slightly different stories depending on the measure. Some may indicate
how interactive the group is (i.e.~how much people interact) while
others may indicate the extent to which ties in the group are mutual or
unrequited.

There are three major uses for network measures. First, the story that
these measures tell us data scientists can help us compile a story about
the group we are interested in. By combining a number of these measures,
we can produce a robust picture of what interaction looks like in these
groups. Second, measuring the characteristics of a network can provide
context for further study. For example, let's say we are actually
interested in a specific individual in a network, we may first want to
understand what the whole group looks like before we measure them.
Third, we may have multiple groups that we want to compare, or the same
group over time. For example, let's say we wanted to explore student
connections in a classroom over time. We may collect network data at
different time points in the semester. The measurements we take from
those time point could help us answer how student interactions shift and
change throughout the semester. Alternatively, we may wish to examine
student interactions across different types of classes (i.e.~large
lecture vs.~small seminar classes). If we collect network data across
types of classes, we could compare network measurements to see if
student connections are more or less dense in certain types of classes
than others. Cool right?

Here, I will introduce you to some classic network measures using the
2008 Grime data that you are all familiar with now. Then, I will bring
in more Grime data and demonstrate the types of stories we can tell
about this group over time.

Let's bring in our usual 2008 Grime data and then clean it up a bit
(delete the selfloops).

\begin{Shaded}
\begin{Highlighting}[]
\NormalTok{grime\_edge\_list }\OtherTok{\textless{}{-}} \FunctionTok{read.csv}\NormalTok{(}\FunctionTok{file.choose}\NormalTok{(), }\AttributeTok{header =} \ConstantTok{TRUE}\NormalTok{)}

\NormalTok{grime\_08 }\OtherTok{\textless{}{-}} \FunctionTok{graph\_from\_data\_frame}\NormalTok{(}\AttributeTok{d=}\NormalTok{ grime\_edge\_list, }\AttributeTok{directed =} \ConstantTok{TRUE}\NormalTok{)}

\NormalTok{grime\_08\_clean }\OtherTok{\textless{}{-}} \FunctionTok{delete.edges}\NormalTok{(grime\_08, }\FunctionTok{E}\NormalTok{(grime\_08)[}\FunctionTok{which\_loop}\NormalTok{(grime\_08)])}

\FunctionTok{plot}\NormalTok{(grime\_08\_clean, }\AttributeTok{main =} \StringTok{"Grime Collaborations}\SpecialCharTok{\textbackslash{}n}\StringTok{2008"}\NormalTok{)}
\end{Highlighting}
\end{Shaded}

\pandocbounded{\includegraphics[keepaspectratio]{Networks-of-People_files/figure-pdf/unnamed-chunk-2-1.pdf}}

\section{Density}\label{density}

The first measurement we will cover is density. A graphs' density
relates to the number of possible ties compared to the number of
observed ties. In a nutshell, it a measurement that shows the proportion
of observed ties compared to the number of ties possible in the network.
A fully dense network suggests that everyone who could be connected is.
Meanwhile, a very sparse network suggests that very few people are
connected. We use density as a measure of interactivity or connectivity
in a group.

The basic maths is a ratio where: Density Ratio = observed edges /
number of possible edges. Bare in mind that an undirected graph has half
the number of possible ties that a directed version of the same graph
has.

Density is measured from 0 to 1 and can be interpreted as the proportion
of observed ties in the network compared to the total possible ties. It
divides the number of observed ties by the number of possible ties in a
graph. A network with a density of 1 has all possible ties while a
density of 0 has none of the possible ties. Be mindful here, direction
matters here. If a graph is directed then there are twice as many
possible ties than in an undirected network. So, you may see a network
that looks well connected but may have a low density. This may be
because it is a directed network with a low reciprocity (i.e.~there may
not be many of the possible return ties).

\begin{Shaded}
\begin{Highlighting}[]
\NormalTok{  dense }\OtherTok{\textless{}{-}} \FunctionTok{edge\_density}\NormalTok{(grime\_08\_clean)}
\CommentTok{\#View it}
\NormalTok{dense}
\end{Highlighting}
\end{Shaded}

\begin{verbatim}
[1] 0.01794872
\end{verbatim}

\section{Reciprocity}\label{reciprocity}

Reciprocity is a measure of how balanced a network is. Another way of
thinking about reciprocity is in terms of tie symmetry. It is a measure
of graph symmetry or the mutual connections between individuals in the
graph. In directed graphs edges between nodes are not necessarily
mutual. One person may send something to another but not receive
something in return. Reciprocity is a way to measure the proportion of
mutual vs.~unrequited connections. The most basic measure of reciprocity
is the ratio. It observes the number of mutual ties (A-\textgreater B,
B-\textgreater A) divided by the total number of edges.

The basic maths behind it is as follows: Reciprocity ratio = number of
mutual dyads(observed) / Total Number of edges(possible)

This provides a scale between 0 and 1 with 1 indicating a perfectly
reciprocal network. When measuring reciprocity, I always ignore loops
because, in my mind, you cannot send and receive something from
yourself. This, however, could be a network-specific thing. I struggle
to think of any networks where loops are particularly useful.

\begin{Shaded}
\begin{Highlighting}[]
\NormalTok{recip }\OtherTok{\textless{}{-}} \FunctionTok{reciprocity}\NormalTok{(grime\_08\_clean, }\AttributeTok{ignore.loops =} \ConstantTok{TRUE}\NormalTok{, }\AttributeTok{mode =} \StringTok{"ratio"}\NormalTok{)}
\CommentTok{\#View it}
\NormalTok{recip}
\end{Highlighting}
\end{Shaded}

\begin{verbatim}
[1] 0.03703704
\end{verbatim}

\section{Transitivity}\label{transitivity}

Transitivity, also known as the clustering coefficient, is a measure of
how tightly knit a network is. At its core, transitivity captures the
tendency for triangles to form within a network --- in other words, how
likely it is that a node's neighbors are also connected to each other.
If person A is connected to both B and C, then in a highly transitive
network, it is likely that B and C are also connected, forming a
triangle. This reflects the idea that ``a friend of a friend is also a
friend.''

While density measures how connected a network is at the global level
--- looking at how many ties exist relative to how many could possibly
exist --- transitivity focuses more on local structure. It considers the
relationships among a node's immediate neighbors, revealing how
clustered those connections are.

Transitivity is typically measured as the proportion of observed closed
triplets (three nodes all connected to each other, forming a triangle)
to the total number of connected triplets (any set of three nodes where
at least two connections exist --- whether or not they form a full
triangle). The formula multiplies the number of triangles by three
because each triangle contains three such triplets, one centered at each
node.

The transitivity score ranges between 0 and 1. A score of 1 indicates
that all possible triangles are present --- every node's neighbors are
fully connected to one another. A score of 0 means there are no such
triangles, and the network has no clustering. In this way, transitivity
gives a sense of how cohesive the local connections are within the
broader network.

\begin{Shaded}
\begin{Highlighting}[]
\NormalTok{transitivity }\OtherTok{\textless{}{-}} \FunctionTok{transitivity}\NormalTok{(grime\_08\_clean)}
\CommentTok{\#View it}
\NormalTok{transitivity}
\end{Highlighting}
\end{Shaded}

\begin{verbatim}
[1] 0
\end{verbatim}

\section{Centraisation}\label{centraisation}

The final measure I want to introduce you to is called centralisation.
We have covered measures of centrality which are concerned with
individuals. Measures of centralisation use similar logic but describe
the graph. They capture the extent to which the network is centred
around one or a few individuals. You can well imagine that a star graph
(a graph where all pendula are connected to a central node) is highly
centralised. Meanwhile, a full graph may be less centrlaised.

There are different measures of centralization that use nodal centrality
to describe the network. For example, centralization.degree() measures
the extent to which the network is centered around highly connected
nodes (those with many neighbors), while centralization.betweenness()
measures the extent to which the network relies on influential nodes
(those that control the flow of information between other nodes).

Measures of centralisation have a range of 0- 1 (starting t see a theme
here - many of these network measures are proportions). Degree
centralisation score of 0 means that all the nodes have the same degree
centrality (i.e.~the grpah is completely uniform). A score of 1 means
that one node in the network has the maximum possible degree (i.e.~all
others are connected to them and only them). Similar logic applies to
other forms of centralisation. I will show you only degree, and
betweenness but you could also try closeness and eigenvector.

These functions give you three different elements. The \$res
demonstrates the scores (degree, between or closeness) for each node in
the network. The \$centralisation item is the scaled score (the
proportion). This is the one we care most about. The \$theoretical\_max
shows the raw, unscaled, theoretical max value for the centralisation
score. If you are pulling many scores together to compare them, you may
want to pull out just the \$centralization score.

\begin{Shaded}
\begin{Highlighting}[]
\FunctionTok{centr\_degree}\NormalTok{(grime\_08\_clean)}
\end{Highlighting}
\end{Shaded}

\begin{verbatim}
$res
 [1]  1  0  0  1  2  2  1  2  1  0  0  1  1  1  1  1  1  1  4  1  0  1  0  1  3
[26]  2  0  2  8  1  0  0  0  0 10  2  1  1  1  1

$centralization
[1] 0.1130835

$theoretical_max
[1] 3042
\end{verbatim}

\begin{Shaded}
\begin{Highlighting}[]
\FunctionTok{centr\_betw}\NormalTok{(grime\_08\_clean)}
\end{Highlighting}
\end{Shaded}

\begin{verbatim}
$res
 [1] 0 0 0 0 0 0 0 0 0 0 0 0 0 0 0 0 0 0 4 0 0 0 0 0 2 0 0 0 7 0 0 0 0 0 0 0 0 0
[39] 0 0

$centralization
[1] 0.004619537

$theoretical_max
[1] 57798
\end{verbatim}

I want you to pause for a moment and think about the network we are
analysing. It has isolates. How might they be impacting these measures?
You may consider removing the isolates for this type of analysis.
REMEMBER, though, you MUST talk about your measurements related to what
you are measuring. You can't remove isolates and say that this group has
a degree centralisation score of\ldots{} which means they rely highly on
one person\ldots.that would be wrong. You would have to say that you
removed those without connections to obtain that score. Therefore,
you're actually talking about those with connections to others, not the
whole group. You as the scientist must make the decision!

Take a look at the following once I remove the isolates.

\begin{Shaded}
\begin{Highlighting}[]
\NormalTok{grime\_08\_isol }\OtherTok{\textless{}{-}} \FunctionTok{delete.vertices}\NormalTok{(grime\_08\_clean, }\FunctionTok{which}\NormalTok{(}\FunctionTok{degree}\NormalTok{(grime\_08\_clean) }\SpecialCharTok{==} \DecValTok{0}\NormalTok{)) }

\FunctionTok{centr\_degree}\NormalTok{(grime\_08\_isol)}
\end{Highlighting}
\end{Shaded}

\begin{verbatim}
$res
 [1]  1  1  2  2  1  2  1  1  1  1  1  1  1  1  4  1  1  1  3  2  2  8  1 10  2
[26]  1  1  1  1

$centralization
[1] 0.1492347

$theoretical_max
[1] 1568
\end{verbatim}

\begin{Shaded}
\begin{Highlighting}[]
\FunctionTok{centr\_betw}\NormalTok{(grime\_08\_isol)}
\end{Highlighting}
\end{Shaded}

\begin{verbatim}
$res
 [1] 0 0 0 0 0 0 0 0 0 0 0 0 0 0 4 0 0 0 2 0 0 7 0 0 0 0 0 0 0

$centralization
[1] 0.008975813

$theoretical_max
[1] 21168
\end{verbatim}

The values have changed!Both have increased suggesting that the isolates
were impacting the original measures. Again, your decision as to whether
you keep or remove isolates.

\section{Example}\label{example}

Right, let's have some fun with this. I am going to read in some more
Grime network years.

\begin{Shaded}
\begin{Highlighting}[]
\NormalTok{grime\_2010 }\OtherTok{\textless{}{-}} \FunctionTok{read.csv}\NormalTok{(}\FunctionTok{file.choose}\NormalTok{(), }\AttributeTok{header =} \ConstantTok{TRUE}\NormalTok{)}
\NormalTok{grime\_2012 }\OtherTok{\textless{}{-}} \FunctionTok{read.csv}\NormalTok{(}\FunctionTok{file.choose}\NormalTok{(), }\AttributeTok{header =} \ConstantTok{TRUE}\NormalTok{)}
\NormalTok{grime\_2014 }\OtherTok{\textless{}{-}} \FunctionTok{read.csv}\NormalTok{(}\FunctionTok{file.choose}\NormalTok{(), }\AttributeTok{header =} \ConstantTok{TRUE}\NormalTok{)}
\NormalTok{grime\_2016 }\OtherTok{\textless{}{-}} \FunctionTok{read.csv}\NormalTok{(}\FunctionTok{file.choose}\NormalTok{(), }\AttributeTok{header =} \ConstantTok{TRUE}\NormalTok{)}
\CommentTok{\#cleaning the loops}
\NormalTok{grime\_2010\_edge }\OtherTok{\textless{}{-}} \FunctionTok{graph\_from\_data\_frame}\NormalTok{(}\AttributeTok{d=}\NormalTok{ grime\_2010, }\AttributeTok{directed =} \ConstantTok{TRUE}\NormalTok{)}
\NormalTok{grime\_2010\_edge }\OtherTok{\textless{}{-}} \FunctionTok{delete.edges}\NormalTok{(grime\_2010\_edge, }\FunctionTok{E}\NormalTok{(grime\_2010\_edge)[}\FunctionTok{which\_loop}\NormalTok{(grime\_2010\_edge)])}

\NormalTok{grime\_2012\_edge }\OtherTok{\textless{}{-}} \FunctionTok{graph\_from\_data\_frame}\NormalTok{(}\AttributeTok{d=}\NormalTok{ grime\_2012, }\AttributeTok{directed =} \ConstantTok{TRUE}\NormalTok{)}
\NormalTok{grime\_2012\_edge }\OtherTok{\textless{}{-}} \FunctionTok{delete.edges}\NormalTok{(grime\_2012\_edge, }\FunctionTok{E}\NormalTok{(grime\_2012\_edge)[}\FunctionTok{which\_loop}\NormalTok{(grime\_2012\_edge)])}

\NormalTok{grime\_2014\_edge }\OtherTok{\textless{}{-}} \FunctionTok{graph\_from\_data\_frame}\NormalTok{(}\AttributeTok{d=}\NormalTok{ grime\_2014, }\AttributeTok{directed =} \ConstantTok{TRUE}\NormalTok{)}
\NormalTok{grime\_2014\_edge }\OtherTok{\textless{}{-}} \FunctionTok{delete.edges}\NormalTok{(grime\_2014\_edge, }\FunctionTok{E}\NormalTok{(grime\_2014\_edge)[}\FunctionTok{which\_loop}\NormalTok{(grime\_2014\_edge)])}

\NormalTok{grime\_2016\_edge }\OtherTok{\textless{}{-}} \FunctionTok{graph\_from\_data\_frame}\NormalTok{(}\AttributeTok{d=}\NormalTok{ grime\_2016, }\AttributeTok{directed =} \ConstantTok{TRUE}\NormalTok{)}
\NormalTok{grime\_2016\_edge }\OtherTok{\textless{}{-}} \FunctionTok{delete.edges}\NormalTok{(grime\_2016\_edge, }\FunctionTok{E}\NormalTok{(grime\_2016\_edge)[}\FunctionTok{which\_loop}\NormalTok{(grime\_2016\_edge)])}
\end{Highlighting}
\end{Shaded}

Let's get the density of each of these networks so we can compare them.
I am going to create objects based on the time stamps with T1, T2.. etc.

\begin{Shaded}
\begin{Highlighting}[]
\NormalTok{dense\_T1 }\OtherTok{\textless{}{-}} \FunctionTok{edge\_density}\NormalTok{(grime\_08\_clean)}
\NormalTok{dense\_T2 }\OtherTok{\textless{}{-}} \FunctionTok{edge\_density}\NormalTok{(grime\_2010\_edge)}
\NormalTok{dense\_T3 }\OtherTok{\textless{}{-}} \FunctionTok{edge\_density}\NormalTok{(grime\_2012\_edge)}
\NormalTok{dense\_T4 }\OtherTok{\textless{}{-}} \FunctionTok{edge\_density}\NormalTok{(grime\_2014\_edge)}
\NormalTok{dense\_T5 }\OtherTok{\textless{}{-}} \FunctionTok{edge\_density}\NormalTok{(grime\_2016\_edge)}
\end{Highlighting}
\end{Shaded}

Now we are going to make some data frames and then bind them together to
create a large dataset with these density measures.

\begin{Shaded}
\begin{Highlighting}[]
\NormalTok{t1 }\OtherTok{\textless{}{-}} \FunctionTok{data.frame}\NormalTok{(}
\AttributeTok{density =}\NormalTok{ dense\_T1,}
\AttributeTok{year =} \DecValTok{2008}\NormalTok{,}
\AttributeTok{id =} \DecValTok{1}
\NormalTok{)}
\NormalTok{t2 }\OtherTok{\textless{}{-}} \FunctionTok{data.frame}\NormalTok{(}
\AttributeTok{density =}\NormalTok{ dense\_T2,}
\AttributeTok{year =} \DecValTok{2010}\NormalTok{,}
\AttributeTok{id =} \DecValTok{2}
\NormalTok{)}
\NormalTok{t3 }\OtherTok{\textless{}{-}} \FunctionTok{data.frame}\NormalTok{(}
\AttributeTok{density =}\NormalTok{ dense\_T3,}
\AttributeTok{year =} \DecValTok{2012}\NormalTok{,}
\AttributeTok{id =} \DecValTok{3}
\NormalTok{)}
\NormalTok{t4 }\OtherTok{\textless{}{-}} \FunctionTok{data.frame}\NormalTok{(}
\AttributeTok{density =}\NormalTok{ dense\_T4,}
\AttributeTok{year =} \DecValTok{2014}\NormalTok{,}
\AttributeTok{id =} \DecValTok{4}
\NormalTok{)}
\NormalTok{t5 }\OtherTok{\textless{}{-}} \FunctionTok{data.frame}\NormalTok{(}
\AttributeTok{density =}\NormalTok{ dense\_T5,}
\AttributeTok{year =} \DecValTok{2016}\NormalTok{,}
\AttributeTok{id =} \DecValTok{5}
\NormalTok{)}

\NormalTok{density\_graph }\OtherTok{\textless{}{-}} \FunctionTok{rbind}\NormalTok{(t1, t2, t3, t4, t5)}
\FunctionTok{print}\NormalTok{(density\_graph)}
\end{Highlighting}
\end{Shaded}

\begin{verbatim}
      density year id
1 0.017948718 2008  1
2 0.014509804 2010  2
3 0.010802469 2012  3
4 0.009697326 2014  4
5 0.008009153 2016  5
\end{verbatim}

We are all good to go! What I am going to do is present this information
in a simple line graph of how the density of the plot changes over time.
To do this, we will need another package - ggplot2.

\begin{Shaded}
\begin{Highlighting}[]
\FunctionTok{library}\NormalTok{(ggplot2)}

\FunctionTok{ggplot}\NormalTok{(density\_graph, }\FunctionTok{aes}\NormalTok{(}\AttributeTok{x =}\NormalTok{ year, }\AttributeTok{y =}\NormalTok{ density)) }\SpecialCharTok{+} \FunctionTok{geom\_line}\NormalTok{(}\AttributeTok{linewidth =} \DecValTok{2}\NormalTok{)}\SpecialCharTok{+} \FunctionTok{ylab}\NormalTok{(}\StringTok{"Network Density"}\NormalTok{) }\SpecialCharTok{+} \FunctionTok{xlab}\NormalTok{(}\StringTok{"Year"}\NormalTok{)}
\end{Highlighting}
\end{Shaded}

\pandocbounded{\includegraphics[keepaspectratio]{Networks-of-People_files/figure-pdf/unnamed-chunk-11-1.pdf}}

What can you see in this graph? Really, the story is that Grime is quite
sparse. At T1 it is sparse and it continues to get more sparse as the
years continue. This gives you an idea of what types of things you can
do with all these measures of networks of people.

\part{Assignments}

\chapter{A1 - My Ego Network}\label{a1---my-ego-network}

Instructions: Follow the steps below and submit either a photograph or
document to Moodle assignment portal under Module 1. YOU MUST NAME YOUR
FILE LIKE THIS: course\_section\_semester\_assignment\_unityID For
example: DSA295\_003\_SP25\_A1\_trleppar

Grading: Full marks are awarded for completion and proper naming. One
mark dropped if not named properly, or for each portion incomplete.

\section{Assignment:}\label{assignment}

You are surrounded by networks. You participate in them and what you do
can reinforce them. Here is an activity to exemplify what networks look
like. We will focus on social networks -- specifically, one of your
social networks.

\subsection{Part 1. Visualising your
network}\label{part-1.-visualising-your-network}

\begin{enumerate}
\def\labelenumi{\Alph{enumi}.}
\tightlist
\item
  On a piece of paper write/type your name near the bottom.
\item
  In the middle of the paper (horizontally left to right), list the
  names of up to 20 people you texted within the last week (consider
  putting a circle around each name). If you do not text, maybe consider
  those you spoke to, direct messaged on social media, or those you
  contacted somehow (up to 20)
\item
  Now draw a line/place arrows (if in word) from your name to each of
  the people you listed.
\item
  Consider the people you listed. Are any of them connected to each
  other somehow? For those in your list that are connected, draw
  lines/place arrows going from them to their contacts/friends/family
  members. -- NOTE -- if none are connected, then please write that.
  Congratulations, you just drew a social network!!! This is called an
  ego-network. You are the ``ego'' (the one answering the questions
  about the connections between you and others). Those who you listed
  are called ``alters''. The connections you drew are called ``ties.''
\end{enumerate}

\subsection{Part 2. Let's do a quick analysis of you in your
network.}\label{part-2.-lets-do-a-quick-analysis-of-you-in-your-network.}

\begin{enumerate}
\def\labelenumi{\Alph{enumi}.}
\tightlist
\item
  Count and write the number of ties that you are connected to.\\
  This is called your degree centrality. It is a measure of how popular
  or ``well-connected'' an entity is within a network.
\end{enumerate}

\subsection{Part 3. Now let's analyse your
alters.}\label{part-3.-now-lets-analyse-your-alters.}

\begin{enumerate}
\def\labelenumi{\Alph{enumi}.}
\tightlist
\item
  For each person you listed, count the number of ties that they have to
  others. If there are none, write N/A.
\end{enumerate}

You have now calculated the degree centrality of each of the alters in
your network. Which people in your network are ``well connected?''

WELL DONE!

\chapter{A2\_Project Prospectus}\label{a2_project-prospectus}

Instructions: Follow the steps below and submit your assigment to Moodle
assignment portal under Module 2. YOU MUST NAME YOUR FILE LIKE THIS:
course\_section\_semester\_assignment\_unityID For example:
DSA295\_003\_SP25\_A2\_trleppar

Grading: Full marks are awarded for completion and proper naming. One
mark dropped if not named properly, or for each portion incomplete.

\section{Assignment:}\label{assignment-1}

The final project is to provide a big report on a network of your
choice. This report will have three parts, transformation,
visualization, and analysis. This assignment is designed to get you
thinking about your final project. Follow the steps in this assignment
and turn in a prospectus on your project.

\subsection{Part 1. Explore possible sources of
data}\label{part-1.-explore-possible-sources-of-data}

A. Visit my
\href{https://drive.google.com/drive/folders/18BRSRSLjBQcbYOiu59cWZ4fzmmK3ZoWJ}{Google
drive folder} (you will need to sign in with your (\textbf{ncsu.edu?})
account to access it. B. Look through the sources of data I have
collected for you and pick at least two. Please read the info.txt file
for each data file to understand the networks that are there. C. In your
prospectus, write a paragraph describing the networks that you are
considering using for your project. Consider: How are they structured?
What constitutes the nodes/edges? D. (Optional) If you are interested in
using your own data (i.e.~collecting your own, or finding other data
elsewhere), then discuss this IN ADDITION to selecting two potential
datasets in the folder.

\subsection{Part 2.}\label{part-2.}

A. Write a paragraph discussing ideas for visualisation and analysis
that you may do using these data. Consider what you are interested in --
communities, individuals, describing the network itself?

\subsection{Part 3.}\label{part-3.}

A. Use the code you have learned so far in class to: a. Bring your at
least one of your data choices into RStudio b. Convert it into a network
object c.~Plot your network

B. Save the image of your network and attach it to the page. (see the
picture below on how).

\begin{figure}[H]

{\centering \pandocbounded{\includegraphics[keepaspectratio]{images/Plots in R.png}}

}

\caption{A screenshot showing how to save plots in RStudio}

\end{figure}%

\chapter{A3 - Data Exploration}\label{a3---data-exploration}

\section{Instructions}\label{instructions}

This is a skeleton markdown file that you will use to complete
Assignment 3 this semester. Use the chunks (see below) to code in. Where
you see \ldots\ldots\ldots\ldots\ldots\ldots\ldots\ldots{} please remove
and write your response to the prompt.

\begin{Shaded}
\begin{Highlighting}[]
\DocumentationTok{\#\#\#\#\#\#\#\#\#\#\#\#\#\#\#\#\#\#\#\#\#\# THIS IS A CHUNK \#\#\#\#\#\#\#\#\#\#\#\#\#\#\#\#\#\#\#\#\#\#\#\#\#\#\#\#\#}
\end{Highlighting}
\end{Shaded}

As usual, follow the steps below and submit your markdown document as a
pdf. You will need to knit this to a pdf or knit it to an html then save
as a pdf to submit.

YOUR MUST NAME YOUR FILE LIKE THIS:
course\_section\_semester\_assignment\_unityID For example:
DSA295\_003\_SP25\_A2\_trleppar

Grading: Full marks are awarded for completion and proper naming. One
mark will be dropped if not named properly, and/or additional marks
dropped for each portion incomplete. This assignment will have 5 points
per portion (4 questions/prompts to complete) and a total of 20 marks.

\section{Selecting your network
data.}\label{selecting-your-network-data.}

Please select one of the datasets that are available to you for your
final project. Then, discuss the following.

\begin{enumerate}
\def\labelenumi{\arabic{enumi})}
\item
  How is this data stored? How do you know this? Discuss the
  characteristics of your network data using terms we have learned in
  class.
  \ldots\ldots\ldots\ldots\ldots\ldots\ldots\ldots\ldots\ldots\ldots\ldots\ldots\ldots\ldots\ldots\ldots.
\item
  What interested you in this network?
  \ldots\ldots\ldots\ldots\ldots\ldots\ldots\ldots\ldots\ldots\ldots\ldots\ldots\ldots\ldots\ldots\ldots.
\end{enumerate}

\section{Bringing in your network}\label{bringing-in-your-network}

Use this chunk below to bring in your network data and convert it into a
network object using the appropriate method for how your network data
are structured.

\begin{Shaded}
\begin{Highlighting}[]
\FunctionTok{library}\NormalTok{(igraph)}
\NormalTok{my\_data }\OtherTok{\textless{}{-}} 
\end{Highlighting}
\end{Shaded}

\begin{enumerate}
\def\labelenumi{\arabic{enumi})}
\setcounter{enumi}{2}
\tightlist
\item
  Please describe your network. What are the nodes? Edges? One/two mode?
  Directed/Undirected?
  \ldots\ldots\ldots\ldots\ldots\ldots\ldots\ldots\ldots\ldots\ldots\ldots\ldots\ldots\ldots\ldots\ldots.
\end{enumerate}

\section{Describing your network
data}\label{describing-your-network-data}

Use this chunk below to create a table showing the number of nodes and
edges that are in your network.

\begin{Shaded}
\begin{Highlighting}[]
\NormalTok{num\_nodes }\OtherTok{\textless{}{-}} \FunctionTok{vcount}\NormalTok{(YOUR GRAPH OBJECT HERE)}
\NormalTok{num\_edges }\OtherTok{\textless{}{-}} \FunctionTok{ecount}\NormalTok{(YOUR GRAPH OBJECT HERE)}

\NormalTok{table }\OtherTok{\textless{}{-}} \FunctionTok{data.frame}\NormalTok{( }
  \AttributeTok{nodes =}\NormalTok{ num\_nodes,}
  \AttributeTok{edges =}\NormalTok{ num\_edges}
\NormalTok{  )}
\end{Highlighting}
\end{Shaded}

Please present the table you created. (Consider using the head(),
tail(), or print() functions).

\section{Visualising the network}\label{visualising-the-network}

\begin{enumerate}
\def\labelenumi{\arabic{enumi})}
\setcounter{enumi}{3}
\tightlist
\item
  Potential Transformations Present a raw visualisation of your network
  (yes, even if it is ugly!).
\end{enumerate}

Please discuss any transformations (not just to the visual, but actual
data transformations) that you think you may need to make to this
network before you go on to analyse it (think about deleting nodes and
ties). If you can't think of anything you NEED to do (i.e.~trim or add),
then what ways do you think you could do (i.e.~subgraph)?
\ldots\ldots\ldots\ldots\ldots\ldots\ldots\ldots\ldots\ldots\ldots\ldots\ldots\ldots\ldots\ldots\ldots\ldots\ldots\ldots\ldots\ldots\ldots{}

WELL DONE!! :)

\chapter{A4 - Visualisation}\label{a4---visualisation}

\section{Instructions}\label{instructions-1}

This is a skeleton markdown file that you will use to complete
Assignment 3 this semester. Use the chunks (see below) to code in. Where
you see \ldots\ldots\ldots\ldots\ldots\ldots\ldots\ldots{} please remove
and write your response to the prompt.

\begin{Shaded}
\begin{Highlighting}[]
\DocumentationTok{\#\#\#\#\#\#\#\#\#\#\#\#\#\#\#\#\#\#\#\#\#\# THIS IS A CHUNK \#\#\#\#\#\#\#\#\#\#\#\#\#\#\#\#\#\#\#\#\#\#\#\#\#\#\#\#\#}
\end{Highlighting}
\end{Shaded}

As usual, follow the steps below and submit your markdown document as a
pdf. You will need to knit this to a pdf or knit it to an html then save
as a pdf to submit.

YOUR MUST NAME YOUR FILE LIKE THIS:
course\_section\_semester\_assignment\_unityID For example:
DSA295\_003\_SP25\_A2\_trleppar

Grading: Full marks are awarded for completion and proper naming. One
mark will be dropped if not named properly, and/or additional marks
dropped for each portion incomplete. This assignment will have 5 points
per portion (4 prompts to complete) and a total of 20 marks.

\section{Place your library() here. You will need to list each one you
will
use!}\label{place-your-library-here.-you-will-need-to-list-each-one-you-will-use}

\section{Bringing in your data}\label{bringing-in-your-data}

By this time, you're a dab hand at bringing in your network data. Please
bring in your network, convert it to a network object and view that
object. Just a reminder, to view the network, just list the name of the
network. If my igraph object (network object) is called `graph' I can
simply code graph in the chunk and I will view it.

Clean and or transform your network data if you need/want to and provide
a raw (unaltered) visualisation using the plot() function.

\section{Basic visualisation}\label{basic-visualisation}

In class, we have covered multiple basic visualizations. Using the
markdown files and your notes from our class about basic network
visualisation, I want you to demonstrate your skill in basic network
visualisation.

Please alter the labels by either changing their size or toggling them.

Change the colours of the network (nodes or edges).

Present a basic visualisation, you feel appears clean.

Discuss what alterations you have done to the network and why you did
them. \ldots\ldots\ldots\ldots\ldots\ldots\ldots\ldots\ldots\ldots{}

\section{Intermediate visualisation}\label{intermediate-visualisation}

You have learned how to tell a story through network visualisation.
These stories centre prominent and influential nodes in the network.

Present a CLEAN network visual demonstrating popular (degree) nodes in
your network.

Present a CLEAN network visual demonstrating influential (betweenness)
nodes in your network.

Present them side-by-side sing the par() function. Consider presenting
sub or main titles so I can tell which is which.

Briefly (1 - 2 sentences) discuss the stories that your visualisations
tell. E.g. who are/are they many or few popular/influential nodes?
\ldots\ldots\ldots\ldots\ldots\ldots\ldots\ldots\ldots\ldots{}

\section{Advanced visualisation}\label{advanced-visualisation}

You have learned how to create interactive, or animated network
visualisations.

Please produce one advanced network visualisation. Bare in mind that you
will need to leave the visual as a widget in this markdown.

Briefly (1 - 2 sentences) discuss this visualisation. In what contexts
may this visualisation be useful?

\chapter{WELL DONE :)}\label{well-done}

\chapter{Final Project Instructions}\label{final-project-instructions}

In this final project you will explore the data that have been working
on all semester long. The aim of this project is for you to do an
in-depth exploration which includes descriptions, visualisations, and
some primary analysis. The learning outcomes for this is to demonstrate
your skill in transforming (10 points), visualising (10 points) and
analysing (20 points) network data.

The first two parts of the assignment, the transformation and
visualisation of your data, are your exploration. Here, I want you to be
paying attention to the nature of your data and how the network is
structured. This should inform the last portion of the project - the
analysis. Based on your exlporation, what types of questions are you
asking about your data? What are some hypotheses you may have about
people in the network or the network itself? You will need to write
about these thoughts as part of your project to demonstrate your ability
to explore network data and generate ideas on how to/what to analyse.

The ``rubric'' for this assignment is like all of our others -
completion. Please fulfill all outlined prompts. Thus far, you have had
assignments on exploring, transforming and visualising your network.
Therefore, these components are worth only a small portion of the final
marks of this assignment. Here is the breakdown:

Part 1: Transforming (10 points) Part 2: Visualising (10 points) Part 3:
Analysing (20 points) Total points = 40.

As usual, make a markdown document and submit is as a pdf.

Here are some instructions for you to follow. Please refer to the
``final project example.html'' document on Moodle as an example final
project for you to draw on for inspiration.

\section{Example Layout and
Instructions}\label{example-layout-and-instructions}

\chapter{FINAL PROJECT TITLE HERE}\label{final-project-title-here}

First, open with a BRIEF paragraph discussing your project. Tell me what
there data are and a little about what you are researching about the
data. Here, you can imagine the types of questions your boss may have
asked you to analyse on these data.

\section{Part 1: Data Transformation}\label{part-1-data-transformation}

In this section you must bring in, clean and transform your network
data.

NOTE: If your network data are already clean you MUST transform the
network some how. For example, create a subgraph, delete/add nodes/edges
even if it does not make much sense. Just create a random object and do
it. The point is you demonstrate the skill. \#\# Part 2: Data
Visualisation In this section, please produce at least one basic,
intermediate, and advanced network visualisation. Beneath each one,
please describe (1-2 sentences) the visualisation you have created
discussing any observations you make about the network.

\section{Part 3:}\label{part-3}

Here, you must speculate on your explorations above. You have
transformed and visualised your network data. What questions are you now
going to ask about them? Based on your exploration, do you hypothesise
certain things about the network or the people therein?

In this you must analyse and generate a report on your network. In your
analysis, please discuss and present deliverables (tables,
visualisations etc.) about the people in the network, the network
itself, and communities within the network.

Be creative about what analysis you would like to perform. You can
combine skills here too. For example, visualise things as you analyse
people/nodes in the networks. Or, you may identify a community of highly
central people. You may then create a subgraph of a community you are
interested in. Be creative!

One final note, remember that this is a report. You need to write this
report as if I do not understand anything about networks, network data,
or R. Be clear about what you are doing. Tie together how your
exploration of the data led you to perform this analysis. For example,
maybe the visualisations suggested that the network was very fragmented,
or highly centralised. This may have led you to want to study certain
people or perform certain analyses.

Final tips: When writing your markdown, use the following header
guidelines: \# Header \#\# Subheader \#\#\# Third-level header

GOOD LUCK!

\bookmarksetup{startatroot}

\chapter{Summary}\label{summary}

Well done!

You have now learned how to wrangle, visualise and analyse network data.
I hope you have enjoyed the course.

\bookmarksetup{startatroot}

\chapter*{References}\label{references}
\addcontentsline{toc}{chapter}{References}

\markboth{References}{References}

\phantomsection\label{refs}
\begin{CSLReferences}{0}{1}
\end{CSLReferences}




\end{document}
